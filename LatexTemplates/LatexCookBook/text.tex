\documentclass[11pt]{article}
\usepackage[document]{ragged2e}
\usepackage[]{geometry}
\usepackage{fancyhdr,amsmath,amsthm,amssymb}
\usepackage[utf8]{inputenc}

\geometry{letterpaper,left=25mm,right=25mm}

%\setlength{\tabcolsep}{10pt}

%\pagestyle{fancy}

%\fancyhead[L]{\fontsize{18}{11} \selectfont Black Bean Burger} % One way to do a title with a header that looks like a title. Could be easier to just make it the title.

\newcommand{\R}{\mathbb{R}}
\newcommand{\Z}{\mathbb{Z}}
\newcommand{\Q}{\mathbb{Q}}
\newcommand{\N}{\mathbb{N}}

\begin{document}
\pagenumbering{gobble}
% Title
{\Huge \textbf{Black Bean Burger}\par}
% Some basic information
\vspace{1em}
\begin{tabular}{@{}l l l}
  \hline
  \noalign{\smallskip}
  Active Time & Total Time & Yield \\
  20 mins & 20 mins & Serves 4 (serving size: 1 patty) \\
  \noalign{\smallskip}
  \hline
\end{tabular}

% Possible text blurb
\vspace{2em}
Possible text blurb if I want.
\vspace{2em}
% Ingredients and Method

\begin{tabular}[t]{ p{5cm} }
  \textbf{Ingredients} \\
  \\
  1 (1-oz.) slice whole-grain bread. Toasted and torn into peices \\
  1/4 cup grated onion \\
  1 tablespoon chopped garlic \\
  1 1/2 teaspoons fresh lime juice \\
  3/4 teaaspoon ground cumin \\
  3/4 teaspoon kosher salt \\
  1/2 teaspoon grated lime \\
  1 (15-oz.) can unsalted black beans, rinsed and drained \\
  1/3 cup coarsely chopped walnuts \\
  1/2 teaspoon hot sauce \\
  1 large egg lightly beaten \\
  4 teaspoons olive oil \\
\end{tabular}
\begin{tabular}[t]{ p{10cm} }
  \textbf{Method} \\
  \\
  \textbf{Step 1} \hspace{3mm} Place bread in a food processor; pulse 5 times. Transfer to a bowl. \\[5mm]
  \textbf{Step 2} \hspace{3mm} Combine onion, garlic, juice, cumin, salt, rind, and beands in processor; pulse 4 to 5 times. Add bean mixture, walnuts, hot sauce, and egg to crumbs; stir well. Divide the mixture into 4 equal portions. Shape each portion into 3/4-inch-thick patty. \\[5mm]
  \textbf{Step 3} \hspace{3mm} Heat oil in a large non-stick skillet over medium-high heat. Add patties to pan; reduce heat to medium and cook 4 minutes on each side or until browned. \\
\end{tabular}

% I want a footnote here
\vspace{3em}
\begin{tabular}[b]{ p{3cm} }
  \hline
\end{tabular}
this is some text I have written in place of lorem ipsum

% ------------------------------- NEW RECIPE ----------------------------------------------------------
\newpage

% Title
{\Huge \textbf{Veggetarian Gravy}\par}
% Some basic information
\vspace{1em}
\begin{tabular}{@{}l l l}
  \hline
  \noalign{\smallskip}
  Active Time & Total Time & Yield \\
  20 mins & 20 mins & Serves 4 (serving size: 1 patty) \\
  \noalign{\smallskip}
  \hline
\end{tabular}

% Possible text blurb
\vspace{2em}
The general methodolgy is the same as any french roux. If you are familiar with that then this should be straight forward.
\vspace{2em}
% Ingredients and Method

\begin{tabular}[t]{ p{5cm} }
  \textbf{Ingredients} \\
  \\
  1/4 cup flour \\
  1/4 cup butter \\
  2 cups veggie broth \\
  soy sauce \\
\end{tabular}
\begin{tabular}[t]{ p{10cm} }
  \textbf{Method} \\
  \\
  \textbf{Step 1} \hspace{3mm} Melt butter in a sauce pan or pot. Add the flour and stir until evenly mixed. \\[5mm]
  \textbf{Step 2} \hspace{3mm} Slowly add veggie broth in small batches, stirring each time until properly incorporated. \\[5mm]
  \textbf{Step 3} \hspace{3mm} Add soy sauce for a darker colour and salty flavour.  \\
\end{tabular}

% I want a footnote here
\vspace{3em}
\begin{tabular}[b]{ p{3cm} }
  \hline
\end{tabular}
A useful recipe for a gravy that can be used in a multitude of scenarios, from a poutine to topping for mashed potatoes. Feel free to vary the ratios of the flour, veggie broth and soy sauce for thicker, more flavourful or more salty taste respectively. 

% ------------------------------- NEW RECIPE ----------------------------------------------------------

\end{document}
