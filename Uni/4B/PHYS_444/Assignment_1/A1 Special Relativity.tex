\documentclass[12pt,oneside]{book}
%==================================================================================
% metadata
\title{PHYS 444
\hfill
Assignment 1
\\
Special Relativity}

\author{
	Alice\footnote{Alice@uWaterloo.ca}
\and % use this for more than one author
	Bob\footnote{Bob@uWaterloo.ca}
}
%==================================================================================
%==================================================================================
% preamble
\usepackage[margin=1in]{geometry}
\usepackage[svgnames,x11namees]{xcolor}
\usepackage{hyperref}\hypersetup{
	colorlinks=true,
	linkcolor={blue!75!black},	
	citecolor={blue!75!black},
	urlcolor={blue!75!black},
}
\usepackage{cancel}		\renewcommand{\CancelColor}{\color{red}} %change cancel color to red
\usepackage{graphicx}	\graphicspath{{graphics/}}
\usepackage{mathtools}	\DeclarePairedDelimiterX\set[1]\lbrace\rbrace{\def\given{\;\delimsize\vert\;}#1}
\usepackage{physics}
\usepackage{nicefrac}
\usepackage{paralist}
\usepackage{makeidx}	\makeindex
\usepackage{amsmath}
\usepackage{amsfonts}
\usepackage{amssymb}
\usepackage{amsthm}
%==================================================================================
% number the examples
\newcounter{example}[chapter]
\newenvironment{example}[2]{\noindent\refstepcounter{example}\par\medskip
	\noindent\textbf{Example~\thechapter.\theexample\quad #1} 
	\par\medskip\noindent#2
	\rmfamily}{\medskip}
\def\exampleautorefname~#1\null{Example~\thechapter.#1\null}
\newenvironment{solution}[1]{\par\medskip\noindent\textbf{Solution}\par\medskip #1}{\medskip}
%==================================================================================
% odonovan macros
\newcommand{\hint}[1]{{[{\bf Hint:} #1]}}
\newcommand{\gls}[1]{{\bf #1}} % not using glossaries here
\newcommand{\ie}{\textit{i.e.}\ }
\newcommand{\eg}{\textit{e.g.}\ }
\newcommand{\fd}[2]{ \frac{d #1 }{d #2 } }	% full derivative

%==================================================================================
%==================================================================================
\begin{document}
\frontmatter
\maketitle
\tableofcontents
\mainmatter
%==================================================================================
%==================================================================================
\setcounter{chapter}{0}
\chapter{Special Relativity}	\label{ch:Special Relativity}
%==================================================================================
%==================================================================================
\section{Introduction}

%==================================================================================
%==================================================================================
\section{Einstein's Postulates}

Instead of imposing an ether upon the experimental evidence Einstein\index{Einstein} found the simplest possible explanation (an application of Occam's razor -- the hypothesis with the fewest assumptions should be selected).

Einstein's postulates are 
\begin{enumerate}
\item
The laws of physics are the same in all inertial reference frames (IRF).
\item
The speed of light is the same in all directions.
\end{enumerate}

These postulates appear contradictory in light\index{pun} of what we know of Galilean relativity but Einstein was able to reconcile them by abandoning absolute time, replacing the usual Galilean transformations with the FitzGerald-Lorentz transformation.
%==================================================================================
%==================================================================================
\section{Invariants}

There are several scalar invariants that are the same in all inertial reference frames...
%==================================================================================
%==================================================================================
\section{Spacetime Diagrams (aka Minkowski Diagrams)}

%==================================================================================
\subsection{Galilean Relativity}

\begin{figure*}
\centering
\includegraphics{spaceTimeGalilean}
\caption{
A spacetime diagram has the time axis vertical and the space axis horizontal. The world-line of an object moving with a constant velocity is a straight line with a slope given by the velocity.
The three world lines, $x=0$, $x=vt$, and $x=ct$, are rotated in Bob's IRF.
Note that the slope of Bob's world-line relative to Alice's is $x=vt=(v/c) (ct)$ so for typical macroscopic velocities $v\sim 30 $ m/s and 
$\frac{x}{ct} = \frac{v}{c} \sim 10^{-7}$, while that for the world-line of the flash is $\frac{x}{ct}=1$.
}
\label{worldline1.fig}
\end{figure*}

%==================================================================================
%==================================================================================
\section{Lorentz Transformations}

\begin{figure}
\centering\includegraphics[scale=0.8]{spaceTime}
\caption{The Spacetime diagram for Alice, Bob, \& Claire in \autoref{ex:AliceBobClaire}}
\end{figure}

\begin{example}{Alice, Bob, \& Claire}
{\label{ex:AliceBobClaire}%
Consider three inertial observers, Alice, Bob, and Claire. Bob has velocity $\vec{v}=\hat{x}\, v$ relative to Alice, and Claire has velocity $\vec{v}\,'=\hat{x}\, v'$ relative to Bob. 

Recall that the Lorentz transformation for one space coordinate is given by
$$
	\mqty[ ct' \\ x' ] 
	=
	\gamma
	\mqty[ 1 & -v/c \\ -v/c & 1 ]
	\mqty[ ct \\ x ],
$$
with $\gamma = \frac{1}{\sqrt{1-(v/c)^2}}$, the usual Lorentz factor.}%
\begin{inparaenum}{(a)}
\item
In the diagram above, sketch the world-lines of Bob and Claire. Label the world lines of Alice, Bob, and Claire.
\item
What are Bob's space-time coordinates, $x'$ \& $t'$ in terms of Alice's, $x$ \& $t$?
\item
What are Claire's space-time coordinates, $x''$ \& $t''$ in terms of Bob's, $x'$ \& $t'$?
\item
Substitute your solution from (b) into those from (c) to find Claire's space-time coordinates, 
$x''$ \& $t''$ in terms of Alice's, $x$ \& $t$.
\hint{Check that your equations are dimensionally correct.}
\item
Now consider Claire's word-line, $x''=0$; substitute this into one of your results from (d), and solve the result for $x=ut$ (with $u$ some constant), Claire's world-line in Alice's coordinates.
\item
What is Claire's velocity, $u$, relative to Alice?
\item
Consider instead Claire's infinitesimal space-time displacement, $(c\,dt'', dx'')$. Use the Lorentz transformation given above to write this in terms of Bob's infinitesimal space-time displacement, $(c\,dt', dx')$. Then divide one by the other to obtain an expression for $\frac{dx''}{dt''}$.
\item
Now use $dx' = -v'\,dt'$ to find an expression for $\frac{dx''}{dt''}$.
\end{inparaenum}
\end{example}

%==================================================================================
\subsubsection{Minkowski Diagrams}

%==================================================================================
%==================================================================================
\section{Causality}
%==================================================================================
\subsubsection{Proper Length and Time}
%==================================================================================
%==================================================================================
\section{Lorentz Covariance}
%==================================================================================
%==================================================================================
\section{4-Vectors}
%==================================================================================
\subsubsection{4-Velocity}
%==================================================================================
\subsubsection{Addition of 4-Velocities}
\begin{example}{4-Velocity}
{Calculate the 4-velocity,
$$
	[U^\mu] 
	= 
	\frac{d}{d\tau} [X^\mu] 
$$
and the 4-acceleration,
\begin{align*}
	(A^\mu)
	&= 
	\frac{d}{d\tau} (U^\mu)
	= 
	\gamma\frac{d}{dt} \gamma( c, \vec{v}).
\end{align*}
}
\end{example}
%==================================================================================
\begin{solution}
\begin{enumerate}[(a)]
\item
This is the solution for part (a) is...
\end{enumerate}
\end{solution}


%==================================================================================
\begin{example}{Bob and Alice Play Ball}{%
Bob throws a ball with speed $u^\prime$ in the positive $x^\prime$ direction at $t^\prime=0$. Its position is then $x^\prime=u^\prime t^\prime$. Alice is at rest in an IRF moving at speed $v$ (in the positive $x$ direction) relative to Bob.}%
\begin{inparaenum}[(a)]
\item
Calculate Alice's measurement of the ball's speed using a Galilean transformation,
$
\smqty[x\\t]
=
\smqty[
1 & -v
\\
0 & 1
]
\smqty[
x^\prime
\\
t^\prime
].
$
\item
Calculate Alice's measurement of the ball's speed using a Lorentz transformation,
$
\smqty[x\\ct]
=
\gamma
\smqty[
1 & -\nicefrac{v}{c}
\\
-\nicefrac{v}{c} & 1
]
\smqty[
x^\prime
\\
ct^\prime
].
$
\item
Show that in the $v,u\ll c$ limits, the solution for (b) reduces to that of (a).
\item
Draw the space-time diagram for Bob and Alice's IRFs and the world-line of the ball. Add the ``rapidities'' to your diagram (\ie the angle, $\alpha$, between Bob's and Alice's world-lines and the angles, $\theta$ and $\theta^\prime$, between Alice's and Bob's world-lines and that of the ball.)

%\centering\input{figures/lines}
\item
If, $\alpha$, the ``rapidity'' is defined such that $\tanh\alpha=\nicefrac{v}{c}$, show that the ball's speed in Alice's IRF is given by the sum of the rapidities of Bob and Bob's ball. (Hint: Use the identity 
$  \tanh(x + y) = \frac{\tanh (x) + \tanh (y)}{1 + \tanh (x) \tanh (y)}$.)
\end{inparaenum}
\end{example}

%==================================================================================
\subsubsection{4-Momentum}

\begin{example}{Griffiths' Electrodynamics Pr.\ 12.35}
{In the past, most experiments in particle physics involved stationary targets: one particle (usually a proton or an electron) was accelerated to a high energy $E$, and collided with a target particle at rest (Fig. 12.29a). Far higher relative energies are obtainable (with the same accelerator) if you accelerate both particles to energy $E$, and fire them at each other (Fig. 12.29b). Classically, the energy E of one particle, relative to the other, is just $4E$ (why?)- not much of a gain (only a factor of 4). But relativistically the gain can be enormous. Assuming the two particles have the same mass, $m$, show that 
$$
	E' = \frac{eE^2}{mc^2} - mc^2.
$$
Suppose you use protons ($mc^2=1$ GeV) with $E = 30$ GeV. What $E'$ do you get? What multiple of E does this amount to? (1 GeV $=10^9$ electron volts.) [Because of this relativistic enhancement, most modern elementary particle experiments involve {\bf colliding beams}, instead of fixed targets.]
}
\end{example}

%==================================================================================
\section{Covariant and Contravariant}
%==================================================================================
\subsubsection{Minkowski Metric}
%==================================================================================
\subsection{Covariant and Contravariant}
%==================================================================================
%==================================================================================
%\section{}


\appendix
%==================================================================================
%==================================================================================
\chapter{Index Notation}

\section{Vectors}
\section{Inner Product}
\subsection{Einstein Summation Notation}
\section{Rank-2 Tensors}
\subsection{Outer Product}

%========================================================================
%========================================================================
\begin{example}{In Summation}
{
Consider the case in which the position of a particle is a function of time,
$
    \vec{r}(t)
    %=x_1(t)\hat{e}_1+x_2(t)\hat{e}_2+x_3(t)\hat{e}_3
    =x_i(t) \ \hat{e}_i
$. The $\hat{e}_i$ are the Cartesian orthonormal basis vectors; assume Einstein summation convention for this question. Express your answers in index notation.}%
\begin{inparaenum}
\item   %%%%%%%%%%%%%%%%%%%%%%%%%%%%%
Using index notation, calculate $r\dot{r}$ and $\vec{r}\cdot\dot{\vec{r}}$, where $\dot{\vec{r}}\equiv \fd{\vec{r}}{t}$, $r=|\vec{r}|$, etc.
\item   %%%%%%%%%%%%%%%%%%%%%%%%%%%%%
Using index notation, calculate $\frac{d}{dt}\left(\vec{r}\cdot\vec{r}\right)$.
\item   %%%%%%%%%%%%%%%%%%%%%%%%%%%%%
%Calculate $\vec{r}\cdot\dot{\vec{r}}$% = r \dot{r}$, where $r=\left|\vec{r}\right|$, etc.
\item   %%%%%%%%%%%%%%%%%%%%%%%%%%%%%
Plot $\vec{r}(t)$ for $\left|\vec{r}\right|=r_\circ$, a constant. 
(Hint: What is $\vec{r}\cdot\dot{\vec{r}}$ for this case?)
\end{inparaenum}
\end{example}

%========================================================================================
%========================================================================================
\begin{example}{Indicial}
{}
\begin{inparaenum}
\item   %%%%%%%%%%%%%%%%%%%%%%%%%%%%%%%%%%%%%%%%%%%%%%%%%%%%%%%%%%%%%%%%%%%%%%
Use index notation to show that
$$
    \qty( \vec{a}\times\vec{b} )\cdot\vec{c}
    =
    \qty( \vec{b}\times\vec{c} )\cdot\vec{a}
    =
    \qty( \vec{c}\times\vec{a} )\cdot\vec{b}
$$
(Hint: Use the properties of the Levi-Civita symbol, $\varepsilon_{ijk}$.)
\item   %%%%%%%%%%%%%%%%%%%%%%%%%%%%%%%%%%%%%%%%%%%%%%%%%%%%%%%%%%%%%%%%%%%%%%
Give a physical interpretation of the result from (a).
\item   %%%%%%%%%%%%%%%%%%%%%%%%%%%%%%%%%%%%%%%%%%%%%%%%%%%%%%%%%%%%%%%%%%%%%%
Use index notation to show that
$$
    \qty( \vec{a}\times\vec{b} ) \times \qty( \vec{c}\times\vec{d} )
    =
    \qty( \qty( \vec{a}\times\vec{b} )\cdot\vec{d} ) \vec{c}
    -
    \qty( \qty( \vec{a}\times\vec{b} )\cdot\vec{c} ) \vec{d}
$$
\end{inparaenum}
\end{example}

%==================================================================================
%==================================================================================
\backmatter

\addcontentsline{toc}{chapter}{Bibliography}
\begin{thebibliography}{9}

\bibitem{textbook}
{\bf Physics from Symmetry} by Jakob Schwichtenberg, Second Edition, ISBN 978-3-319-66631-0.

\bibitem{SR1}
Wikipedia contributors. {\bf Special Relativity.} Wikipedia, The Free Encyclopedia. January 7, 2019, 08:26 UTC. 
Available at: \href{https://en.wikipedia.org/wiki/Special_relativity}{https://en.wikipedia.org/wiki/Special\_relativity}
Accessed January 8, 2019.

\bibitem{group1}
Wikipedia contributors. {\bf Minkowski diagram.} Wikipedia, The Free Encyclopedia. January 7, 2019, 08:26 UTC. 
Available at: \href{https://en.wikipedia.org/wiki/Minkowski_diagram}{https://en.wikipedia.org/ wiki/Minkowski\_diagram}
Accessed January 8, 2019.
   
\end{thebibliography}

%==================================================================================
\newpage
\addcontentsline{toc}{chapter}{\indexname}
\printindex

\end{document}


%%==================================================================================
%\begin{example}{}{%
%This is the question.
%}%
%\begin{inparaenum}[(a)]
%\item
%This is the part.
%\end{inparaenum}
%\end{example}
