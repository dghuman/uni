\documentclass[10pt]{article}
\usepackage[]{ragged2e}
\usepackage{fancyhdr,amsmath,amsthm,amssymb,bbm,mathtools,xfrac}
\usepackage[utf8]{inputenc}
\usepackage[letterpaper,left=25mm,right=25mm]{geometry}

\setlength{\parskip}{1em}
\setlength{\parindent}{0em}

\newcommand{\Z}{\mathbb{Z}}
\newcommand{\R}{\mathbb{R}}
\newcommand{\Q}{\mathbb{Q}}
\newcommand{\C}{\mathbb{C}}
\newcommand{\N}{\mathbb{N}}
\newcommand{\Sp}{\mathbb{S}}
\newcommand{\Pro}{\mathbb{P}}
\newcommand{\di}[2][]{\frac{\partial #1}{\partial #2}}
\newcommand{\del}[2][]{\frac{d #1}{d #2}}
\DeclareMathOperator{\Ima}{Im}

\linespread{1.25}
\pagestyle{fancy}
\fancyhf{}
\lhead{PMATH 348 $|$  Assignment 5}

\rhead{Dilraj Ghuman $|$ 20564228}

\begin{document}

\textbf{Question 1}

\textbf{(a)} We find the splitting field of $f(x) = x^{11} - 2$ by finding the roots. In particular, notice that we have an immediate root of $\sqrt[\leftroot{1}\uproot{0}11]{2}$, and if we suppose $\xi_{11}$ the 11th root of unity, then we have that the roots of this polynomial will be $\sqrt[\leftroot{1}\uproot{0}11]{2}, \xi_{11}\sqrt[\leftroot{1}\uproot{0}11]{2}, \xi_{11}^{2}\sqrt[\leftroot{1}\uproot{0}11]{2}, \dots, \xi_{11}^{10}\sqrt[\leftroot{1}\uproot{0}11]{2}$. So, the splitting field will be $\Q(\sqrt[\leftroot{1}\uproot{0}11]{2}, \xi_{11})$. Notice, $\text{deg}_{\Q}\left(\sqrt[\leftroot{1}\uproot{0}11]{2}\right) = 11$ and $\text{deg}_{\Q}(\xi_{11}) = 10$ since 11 is prime. These two are coprime, and thus by a lemma we proved in the previous assignment,
\[ [\Q(\sqrt[\leftroot{1}\uproot{0}11]{2},\xi_{11}): \Q] = [\Q(\sqrt[\leftroot{1}\uproot{0}11]{2}):\Q][\Q(\xi_{11}):\Q] = 11\cdot 10 = 110\, .\]

\textbf{(b)} We find the splitting field of $f(x) = x^{4} - x^{2} + 4$ by first finding the roots:
\[ f(x) = (x^{2} + 2)^{2} - 5x^{2} = (x^{2} + 2 - \sqrt{5}x)(x^{2} + 2 + \sqrt{5})\]
\[ f(x) = \left(x - \frac{\sqrt{5} - i\sqrt{3}}{2}\right) \left(x - \frac{\sqrt{5} + i\sqrt{3}}{2}\right) \left(x - \frac{-\sqrt{5} - i\sqrt{3}}{2}\right) \left(x - \frac{-\sqrt{5} + i\sqrt{3}}{2}\right)\, .\]
These aren't nice roots, but we only need to adjoin $\sqrt{5}$ and $i\sqrt{3}$ onto $\Q$ to have the splitting field, that is, the splitting field is $\Q(\sqrt{5},i\sqrt{3})$. To compute $[\Q(\sqrt{5},i\sqrt{3}):\Q]$, we notice
\[ \text{deg}_{\Q}(\sqrt{5}) = \text{deg}_{\Q}(x^{2} - 5) = 2 \quad \text{deg}_{\Q}(\sqrt{3}) = \text{deg}_{\Q}(x^{2} + 3) = 2\, .\]
Since these are not coprime, we can't apply our little trick. Instead, we use the fact that $f(x)$ is irreducible over $\Q$. This would imply that any field extension using this polynomial better have degree atleast 4, and since $f(x)$ will split over $\Q(\sqrt{5},i\sqrt{3})$, it better be atleast degree 4. However, from last assignment, we also have an inequality that says $[\Q(\sqrt{5},i\sqrt{3}):\Q] \leq 4$, and thus we would require that $[\Q(\sqrt{5},i\sqrt{3}):\Q] = 4$.

So, to see that $f(x)$ is irreducible, we use the mod-3 irreducibility test. Then,
\[ \bar{f}(x) = x^{4} - x^{2} + 1 \quad \bar{f}(0) = 1,\bar{f}(1) = 1, \bar{f}(2) = 1 \, .\]
So, we need to check if this factors into degree 2 irreducible polynomials. The irreducible polynomials of $\Z_{3}[x]$ are
\[ x^{2}+2\, ,\, x^{2}+1\, , \, x^{2}+x+2\, ,\, x^{2}-x+2\,,\,2x^{2}+x+1\,,\,2x^{2}-x+1 \, .\]
Immediatly, we see that the polynomials with leading coefficient 2 can only multiply one another since the result better be monic, but the product is $x^{4} - x + 1$. Similarly, the polynomials with only an $x^{2}$ term can only multiply one another since the result with another polynomial would leave them with an extra $x^{3}$ term, and they multiply together to $x^{4} - 1$. Finally, the remaining two multiply to $x^{4} + 4$. Thus, we have irreducibility, and
\[ [\Q(\sqrt{5},i\sqrt{3}):\Q] = 4\, .\]

\textbf{(c)} 


\end{document}
