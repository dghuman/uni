\documentclass[10pt]{article}
\usepackage[]{ragged2e}
\usepackage{fancyhdr,amsmath,amsthm,amssymb,bbm,mathtools,xfrac}
\usepackage[utf8]{inputenc}
\usepackage[letterpaper,left=25mm,right=25mm]{geometry}

\setlength{\parskip}{1em}
\setlength{\parindent}{0em}

\newcommand{\Z}{\mathbb{Z}}
\newcommand{\R}{\mathbb{R}}
\newcommand{\Q}{\mathbb{Q}}
\newcommand{\C}{\mathbb{C}}
\newcommand{\N}{\mathbb{N}}
\newcommand{\Sp}{\mathbb{S}}
\newcommand{\Pro}{\mathbb{P}}
\newcommand{\di}[2][]{\frac{\partial #1}{\partial #2}}
\newcommand{\del}[2][]{\frac{d #1}{d #2}}
\DeclareMathOperator{\Ima}{Im}

\linespread{1.25}
\pagestyle{fancy}
\fancyhf{}
\lhead{PMATH 348 $|$  Assignment 4}

\rhead{Dilraj Ghuman $|$ 20564228}

\begin{document}

\textbf{Question 1}

\textbf{(a)} Since $\alpha \in K$ is algebraic over $F$, and $F \subset E$ as a subfield, then $\alpha$ is algebraic over E. By definition, $\exists\, f(x)\in F[x]$ minimial polynomial of $\alpha$ over $F$. There are two cases to consider for this polynomial:

(i) Suppose $f(x)$ is irreducable in $E[x]$ where the coeffecients of $f(x)$ are sent under the canonical inclusion map. Then, since $f(\alpha) = 0$ and it is monic by construction, $f(x)$ is the minimal polynomial over $E$ of $\alpha$. With this determined
\[ [E(\alpha):E] = \text{deg}_{E}(f(x)) = \text{deg}_{F}(f(x)) = [F(\alpha):F] \,.\]

(ii) Suppose $f(x)$ is reducable in $E[x]$ where the coeffecients are mapped as before. Then, $\exists\, h(x),g(x)\in E[x]$ such that $\text{deg}_{E}(h(x)) \geq 1$, $g(x)$ is the minimal polynomial of $\alpha$ over $E$ and $f(x) = h(x)g(x)$. Therefore,
\[ [E(\alpha):E] = \text{deg}_{E}(g(x)) \leq \text{deg}_{E}(h(x)g(x)) = \text{deg}_{F}(f(x)) = [F(\alpha):F] \, .\]

\textbf{(b)} We first note that $E(\alpha)/E$, $E(\alpha)/F$, $F(\alpha)/F$ and $E(\alpha)/F(\alpha)$ are all finite extensions. So, we are motivated to apply the Tower Theorem:
\[ [E(\alpha):F] = [E(\alpha):F(\alpha)][F(\alpha):F] \geq [E(\alpha):F(\alpha)][F(\alpha):E] \]
where the inequality comes from \textbf{(a)}. Then, with another application of the Tower theorem
\[ [E(\alpha):F(\alpha)] \leq \frac{[E(\alpha):F]}{[E(\alpha):E]} = \frac{[E(\alpha):E][E:F]}{[E(\alpha):E]} = [E:F]\]
as required.

\newpage
\textbf{Question 2}

Since $F(\alpha,\beta)/F(\alpha),F(\alpha,\beta)/F(\beta),F(\alpha)/F,F(\beta)/F,$ and $F(\alpha,\beta)/F$ are all finite extensions, we can apply the Tower Theorem. Let $\text{deg}_{F}(\alpha) = m$, and $\text{deg}_{F}(\beta) = n$, then
\[ [F(\alpha,\beta):F] = [F(\alpha,\beta):F(\alpha)][F(\alpha):F] = [F(\alpha,\beta):F(\alpha)]m \]
\[ [F(\alpha,\beta):F] = [F(\alpha,\beta):F(\beta)][F(\beta):F] = [F(\alpha,\beta):F(\beta)]n \,.\]
So we clearly have that $n \, |\, [F(\alpha,\beta):F]$ and $m \, |\, [F(\alpha,\beta):F]$, and since these are coprime, $nm \, | \,[F(\alpha,\beta):F]$.

Now we use the previous question to see that
\[ [F(\alpha,\beta):F] = [F(\alpha,\beta):F(\alpha)][F(\alpha):F] \leq [F(\beta):F][F(\alpha):F] = mn \]
and since $mn \, | \, [F(\alpha,\beta):F]$, we have that
\[ [F(\alpha,\beta):F] = mn = [F(\beta):F][F(\alpha):F] \, .\]


\newpage
\textbf{Question 3}

\textbf{(a)} From the previous assignment, we recall that $\text{deg}_{\Q}(i + \sqrt{2}) = 4$ and $\text{deg}_{\Q}(\cos(\frac{\pi}{9})) = 3$, which are coprime. So, from the previous lemma
\[ [\Q(\alpha,\beta): \Q] = [\Q(\alpha):\Q][\Q(\beta):\Q] = (3)(4) = 12 \, .\]

\textbf{(b)} By the Tower theorem,
\[ [\Q(\sqrt{p},\sqrt{q}):\Q] = [\Q(\sqrt{p},\sqrt{q}):\Q(\sqrt{p}][\Q(\sqrt{p}):\Q] \,.\]
We recognize that each square root of a prime will have a minimal polynomial of order 2, namely $x^{2} - p$ and $x^{2} - q$ over $\Q$. So, all we need to show is that this minimal polynomial is the same in the adjoined field over the opposite extension. In particular, in this case we consider $x^{2} - q$ over $\Q(\sqrt{p})$. Clearly this is still irreducible, as otherwise we would have a power of $\sqrt{p}$ that is $q$, but they are distinct primes, and so that can not be the case. Thus,
\[ [\Q(\sqrt{p},\sqrt{q}):\Q] = [\Q(\sqrt{p},\sqrt{q}):\Q(\sqrt{p}][\Q(\sqrt{p}):\Q] = 2\cdot 2 = 4 \,.\]

\textbf{(c)} We notice that this polynomial is reducible, in particular,
\[ x^{3} + x + 1 = (x + 1)(x^{2} - x + 2) \, \in \Z_{3}[x] \, .\]
Here, we see that $x^{2} - x + 2$ is irreducible and so is $x + 1$, so we just need to consider $\alpha$ in these two cases.

Suppose $\alpha = 2$, then the minimal polynomial is $x + 1$, so $\text{deg}_{\Z_{3}}(\alpha) = 1$, and thus
\[[\Z_{3}(\alpha):\Z_{3}] = 1 \, .\]

Suppose $\alpha$ is a root of the irreducible and monic $x^{2} - x + 2$, then we have that
\[[\Z_{3}(\alpha):\Z_{3}] = 2 \, .\]

\textbf{(d)} Since $\R(t) = \R(t,t^{2})$, we can think of $\R(t^{2}) = F$ as a new field, and then we see that the question becomes to find $[F(t):F]$, which is to say we want a minimal polynomial in $F$ of $t$. So, what polynomial in $\R(t^{2})[x]$ has a root of $t$. Well, we see that the natural choice would be $x^{2} - t^{2} \in \R(t^{2})[x]$. So, the minimal polynomial, call it $p(x)$, must divide this polynomial, and hence $\text{deg}_{\R(t^{2})}(t) \in \{1,2\}$. But, if the degree were 1, then that would imply that $t\in \R(t^{2})$, which is false, so we must have that the degree of $t$ is 2. Then, $[\R(t):\R(t^{2})] = 2$, as required.

\newpage
\textbf{Question 4}

We build the natural homomorphism and show that it is an isomorphism. That is, for any rational polynomial $\frac{f(\alpha)}{g(\alpha)}\in F(\alpha), g(\alpha) \neq 0$, which is by definition of adjoining, we define the map $\varphi: F(\alpha) \to F(x)$ by
\[ \varphi\left(\frac{f(\alpha)}{g(\alpha)}\right) = \frac{f(x)}{g(x)} \, \in F(x) \, .\]
This is the natural homomorphism, and since all we do is relabel, this better be a homomorphism. Notice here, however, that since $\alpha$ is trancendental over $F$, $\text{ker}(\varphi) = \{0\}$. That is, the kernel is trivial, and so we have injection. Moreover, notice that the map has a natural inverse that is also injective, thus we have surjection, and 
\[ F(\alpha) \cong F(x) \]
as needed. Since $\beta$ is trancendental like $\alpha$, the above holds for it aswell and we get $F(\beta) \cong F(x)$, and by chaining isomorphisms, $F(\beta) \cong F(\alpha)$.

\newpage
\textbf{Question 5}

We use the Tower theorem and what we did in \textbf{Question 1} to prove this statement. First, since $f(x) \in \Q[x]$ is irreducible, then by Kronecker's theorem, $\exists \, \alpha$ in some field extension of $\Q$ such that $f(\alpha) = 0$. Thus, we see that $f(x)$ is the minimal polynomial of $\alpha$ over $\Q$, up to some unit multiple to make it monic. This gives us the following extensions we can use
\[ E(\alpha)/\Q \quad E(\alpha)/\Q(\alpha) \quad \Q(\alpha)/\Q \quad E(\alpha)/E \, .\]
Since these are all finite (due to $E/\Q$ being finite), we apply the Tower Theorem along with inequalities from \textbf{Question 1} to get
\[ [E(\alpha):Q] = [E(\alpha):E][E:\Q] \leq [\Q(\alpha):\Q]\cdot 2 = 4n \]
\[ [E(\alpha):Q] = [E(\alpha):\Q(\alpha)][\Q(\alpha):\Q] \leq [E:\Q]\cdot 2n = 4n \]
\[ \implies 2n\, | \, [E(\alpha):Q] \quad [E(\alpha):Q] \leq 4n\]
\[ \implies [E(\alpha):Q] \in \{2n, 4n\} \, .\]
So, we need only consider these two cases of $[E(\alpha):Q]$.

Suppose $[E(\alpha):Q] = 4n$, then by the Tower Theorem, $[E(\alpha):E] = 2n$, and this says that the minimal polynomial of $\alpha$ over $E$ is of order $2n$, but we already know that $f(x)\in E[x]$ is the minimal polynomial over $\Q$, and thus can not have reduced in degree in the extension, and hence $\text{deg}_{E}f(x) = 2n$ and it is irreducible.

Suppose $[E(\alpha):Q] = 2n$, then by the Tower Theorem, $[E(\alpha):E] = n$, that is the minimal polynomial of $\alpha$ over $E$ has degree $n$. Since $f(x) \in \Q[x]$ was the minimal polynomial (up to unit multiple), we expect its extension to contain the minimal polynomial over $E$, but the degree must be $n$. Suppose the minimal polynomial is $p(x)\in E[x]$, then we must have that $f(x) = p(x)h(x)$, with $h(x)\in E[x]$ some other degree $n$ polynomial. Clearly $p(x)$ is irreducible, so we need to consider the reducibility of $h(x)$. However, since $\alpha$ was chosen arbitrarily, any root of $h(x)$ will follow a similar path and hence will require it also being of degree n. Thus, $h(x)$ better be irreducible.
\end{document}
