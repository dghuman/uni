\documentclass[10pt]{article}
\usepackage[]{ragged2e}
\usepackage{fancyhdr,amsmath,amsthm,amssymb,bbm,mathtools,xfrac}
\usepackage[utf8]{inputenc}
\usepackage[letterpaper,left=25mm,right=25mm]{geometry}

\setlength{\parskip}{1em}
\setlength{\parindent}{0em}

\newcommand{\Z}{\mathbb{Z}}
\newcommand{\R}{\mathbb{R}}
\newcommand{\Q}{\mathbb{Q}}
\newcommand{\C}{\mathbb{C}}
\newcommand{\N}{\mathbb{N}}
\newcommand{\Sp}{\mathbb{S}}
\newcommand{\Pro}{\mathbb{P}}
\newcommand{\di}[2][]{\frac{\partial #1}{\partial #2}}
\newcommand{\del}[2][]{\frac{d #1}{d #2}}
\DeclareMathOperator{\Ima}{Im}

\linespread{1.25}
\pagestyle{fancy}
\fancyhf{}
\lhead{PMATH 348 $|$  Assignment 4}

\rhead{Dilraj Ghuman $|$ 20564228}

\begin{document}

\textbf{Question 1}

\textbf{(a)} Since $\alpha \in K$ is algebraic over $F$, and $F \subset E$ as a subfield, then $\alpha$ is algebraic over E. By definition, $\exists\, f(x)\in F[x]$ minimial polynomial of $\alpha$ over $F$. There are two cases to consider for this polynomial:

(i) Suppose $f(x)$ is irreducable in $E[x]$ where the coeffecients of $f(x)$ are sent under the canonical inclusion map. Then, since $f(\alpha) = 0$ and it is monic by construction, $f(x)$ is the minimal polynomial over $E$ of $\alpha$. With this determined
\[ [E(\alpha):E] = \text{deg}_{E}(f(x)) = \text{deg}_{F}(f(x)) = [F(\alpha):F] \,.\]

(ii) Suppose $f(x)$ is reducable in $E[x]$ where the coeffecients are mapped as before. Then, $\exists\, h(x),g(x)\in E[x]$ such that $\text{deg}_{E}(h(x)) \geq 1$, $g(x)$ is the minimal polynomial of $\alpha$ over $E$ and $f(x) = h(x)g(x)$. Therefore,
\[ [E(\alpha):E] = \text{deg}_{E}(g(x)) \leq \text{deg}_{E}(h(x)g(x)) = \text{deg}_{F}(f(x)) = [F(\alpha):F] \, .\]

\textbf{(b)} We first note that $E(\alpha)/E$, $E(\alpha)/F$, $F(\alpha)/F$ and $E(\alpha)/F(\alpha)$ are all finite extensions. So, we are motivated to apply the Tower Theorem:
\[ [E(\alpha):F] = [E(\alpha):F(\alpha)][F(\alpha):F] \geq [E(\alpha):F(\alpha)][F(\alpha):E] \]
where the inequality comes from \textbf{(a)}. Then, with another application of the Tower theorem
\[ [E(\alpha):F(\alpha)] \leq \frac{[E(\alpha):F]}{[E(\alpha):E]} = \frac{[E(\alpha):E][E:F]}{[E(\alpha):E]} = [E:F]\]
as required.

\newpage
\textbf{Question 2}

Since $F(\alpha,\beta)/F(\alpha),F(\alpha,\beta)/F(\beta),F(\alpha)/F,F(\beta)/F,$ and $F(\alpha,\beta)/F$ are all finite extensions, we can apply the Tower Theorem. Let $\text{deg}_{F}(\alpha) = m$, and $\text{deg}_{F}(\beta) = n$, then
\[ [F(\alpha,\beta):F] = [F(\alpha,\beta):F(\alpha)][F(\alpha):F] = [F(\alpha,\beta):F(\alpha)]m \]
\[ [F(\alpha,\beta):F] = [F(\alpha,\beta):F(\beta)][F(\beta):F] = [F(\alpha,\beta):F(\beta)]n \,.\]
So we clearly have that $n \, |\, [F(\alpha,\beta):F]$ and $m \, |\, [F(\alpha,\beta):F]$, and since these are coprime, $nm \, | \,[F(\alpha,\beta):F]$.

\end{document}
