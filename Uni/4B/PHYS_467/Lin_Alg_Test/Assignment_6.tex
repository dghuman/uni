\documentclass[10pt]{article}
\usepackage[]{ragged2e}
\usepackage{fancyhdr,amsmath,amsthm,amssymb,bbm}
\usepackage[utf8]{inputenc}
\usepackage[letterpaper,left=25mm,right=25mm]{geometry}

\setlength{\parskip}{1em}
\setlength{\parindent}{0em}

\newcommand{\Z}{\mathbb{Z}}
\newcommand{\R}{\mathbb{R}}
\newcommand{\Q}{\mathbb{Q}}
\newcommand{\C}{\mathbb{C}}
\newcommand{\N}{\mathbb{N}}
\newcommand{\Sp}{\mathbb{S}}
\newcommand{\Pro}{\mathbb{P}}
\newcommand{\di}[2][]{\frac{\partial #1}{\partial #2}}
\newcommand{\del}[2][]{\frac{d #1}{d #2}}

\DeclareMathOperator{\Ima}{Im}

\linespread{1.25}
\pagestyle{fancy}
\fancyhf{}
\lhead{PHYS 467 $|$  Assignment 0}

\rhead{Dilraj Ghuman $|$ 20564228}

\begin{document}
\textbf{Question 1}

\textbf{(a)} Suppose $A$ is a $d \times d$ matrix, then we know that
\[ e^{A} = \sum_{n=0}^{\infty}\frac{1}{n!}A^{n}. \]

\textbf{(b)} We make use of the exponential expansion we did just above. In particular, we see that
\[ e^{-iUHU^{\dagger}t} = \sum_{n=0}^{\infty}\frac{(-it)^{n}}{n!}(UHU^{\dagger})^{n} = \sum_{n=0}^{\infty}\frac{(-it)^{n}}{n!}\underbrace{UHU^{\dagger}\cdot UHU^{\dagger} \dots UHU^{\dagger}}_{n \text{times}}\]
but we know that $U$ is unitary, so we have that $U^{\dagger} = U^{-1}$, so we get that
\[ e^{-iUHU^{\dagger}t} = \sum_{n=0}^{\infty}\frac{(-it)^{n}}{n!}UH^{n}U^{\dagger} = Ue^{-iHt}U^{\dagger} \]
as expected.

\newpage
\textbf{Question 2}

\textbf{(a)} This is done with a direct computation. Notice,
\[ ee^{\dagger} =
\begin{pmatrix}
  \cos^{2}\frac{\theta}{2} & e^{-i \phi}\sin\frac{\theta}{2}\cos\frac{\theta}{2} \\
  e^{i\phi}\sin\frac{\theta}{2}\cos\frac{\theta}{2} & \sin^{2}\frac{\theta}{2} \\
\end{pmatrix}
\]
and from inspection, we see that
\[ (ee^{\dagger})^{\dagger} =
\begin{pmatrix}
  \cos^{2}\frac{\theta}{2} & e^{-i \phi}\sin\frac{\theta}{2}\cos\frac{\theta}{2} \\
  e^{i\phi}\sin\frac{\theta}{2}\cos\frac{\theta}{2} & \sin^{2}\frac{\theta}{2} \\
\end{pmatrix}
= ee^{\dagger} \]
as expected.

\textbf{(b)} We can see $b$ and $c$ just from looking at the matrix. In particular, using the expansion of $e^{i\phi} = \cos\phi + i\sin\phi$ we see that we must have
\[ b = \cos\phi \sin\frac{\theta}{2} \cos\frac{\theta}{2} \hspace{2em} c = \sin\phi \sin\frac{\theta}{2} \cos\frac{\theta}{2}. \]
On the other hand, we see that obtaining $a$ and $d$ is just a matter of solving the system,
\[ a + d = \cos^{2}\frac{\theta}{2} \hspace{2em} \& \hspace{2em} a - d = \sin^{2}\frac{\theta}{2} \]
\[ \implies 2a = \cos^{2}\frac{\theta}{2} + \sin^{2}\frac{\theta}{2} = 1 \implies a = \frac{1}{2} \]
\[ \implies d = \cos^{2}\frac{\theta}{2} -\frac{1}{2}. \]
Hence, we have our real $a,b,c$ and $d$.

\newpage
\textbf{Question 3}

\textbf{(a)} We know that the eigenvectors will satisfy the eigenvalue problem when applied to $X$, and from inspection we can find that
\[ f_{0} =
\begin{pmatrix}
  1 \\
  1 \\
\end{pmatrix}
\hspace{2em} \& \hspace{2em} f_{1} =
\begin{pmatrix}
  -1 \\
  1 \\
\end{pmatrix}
\]
which are expressed in the basis of $\{e_{0},e_{1}\}$.

\textbf{(b)} We know that the $+1$ eigenspace is simply $\{e_{0}\}$ for $Z$ and $\{f_{0}\}$ for $X$, so we would expect the eigenbasis of a tensor space to be the tensor of the corresponding eigenspaces, that is $\{f_{0} \otimes e_{0} \otimes f_{0\}}$. Since we have all of these vectors in the standard basis representation, we can explicitly write out the vector that would be this tensor as
\[ f_{0} \otimes e_{0} \otimes f_{0} =
\begin{pmatrix}
  1 \\
  1 \\
  0 \\
  0 \\
  1 \\
  1 \\
  0 \\
  0 \\
\end{pmatrix}
\]

\textbf{(c)} Here we will avoid writing out the tensor product explicitly, since it is instead easier to compute using the multilinearity of the tensor. That is, the tensor product results in a tensor, and hence is multilinear and we can just consider the null spaces of each component of the tensor. Simply put, we need the null spaces of $I+X$ and $I+Z$, and then we just tensor them accordingly.

By inspection, we notice that the null space of $I+X$ better be $f_{1}$, since we want all vectors whose inverse is obtained by swapping components, which is spanned by $f_{1}$. Similarly, we see that for the null space of $I+Z$, we want all vectors that only have second component, which is spanned by $e_{1}$. Thus, we have that the basis of our null space, $B_{\text{null}}$, will be
\[ B_{\text{null}} = \{f_{1} \otimes e_{1} \otimes f_{1} \} \]
as required.

\end{document}
