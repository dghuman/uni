\documentclass[10pt]{article}
\usepackage[]{ragged2e}
\usepackage{fancyhdr,amsmath,amsthm,amssymb,bbm,braket}
\usepackage[utf8]{inputenc}
\usepackage[letterpaper,left=25mm,right=25mm]{geometry}

\setlength{\parskip}{1em}
\setlength{\parindent}{0em}

\newcommand{\Z}{\mathbb{Z}}
\newcommand{\R}{\mathbb{R}}
\newcommand{\Q}{\mathbb{Q}}
\newcommand{\C}{\mathbb{C}}
\newcommand{\N}{\mathbb{N}}
\newcommand{\Sp}{\mathbb{S}}
\newcommand{\Pro}{\mathbb{P}}
\newcommand{\di}[2][]{\frac{\partial #1}{\partial #2}}
\newcommand{\del}[2][]{\frac{d #1}{d #2}}

\DeclareMathOperator{\Ima}{Im}

\linespread{1.25}
\pagestyle{fancy}
\fancyhf{}
\lhead{PHYS 467 $|$  Assignment 2}

%\rhead{Dilraj Ghuman $|$ 20564228}

\begin{document}
\textbf{Question 1}

\textbf{Question 2}

\textbf{Question 3}

\textbf{(a)} We recall that
\[ H = \frac{1}{\sqrt{2}}
\begin{bmatrix}
  1 & 1 \\
  1 & -1 \\
\end{bmatrix}
\,
Z = 
\begin{bmatrix}
  1 & 0 \\
  0 & -1 \\
\end{bmatrix}
\,
X =
\begin{bmatrix}
  0 & 1 \\
  1 & 0 \\
\end{bmatrix}
\]
Then, we see that
\[ HXH = \frac{1}{2}
\begin{bmatrix}
  1 & 1 \\
  1 & -1 \\
\end{bmatrix}
\begin{bmatrix}
  1 & 0 \\
  0 & -1 \\
\end{bmatrix}
\begin{bmatrix}
  1 & 1 \\
  1 & -1 \\
\end{bmatrix}
= \frac{1}{2}
\begin{bmatrix}
  1 & 1 \\
  1 & -1 \\
\end{bmatrix}
\begin{bmatrix}
  1 & -1 \\
  1 & 1 \\
\end{bmatrix}
=
\frac{1}{2}
\begin{bmatrix}
  2 & 0 \\
  0 & -2 \\
\end{bmatrix}
= Z \]
as expected.

\textbf{(b)}
To avoid drawing out the diagrams in tikz, I label them as 1, 2, and 3 from top to bottom. So, circuit 1 is the one with 4 hadamard matrices.

(1) We just need to expand the tensor expression, that is the first diagram can equivalently be written as
\[ (H \otimes H)(CU)(H \otimes H) = (H \otimes H)\left(\ket{0}\bra{0}\otimes I + \ket{1}\bra{1}\otimes X\right)(H\otimes H) \]
where $CU$ is the $CNOT$ gate. We notice that it will be convenient to use BraKet notation. In that notation, we recognize that $X = \ket{1}\bra{0} + \ket{0}\bra{1}$ and $H = \frac{1}{\sqrt{2}}\left((\ket{0} + \ket{1})\bra{0} + (\ket{0} - \ket{1})\bra{1}\right)$
\end{document}
