\documentclass[10pt]{article}
\usepackage[]{ragged2e}
\usepackage{fancyhdr,amsmath,amsthm,amssymb,bbm,braket,mathtools}
\usepackage[utf8]{inputenc}
\usepackage[letterpaper,left=25mm,right=25mm]{geometry}

\setlength{\parskip}{1em}
\setlength{\parindent}{0em}

\newcommand{\Z}{\mathbb{Z}}
\newcommand{\R}{\mathbb{R}}
\newcommand{\Q}{\mathbb{Q}}
\newcommand{\C}{\mathbb{C}}
\newcommand{\N}{\mathbb{N}}
\newcommand{\Sp}{\mathbb{S}}
\newcommand{\Pro}{\mathbb{P}}
\newcommand{\di}[2][]{\frac{\partial #1}{\partial #2}}
\newcommand{\del}[2][]{\frac{d #1}{d #2}}
\newcommand{\norm}[1]{\left\lVert#1\right\rVert}
\newcommand{\Que}[1]{\textbf{Question #1}}

\DeclareMathOperator{\Ima}{Im}

\linespread{1.25}
\pagestyle{fancy}
\fancyhf{}
\lhead{PHYS 467 $|$  Assignment 4}

%\rhead{Dilraj Ghuman $|$ 20564228}

\begin{document}
\Que{1}

\textbf{(a)} First, we see that the initial step can be completed in $K$ queries at most, and if this is the case, then the remaining problem becomes a search problem of finding $K$ marked items in the list of $N$. Notice there will be some constraints on $N$ for this case; if $K > N/2$, then we are guerenteed a collision in the set $K$, even in the worst case. Thus, for the most possible number of queries, we need $K \leq N/2$. In this case, we will get
\[ O(K + \sqrt{N/K}) \]
queries, where the first comes from the initial step, and the last order comes from the quantum search algorithm.

\textbf{(b)} Notice, if we choose $K = N^{1/3}$, we get
\[ \implies O(N^{1/3} + \sqrt{N/N^{1/3}}) = O(N^{1/3} + \sqrt{N^{2/3}}) = O(2N^{1/3}) = O(N^{1/3})\]
as required.


\newpage
\Que{2}

\textbf{(a)} First, we know the state in $C$ is already $H^{b}\ket{s}$, so we only need to consider the states prepared in the $A_{1}$ and $A_{2}$ systems to understand what kind of density matrix we want. In particular, we know that Alice knows what $s$ and $b$ are, since that is how she prepares the state $H^{b}\ket{s}$. We see that
\[ \ket{s}\otimes \ket{b}\otimes H^{b}\ket{s} \in A_{1}A_{2}C \, .\]
Therefore, the density matrix will be
\[ \ket{s}\bra{s} \otimes \ket{b}\bra{b} \otimes H^{b}\ket{s}\bra{s}H^{b} \]
since $(H^{b})^{\dagger} = H^{b}$ in this case. Therefore, expanding each interms of their known components, we get
\[
\ket{s}\bra{s} \otimes \ket{b}\bra{b} \otimes 
\frac{1}{2^{b}}\left(\ket{+}\bra{0} + \ket{-}\bra{1}\right)^{b}
\ket{s}\bra{s}
\left(\ket{+}\bra{0} + \ket{-}\bra{1}\right)^{b} \,.
\]
Further, we can eliminate the dependence on $b$ by being clever with our construction, that is we can write
\[ \ket{b}\otimes H^{b}\ket{s} = \frac{1}{\sqrt{2}}\ket{0}\otimes \ket{s} + \frac{1}{\sqrt{2}}\ket{1}\otimes H\ket{s}\]
which builds the dependence upon $b$ into the state. So, the final density is
\[ \ket{s}\bra{s} \otimes \frac{1}{2}\left(\ket{0}\bra{0}\otimes \ket{s}\bra{s} + \ket{1}\bra{1}\otimes H\ket{s}\bra{s}H\right)\, ,\]
where $H = \frac{1}{\sqrt{2}}(\ket{+}\bra{0} + \ket{-}\bra{1})$. 

\textbf{(b)} Eve has only caught the state in the system $C$, so we must have either
\[\rho_{0} = \frac{1}{2}\left(\ket{0}\bra{0} + H\ket{0}\bra{0}H\right) \quad \& \quad \rho_{1} = \frac{1}{2}\left(\ket{1}\bra{1} + H\ket{1}\bra{1}H\right)\]

\textbf{(c)} We use classical probability theory to see that we simply need to compute
\[ P(s=e) = \frac{1}{2}\text{Tr}\left(P_{0}\rho_{0}\right) + \frac{1}{2}\text{Tr}(P_{1}\rho_{1})\]
where
\[ P_{0}\rho_{0} = \frac{1}{2}\left(I + \frac{Z + X}{\sqrt{2}}\right)\frac{1}{2}\left(\ket{0}\bra{0} + H\ket{0}\bra{0}H\right) \]
\[ = \frac{1}{4}\left(\ket{0}\bra{0} + H\ket{0}\bra{0}H + \frac{1}{\sqrt{2}}(\ket{0}\bra{0} + \ket{1}\bra{0}) + \frac{1}{\sqrt{2}}(ZH\ket{0}\bra{0}H + XH\ket{0}\bra{0}H) \right) \]
and so the trace of this thing would be
\[ \frac{1}{4}\left(1 +\frac{1}{2} + \frac{1}{\sqrt{2}} + \frac{1}{\sqrt{2}}(\frac{1}{2}) + \frac{1}{2} + \frac{1}{2} + \frac{1}{\sqrt{2}}\left(-\frac{1}{2}\right) + \frac{1}{2}\right) = \frac{1}{4}\left(3 + \frac{1}{\sqrt{2}}\right)\, .\]
On the other hand, we have
\[ P_{1}\rho_{1} = \frac{1}{2}\left(I - \frac{Z + X}{\sqrt{2}}\right) \frac{1}{2}\left(\ket{1}\bra{1} + H\ket{1}\bra{1}H\right)\]
\[ = \frac{1}{4}\left(I - \frac{Z + X}{\sqrt{2}}\right)\left(\ket{1}\bra{1} + H\ket{1}\bra{1}H\right)\]
\[ = \frac{1}{4}\left(\ket{1}\bra{1} + H\ket{1}\bra{1}H - \frac{\ket{1}\bra{1} + \ket{0}\bra{1}}{\sqrt{2}} - \frac{1}{\sqrt{2}}\left(ZH\ket{1}\bra{1}H + XH\ket{1}\bra{1}H\right) \right) \]
\[ = \frac{1}{4}\left(\frac{1}{2} - \frac{1}{\sqrt{2}}\left(\frac{1}{2} -\frac{1}{2}\right) + 1 + \frac{1}{2} - \frac{1}{\sqrt{2}} - \frac{1}{\sqrt{2}}\left(-\frac{1}{2} - \frac{1}{2}\right)\right) = \frac{2}{4}\]
and thus the total will be
\[ P(s=e) = \frac{1}{2}\left(\frac{1}{4}\left(3 + \frac{1}{\sqrt{2}}\right)\right) + \frac{1}{2}\left(\frac{1}{2}\right) = \frac{1}{8}\left(5 + \frac{1}{\sqrt{2}}\right)\]
as required.

\textbf{(d)} In this scenario, we are assuming $s=0, b=0$ and $e =0$, so we can first find the post-measurement state to be
\[ \ket{\psi} = \frac{P_{0}\rho_{0} P_{0}}{\text{Tr}(P_{0}\rho_{0})} = \frac{1}{\frac{1}{4}\left(3 + \frac{1}{\sqrt{2}}\right)}\left(\frac{1}{2}\left(I + \frac{Z + X}{\sqrt{2}}\right)\frac{1}{2}\left(\ket{0}\bra{0} + H\ket{0}\bra{0}H\right)\left(\frac{1}{2}\left(I + \frac{Z + X}{\sqrt{2}}\right)\right)\right) \]
\[ = \frac{4}{3 + \frac{1}{\sqrt{2}}}\frac{1}{8}\left(\ket{0}\bra{0} + H\ket{0}\bra{0}H + \frac{1}{\sqrt{2}}(\ket{0}\bra{0} + \ket{1}\bra{0}) + \frac{1}{\sqrt{2}}(ZH\ket{0}\bra{0}H + XH\ket{0}\bra{0}H)\right)\left(I + \frac{Z + X}{\sqrt{2}}\right) \]

\newpage
\Que{3}

\textbf{(a)} This is just a matter of applying our definition of a matrix exponential. We recall that
\[ e^{i\theta Z} = I\cos(\theta) + iZ\sin(\theta)\]
But, we notice that
\[ D_{1}(\rho) = \frac{1}{2}e^{i\theta Z}\left(\rho e^{-i2\theta Z} + e^{-i2 \theta Z}\rho\right)e^{i\theta Z} \]
and so we first compute the inner bracket to get
\[ \rho e^{-i2\theta Z} + e^{-i2 \theta Z}\rho = \rho \cos(2\theta) - i\rho Z\sin(2\theta) + \rho \cos(2\theta) - iZ\rho \sin(2\theta) \]
Notice that
\[ \rho Z =
\begin{pmatrix}
  a & b\\
  c & d\\
\end{pmatrix}
\begin{pmatrix}
  1 & 0 \\
  0 & -1 \\
\end{pmatrix}
=
\begin{pmatrix}
  a & -b\\
  c & -d\\
\end{pmatrix}
\]
\[ Z\rho =
\begin{pmatrix}
  1 & 0 \\
  0 & -1 \\
\end{pmatrix}
\begin{pmatrix}
  a & b\\
  c & d\\
\end{pmatrix}
=
\begin{pmatrix}
  a & b\\
  -c & -d\\
\end{pmatrix}
\]
So, we have
\[2
\begin{pmatrix}
  a & b \\
  c & d\\
\end{pmatrix}
\cos(2\theta) - 2i\sin(2\theta)
\begin{pmatrix}
  a & 0\\
  0 & -d\\
\end{pmatrix}
= 2
\begin{pmatrix}
  a(\cos(2\theta) - i\sin(2\theta)) & b\cos(2\theta) \\
  c\cos(2\theta) & d(\cos(2\theta) + i\sin(2\theta))\\
\end{pmatrix}
\, .\]
So, we see that
\[ D_{1}(\rho) =2\frac{1}{2}
\begin{pmatrix}
  e^{i\theta} & 0\\
  0 & e^{-i\theta}\\
\end{pmatrix}
\begin{pmatrix}
  a(\cos(2\theta) - i\sin(2\theta)) & b\cos(2\theta) \\
  c\cos(2\theta) & d(\cos(2\theta) + i\sin(2\theta))\\
\end{pmatrix}
\begin{pmatrix}
  e^{i\theta} & 0\\
  0 & e^{-i\theta}\\
\end{pmatrix}
\]
\[ = 2\frac{1}{2}
\begin{pmatrix}
  e^{i\theta} & 0\\
  0 & e^{-i\theta}\\
\end{pmatrix}
\begin{pmatrix}
  ae^{-i\theta} &  b\cos(2\theta)e^{-i\theta} \\
  c\cos(2\theta)e^{i\theta} & de^{i\theta}\\
\end{pmatrix}
=
\begin{pmatrix}
  a & b\cos(2\theta) \\
  c\cos(2\theta) & d\\
\end{pmatrix}
\]
as expected.

\textbf{(b)} First, we see that
\[Z \rho Z =
\begin{pmatrix}
  1 & 0\\
  0 & -1\\
\end{pmatrix}
\begin{pmatrix}
  a & b\\
  c & d\\
\end{pmatrix}
\begin{pmatrix}
  1 & 0\\
  0 & -1\\
\end{pmatrix}
= 
\begin{pmatrix}
  1 & 0\\
  0 & -1\\
\end{pmatrix}
\begin{pmatrix}
  a & -b\\
  c & -d\\
\end{pmatrix}
=
\begin{pmatrix}
  a & -b\\
  -c & d\\
\end{pmatrix}
\]
and so we see that
\[ D_{2}(\rho) = (1-p)
\begin{pmatrix}
  a & b\\
  c & d\\
\end{pmatrix}
+ p
\begin{pmatrix}
  a & -b\\
  -c & d\\
\end{pmatrix}
=
\begin{pmatrix}
  (1-p)a + pa & (1-p)b - pb \\
  (1-p)c -pc & (1-p)d + pd \\
\end{pmatrix}
=
\begin{pmatrix}
  a & (1-2p)b \\
  (1-2p)c & d \\
\end{pmatrix}
\]
as expected.

\textbf{(c)} The easiest way to start this is to approach it using Dirac notation. In particular, we see we can represent our circuit as
\[ (I\otimes \ket{0}\bra{0} + Z\otimes \ket{1}\bra{1})(R\otimes I)(\ket{0}\otimes \rho= (R\ket{0}\otimes \ket{0}\bra{0} + ZR\ket{0}\otimes \ket{1}\bra{1})(I\otimes \rho) \]
where we know
\[ R\ket{0} = \sqrt{1-p}\ket{0} + \sqrt{p}\ket{1} \quad \& \quad ZR\ket{0} = \sqrt{1-p}\ket{0} - \sqrt{p}\ket{1}\]
and so
\[ U =
\begin{pmatrix}
  \sqrt{1-p} & 0\\
  \sqrt{p} & 0\\
  0 & \sqrt{1-p} \\
  0 & -\sqrt{p} \\
\end{pmatrix}
\]
as required.

\textbf{(d)} We continue from where we left off, in that we know $U$, and we know how $V$ must act on $U$, since it acts only in the $E$ space, we must have that if
\[ V =
\begin{pmatrix}
  v_{1} & v_{2} \\
  v_{3} & v_{4} \\
\end{pmatrix}
\]
then
\[ (V\otimes I)U = 
\begin{pmatrix}
  \sqrt{1-p}v_{1} + \sqrt{p}v_{2} & 0\\
  \sqrt{1-p}v_{3} + \sqrt{p}v_{4} & 0\\
  0 & \sqrt{1-p}v_{1} -\sqrt{p}v_{2} \\
  0 & \sqrt{1-p}v_{3} -\sqrt{p}v_{4} \\
\end{pmatrix}
\, .\]
So, if $\rho$ is the arbitrary matrix from before, we are looking to find
\[
\begin{pmatrix}
  \sqrt{1-p}v_{1} + \sqrt{p}v_{2} & 0\\
  \sqrt{1-p}v_{3} + \sqrt{p}v_{4} & 0\\
  0 & \sqrt{1-p}v_{1} -\sqrt{p}v_{2} \\
  0 & \sqrt{1-p}v_{3} -\sqrt{p}v_{4} \\
\end{pmatrix}
\rho
\left(
\begin{pmatrix}
  \sqrt{1-p}v_{1} + \sqrt{p}v_{2} & 0\\
  \sqrt{1-p}v_{3} + \sqrt{p}v_{4} & 0\\
  0 & \sqrt{1-p}v_{1} -\sqrt{p}v_{2} \\
  0 & \sqrt{1-p}v_{3} -\sqrt{p}v_{4} \\
\end{pmatrix}
\right)^{T}
\]
\[ =
\begin{pmatrix}
  \sqrt{1-p}v_{1} + \sqrt{p}v_{2} & 0\\
  \sqrt{1-p}v_{3} + \sqrt{p}v_{4} & 0\\
  0 & \sqrt{1-p}v_{1} -\sqrt{p}v_{2} \\
  0 & \sqrt{1-p}v_{3} -\sqrt{p}v_{4} \\
\end{pmatrix}
\]

\end{document}
