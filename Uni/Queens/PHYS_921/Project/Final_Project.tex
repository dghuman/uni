\documentclass[10pt]{article}
\usepackage[]{ragged2e}
\usepackage{fancyhdr,amsmath,amsthm,amssymb,bbm,graphicx,array,bm,tensor,braket,mathtools,tensor}
\usepackage{mathtools,tkz-euclide,slashed,url,hyperref}
\usepackage[utf8]{inputenc}
\usepackage[letterpaper,left=25mm,right=25mm,top=15mm]{geometry}

\setlength{\parskip}{1em}
\setlength{\parindent}{0em}

\newcommand{\Z}{\mathbb{Z}}
\newcommand{\R}{\mathbb{R}}
\newcommand{\Q}{\mathbb{Q}}
\newcommand{\C}{\mathbb{C}}
\newcommand{\N}{\mathbb{N}}

\DeclareMathOperator{\Ima}{Im}

\linespread{1.25}
%\pagestyle{fancy}
%\fancyhf{}
%\lhead{PHYS 921 $|$  Assignment 1}

%\rhead{Dilraj Ghuman $|$ 20191345}

\begin{document}
\begin{center}
  {\Large \bf M\"{o}ssbauer Neutrino Oscillations}\\
  {\small \bf Dilraj Ghuman}
\end{center}
\vspace{2em}

\section{Introduction}

In 1958, M\"{o}ssbauer discovered the recoil-less emmission gammas in crystal structures. This behaviour allowed for certain resonance energies to be carried by the photons that would otherwise not be observed in the standard emmission process. A natural progression would thus be to question whether this recoil-less process can be observed with other emission products such as neutrinos, in crystal structures. A potential process that could induce such an interaction would be the decay :
\[^{3} \textnormal{H} \to\,  ^{3}\textnormal{He} + e^{-} + \bar{\nu}_{e}\]
and the detection using the inverse process,
\[^{3}\textnormal{He} + e^{-} + \bar{\nu}_{e} \to\, ^{3} \textnormal{H}\, .\]

\section{External Wave-Packet Formalism}
In essence, the benefit of the \textit{external wave-packet formalism} is to treat the larger nuclei as external to the key internal interactions, like the anti-neutrino and electron in our case. For this reason, we can imagine a Feynman diagram that looks like figure \ref{fig:feyn}.

\tikzset{
  boson/.style={decorate, decoration={snake}},
  middlearrow/.style={
    decoration={markings,
      mark= at position 0.5 with {\arrow{#1}} ,
    },
    postaction={decorate}
  }
}

\begin{figure}[h]
  \centering
    \begin{tikzpicture}[scale=3]
    % Neutrino
    \draw [middlearrow={latex}] (0,0) -- (0.5,0);
    \node [above] at (0.25,0.1) {$\nu$};
    % H_D
    \draw [middlearrow={latex}] (0.5,0) -- (1,0.5);
    \node [right] at (1,0.5) {$\text{H}_{D}$};
    % He_D
    \draw [middlearrow={latex}] (1,-0.5) -- (0.5,0);
    \node [right] at (1,-0.5) {$\text{He}_{D}$};
    % H_S
    \draw [middlearrow={latex}] (-0.5,-0.5) -- (0,0);
    \node [left] at (-0.5,-0.5) {$\text{H}_{S}$};
    % He_S
    \draw [middlearrow={latex}] (0,0) -- (-0.5,0.5);
    \node [left] at (-0.5,0.5) {$\text{He}_{S}$};
    \end{tikzpicture}
    \caption{A simplified Feynman diagram of the interaction we are interested in.}
    \label{fig:feyn}
\end{figure}

\subsection{Quantum Approach}
We have to first go through a simple formalism where we use quantum mechanics to get an estimate on the production rate of neutrinos without accounting for oscillations. We see in the paper that they reference the following hamiltonian for the tritium source decay
\begin{equation}
  H_{S}^{+} = \int d^{3}x\frac{1}{\sqrt{2}}G_{F}\cos\theta_{c}\bra{^{3}\textnormal{He}}J^{\mu}\ket{^{3}\textnormal{H}}\bar{\psi}_{e,S}\gamma_{\mu}(1-\gamma^{5})\psi_{\nu},
\end{equation}
and the helium detection process,
\begin{equation}
  H_{D}^{-} = \int d^{3}x\frac{1}{\sqrt{2}}G_{F}\cos\theta_{c}\bra{^{3}\textnormal{H}}J^{\mu}\ket{^{3}\textnormal{He}}\bar{\psi}_{\nu}\gamma_{\mu}(1-\gamma^{5})\psi_{e,D}\, .
\end{equation}
To calculate the rate of neutrino production, we need to use these hamiltonians in combination with the ground state function(s):
\begin{equation}\label{ground_state}
  \psi_{A,B,0}(\bm{x}, t) = \left[\frac{m_{A},\omega_{A,B}}{\pi}\right]^{\frac{3}{4}}\exp\left[-\frac{1}{2}m_{A}\omega_{A,B}|\bm{x} - \bm{x}_{B}|^{2}\right]e^{-iE_{A,B}t}\, ,
\end{equation}
for $A \in \{\text{H}, \text{He}\}$ and $B\in \{S, D\}$ for the atom type and location respectively. Applying Fermi's golden rule,
\begin{align*}
  \Gamma_{i \to f} & = 2\pi|\bra{f}H_{I}\ket{i}|^{2}\rho(E_{f}) \\
  & = 2\pi|\bra{^{3}\text{He}_{S},\, ^{3}\text{H}_{D}}H_{S}^{+}\ket{^{3}\text{H}_{S},\, ^{3}\text{He}_{D}}|^{2}\rho(E_{f})
\end{align*}
where we need to replace the nuclear states with their corresponding ground state functions described in equation \ref{ground_state}. This is going to get ugly fast as everything is expanded, and this is before we evaluate the $\beta^{-}$ decay processes. In order to avoid becoming over encumbered in this first section (especially since it is before the QFT approach), we note that the leading coeffecients to the $\Gamma_{0}$ term begin to appear as we factor out the $G_{F}$ and $\cos\theta_{c}$ terms. That being said, the final result takes the form
\begin{equation}
  \Gamma_{p} = \Gamma_{0}X_{S}\, ,
\end{equation}
where
\begin{equation}
  \Gamma_{0} = \frac{G^{2}_{F}\cos^{2}\theta_{c}}{\pi}|\psi(R)|^{2}m_{e}^{2}(|M_{V}|^{2}+g^{2}_{A}|M_{A}|^{2})\left(\frac{E_{S,0}}{m_{e}}\right)^{2}\kappa_{S}\, .
\end{equation}
To avoid repeating the paper reviewed, I won't explain the meaning of each term here other than the ones of interest to the derivation. The nuclear matrix elements $M_{V}$ and $M_{A}$ arise from the two different $\beta^{-}$ decay processes, Fermi and Gamow-Teller. The $X_{S}$ term carries the energy and mass terms (and hence the momentum) from the introduced groundstate wavefunctions for the source,
\begin{equation}
  X_{S} = 8\left(\eta_{S} + \frac{1}{\eta_{S}}\right)^{-3}e^{-\frac{p^{2}}{\sigma_{pS}^{2}}}
\end{equation}
where
\begin{equation}
  \eta_{S} = \sqrt{\frac{m_{\text{H}}\omega_{\text{H},S}}{m_{\text{He}}m_{\text{He},S}}}, \quad \sigma^{2}_{pS} = m_{\text{H}}\omega_{\text{H},S} + m_{\text{He}}\omega_{\text{He},S}\, .
\end{equation}
For the cross section, we consider the detection process and recall that the simplest relation of the differential cross-section is related to the scattering amplitude,
\begin{equation}
  \frac{d\sigma}{d\Omega}(\theta, \phi) = |f(\theta,\phi)|^{2}\, .
\end{equation}
We can follow along in \cite{bahcall, Akhmedov_2008} to get the final result as
\begin{equation}
  \sigma(E) = B_{0}X_{D}\delta(E - E_{D,0})\, ,
\end{equation}
where
\begin{equation}
  B_{0} = 4\pi G_{F}^{2}\cos^{2}\theta_{c}|\psi_{e}(R)|^{2}(|M_{V}|^{2} + g_{A}^{2}|M_{A}|^{2})\kappa_{D}\, .
\end{equation}
Here $X_{D}$ is the detection analogue of the source $X_{S}$. The text continues to then combine the production, propogation and detection processes to reach the total rate,
\begin{equation}
  \Gamma = \frac{1}{4\pi L^{2}}\int_{0}^{\infty}\rho(E)\sigma(E)dE \approx \frac{\Gamma_{0}B_{0}}{4\pi L^{2}}X_{S}X_{D}\frac{(\gamma_{S} + \gamma_{D})/2\pi}{(E_{S,0} - E_{D,0})^{2} + (\gamma_{S} + \gamma_{D})^{2}/4}\, ,
\end{equation}
where $\gamma_{S}$ and $\gamma_{D}$ are the energy widths assoiciated with production and detection. This result is used primarily as a comparison with the QFT treatment that is following.

\subsection{QFT Approach}
We want to use our standard Feynman rules for a weak interaction for the neutrino propogation using Figure \ref{fig:feyn}. We can start with the general form the amplitude (\cite{Beuthe_2003}) will take as being
\begin{align}
  \mathcal{A} & = \bra{f}\hat{T}\left(\exp\left(-i\int d^{4}x\mathcal{H}_{I}\right)\right) - \bm{1}\ket{i} \\
  & = \bra{^{3}\text{He}_{S},\, ^{3}\text{H}_{D}}\hat{T}\left(\exp\left(-i\int d^{4}x\mathcal{H}_{I}\right)\right) - \bm{1}\ket{^{3}\text{H}_{S},\, ^{3}\text{He}_{D}}\, ,
\end{align}
where $\mathcal{H}_{I}$ is the interaction lagrangian. Using equation \ref{ground_state} for the states in the bras and kets,
\begin{equation*}
  \mathcal{A} = \bra{^{3}\text{He}_{S},\, ^{3}\text{H}_{D}}\left(\int d^{3}x_{1}dt_{1}\ket{x_{1},t_{1}}\bra{x_{1},t_{1}}\right)\hat{T}\left(\exp\left(-i\int d^{4}x\mathcal{H}_{I}\right)\right)\left(\int d^{3}x_{2}dt_{2}\ket{x_{2},t_{2}}\bra{x_{2},t_{2}}\right)\ket{^{3}\text{H}_{S},\, ^{3}\text{He}_{D}}\, ,
\end{equation*}
but we associate $(x_{1},t_{1})$ with the source and $(x_{2},t_2)$ with the detector. It is also clear enough to see we can introduce another set of $(x_{1}',t_{1}')$ and $(x_{2}', t_{2}')$ to be completely rigorous in how our kets and bras pass by the propogator, but eventually have them removed using the inevitable $\delta$ functions that show up. As such, we get
\begin{align*}
  \mathcal{A} = \int & d^{3}x_{1}dt_{1}\int d^{3}x_{2}dt_{2}\braket{^{3}\text{He}_{S}|x_{1},t_{1}}\braket{x_{1},t_{1}|\,^{3}\text{H}_{S}}\braket{^{3}\text{H}_{D}|x_{2},t_{2}}\braket{x_{2},t_{2}|\,^{3}\text{He}_{D}} \\
  & \cdot \bra{x_{1}, t_{1}}\hat{T}\left(\exp\left(-i\int d^{4}x\mathcal{H}_{I}\right)\right)\ket{x_{2},t_{2}} \, ,
\end{align*}
and subbing in the states,
\begin{align*}
  \mathcal{A} = \int & d^{3}x_{1}dt_{1}\int d^{3}x_{2}dt_{2} \left(\frac{m_{\text{He}},\omega_{\text{He},S}}{\pi}\right)^{\frac{3}{4}}\exp\left[-\frac{1}{2}m_{\text{He}}\omega_{\text{He},S}|\bm{x}_{1} - \bm{x}_{S}|^{2}\right]e^{iE_{\text{He},S}t_{1}}\\
  & \cdot \left(\frac{m_{\text{H}},\omega_{\text{H},S}}{\pi}\right)^{\frac{3}{4}}\exp\left[-\frac{1}{2}m_{\text{H}}\omega_{\text{H},S}|\bm{x}_{1} - \bm{x}_{S}|^{2}\right]e^{-iE_{\text{H},S}t_{1}} \\
  & \cdot \left(\frac{m_{\text{H}},\omega_{\text{H},D}}{\pi}\right)^{\frac{3}{4}}\exp\left[-\frac{1}{2}m_{\text{H}}\omega_{\text{H},D}|\bm{x}_{2} - \bm{x}_{D}|^{2}\right]e^{iE_{\text{H},D}t_{2}}\\
  & \cdot \left(\frac{m_{\text{He}},\omega_{\text{He},D}}{\pi}\right)^{\frac{3}{4}}\exp\left[-\frac{1}{2}m_{\text{He}}\omega_{\text{He},D}|\bm{x}_{2} - \bm{x}_{D}|^{2}\right]e^{-iE_{\text{He},D}t_{2}}\\
  & \cdot \bra{x_{1}, t_{1}}\hat{T}\left(\exp\left(-i\int d^{4}x\mathcal{H}_{I}\right)\right)\ket{x_{2},t_{2}} \, .
\end{align*}

This is beginning to look like the transition amplitude in the text \cite{Akhmedov_2008}. The difficult part begins now in attempting to understand the propagator in this weak interaction with external atoms. It is simplest here to use the Feynman rules for the relevent vertices. We note there will be contributions from the tritium decay, the inverse process, and the propogation of the neutrino and electron. Though all the processes aside from the neutrino are external, they still will complicate this. First, we note that our neutrino is a fermion, so we know the propagator is of the form,
\begin{equation}
  \Pi = \frac{i(\slashed{p} + m_j)}{p_{0}^{2}-\bm{p}^{2} - m_{j}^{2} + i\epsilon}\, ,
\end{equation}
where the $j$ index will relate to the PMNS matrix for oscillations. Next, we need to account for the charged current vertices where our lepton and neutrino interact. Luckily, this is a well known vertex in texts \cite{ew_lec} and takes the form
\begin{equation}
  \mathcal{V} = -i\frac{g}{2\sqrt{2}}\gamma_{\mu}(1 - \gamma_{5}) \, .
\end{equation}
Moreover, we need to account for the external electron as a Dirac spinor using $u_{e,D}$ and $u_{e,S}$ where the first index is for the electron and the second is for the location. We can similarly account for the Helium and Tritium with $u_{\text{He}}$ and $u_{\text{H}}$ respectively. Putting these together, we get
\begin{equation}
  \mathcal{B} = -\left(\frac{g}{2\sqrt{2}}\right)^{2}\int \frac{d^{4}p}{(2\pi)^{4}}e^{-ip_{0}(t_{2} - t_{1}) + i\bm{p}(\bm{x}_{2} - \bm{x}_{1})}\bar{u}_{e,S}\gamma_{\mu}(1-\gamma_{5})\frac{i(\slashed{p} + m_j)}{p_{0}^{2}-\bm{p}^{2} - m_{j}^{2} + i\epsilon}(1 + \gamma_{5})\gamma_{\nu}u_{e,D} \, .
\end{equation}
Now we have to account for the tritium and helium terms in their decay and capture interactions respectively. The vertex of interest is provided by Figure \ref{fig:decay}.

\begin{figure}[h]
  \centering
    \begin{tikzpicture}[scale=3]
    % Neutrino
    \draw [boson] (0,0) -- (0.5,0);
    \node [above] at (0.25,0.1) {$W^{-}$};
    % lepton
    \draw [middlearrow={latex}] (0.5,0) -- (1,0.5);
    \node [right] at (1,0.5) {$\ell$};
    % neutrino
    \draw [middlearrow={latex}] (1,-0.5) -- (0.5,0);
    \node [right] at (1,-0.5) {$\nu_{\ell}$};
    % H
    \draw [middlearrow={latex}] (-0.5,-0.5) -- (0,0);
    \node [left] at (-0.5,-0.5) {$\text{H}$};
    % He
    \draw [middlearrow={latex}] (0,0) -- (-0.5,0.5);
    \node [left] at (-0.5,0.5) {$\text{He}$};
    \end{tikzpicture}
    \caption{The decay of Tritium}
    \label{fig:decay}
\end{figure}



\newpage
%\bibliographystyle{prsty}
\bibliographystyle{plain}

\bibliography{bib}


\end{document}
