\chapter{Introduction}

The cosmic sky has entranced humans for as far as recorded history can trace. As technology evolved, so too did the observation of the universe around us; from the naked eye to primitive telescopes, and eventually to present day space telescopes, like the Hubble Space Telescope and the upcoming James Web Space Telescope \cite{jwst}. These growing technological leaps have also resulted in the exploration of the incredibly small and eventually resulted in the discovery of the neutrino \cite{aneut}. It was perhaps inevitable that these two seemingly separate areas of physics would eventually meet. Neutrino Astronomy is the melding of these two seemingly seperate subjects, where the goal is to ultimately to probe the cosmic sky but using the fundemental particle known as a neutrino.

Neutrinos are an excellent candidate for probing the cosmos due to their neutral charge and low interaction cross-section \cite{pdg}. These fundelemental particles can, for the most part, travel uninhibited by the magnetic/electric fields, dense clouds, and other cosmic debris that litter space. Where photons and electromagnetic radiation can be tampered by these phenomena, neutrinos provide the potential for direct imaging and probing of galactic events. 

Neutrino telescopes are the tool of neutrino astronomy and used to detect neutrinos on Earth. By accounting for the low interaction cross-section, these telescopes compensate with large volumes to maximize detection. By using the interaction processes understood by the standard model, these telescopes can detect signals produced by these interactions to probe oscillations and predict the source. 

In order to utilise these telescopes effectively, the reconstruction of the detected neutrinos is incredibly important. This thesis will cover the reconstruction techniques implemented for the Pacific Ocean Neutrino Explorer, particularly for the reconstruction of muons. Muon provide a big background for the detector, generally from cosmic rays, and characterizing the abaility of the detector to reconstruct the direction the muon has come from can greatly improve its ability to distinguish background from true events. This involves the simulation of events, as the experiment has not been physically realised just yet, processing of the signals, using a linear fit and constructing a likelihood based method for reconstructing tracks. The results of this endeavour are discussed following the reconstruction.






