\chapter{Introduction}

The cosmic sky has entranced humans for as far as recorded history can trace. As technology evolved, so too did the observation of the universe around us; from the naked eye to primitive telescopes, and eventually to present day space telescopes, like the Hubble Space Telescope and the upcoming James Web Space Telescope \textbf{(NEED TO CITE THESE)}. These growing technological leaps have also resulted in the exploration of the incredibly small and eventually resulted in the discovery of the neutrino \cite{aneut}. It was perhaps inevitable that these two seemingly separate areas of physics would eventually meet. 

\section{Neutrinos}

The neutrino is a fundemental particle first proposed by Wolfgang Pauli \cite{nu_proposition}, and then later discovered in 1956 using the byproducts of $\beta^{-}$ decay \cite{aneut}. As research continued into the elusive neutrino, another flavour of neutrino was discovered in 1962 called the muon neutrino ($\nu_{\mu}$) \cite{m_nu} and eventually the final flavour of the tau neutrino ($\nu_\tau$) \cite{t_nu}. 

\subsection{Interactions}

Neutrinos are neutral and interact only through the Weak interaction. The Weak interaction is a force that is mediated by the $W^{\pm}$ and $Z^{0}$ massive bosons, and is the force responsible for decays. The main vertices involved in neutrino interactions are shown in Figure \cite{fig:n_vert}, where the interacting lepton corresponds with the interacting neutrino flavour. 

\begin{figure}
  \centering
  \begin{tikzpicture}[scale=3]
    % W Boson
    \draw [boson] (0,0) -- (0.5,0);
    \node [left] at (0,0) {$W^{\pm}$};
    % lepton
    \draw [middlearrow={latex}] (0.5,0) -- (1,0.5);
    \node [right] at (1,0.5) {$\ell$};
    % neutrino
    \draw [middlearrow={latex}] (1,-0.5) -- (0.5,0);
    \node [right] at (1,-0.5) {$\nu_{\ell}$};
  \end{tikzpicture}
  \hspace{2em}
  \begin{tikzpicture}[scale=3]
    % Z Boson
    \draw [boson] (0,0) -- (0.5,0);
    \node [left] at (0,0) {$Z^{0}$};
    % lepton
    \draw [middlearrow={latex}] (0.5,0) -- (1,0.5);
    \node [right] at (1,0.5) {$\nu_{\ell}$};
    % neutrino
    \draw [middlearrow={latex}] (1,-0.5) -- (0.5,0);
    \node [right] at (1,-0.5) {$\nu_{\ell}$};
  \end{tikzpicture}
  \caption{The Feynmann diagrams for the vertices that would be included in neutrino interactions using the charged $W^{\pm}$ boson on the left and the neutral $Z^{0}$ boson on the right.}
  \label{fig:n_vert}
\end{figure}

All interaction involving neutrino production or detection utilize these vertices in some shape or form. We refer to interactions that use the $W^{\pm}$ boson as the Charged Current (CC) interaction \cite{currents}, and those that use the $Z^{0}$ boson as being Neutral Current (NC) interactions \cite{currents}. 

\subsection{Production \& Sources}

Solar: Neutrinos produced from the fusion process inside the sun. Go into detail about the process.

Atmospheric: Cosmic rays that interact with the atmosphere produce neutrinos that then shower the earth. Go into detail about process

Geo: Rare decays in the crust eject neutrinos which can be detected. 

Active Galactic Nucleus: Active centres of galaxies that are hubs for producing high energy neutrinos

Supernova: Massive Explosions of dying stars that eject high energy neutrinos.

\subsection{Oscillations}

Alongside the discovery of the neutrino and their flavours, another problem arose in the field of neutrino physics: the solar neutrino problem \cite{lowe_nu}. During the 1960's, an experiment was proposed by Bahcall and Davis to measure the solar neutrino flux, referred to the Homestake experiment \cite{davis, bahcall}.

Need to talk about the distance dependence of oscillations, perhaps will need to cover the probability of observing a particular mass of neutrino given the length it travels $L$. Could do the simple two neutrino example, but should probably do the full formula from some source text.






