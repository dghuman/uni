The neutrino, a fundemental particle, offers the potential to image parts of the universe never before seen and can provide an early warning for cosmic events. With their ability to carry information across the universe unperturbed, neutrinos offer a clear image of the cosmos and can provide insight into its nature with relative ease. Learning from successful neutrino telescopes such as IceCube, the Pacific Ocean Neutrino Explorer (P-ONE) will be built in the Cascadia Basin in the Pacific Ocean, supported by an international collaboration. Located 2660 meters below sea level, P-ONE will consist of 70 strings each equipped with at least 20 sensitive photodetectors and 2 calibrators in an infrastructure provided by Ocean Networks Canada. A key step in the data analysis pipeline is the reconstruction of the path of particles as they pass through the detector, which has been outlined in this report. Using simulated data with proprietary code from IceCube, the reconstruction has been analyzed for cosmic muons to understand the efficiency and capabilities of the reconstruction algorithm in accurately reproducing the muon tracks. 
