\documentclass[11pt]{article}
\usepackage[]{ragged2e}
\usepackage{fancyhdr,amsmath,amsthm,amssymb,bbm}
\usepackage[utf8]{inputenc}

\usepackage[letterpaper,left=25mm,right=25mm]{geometry}

\setlength{\parskip}{1em}
\setlength{\parindent}{0em}

\newcommand{\Z}{\mathbb{Z}}
\newcommand{\R}{\mathbb{R}}
\newcommand{\Q}{\mathbb{Q}}
\newcommand{\C}{\mathbb{C}}

\DeclareMathOperator{\Ima}{Im}

\linespread{1.25}
\pagestyle{fancy}
\fancyhf{}
\lhead{PHYS 359 $|$ Assignment 1}

\rhead{Dilraj Ghuman $|$ 20564228}

\begin{document}

\textbf{Problem 1}

\textbf{(a)} To show this relation, we simply do a direct computation.

$$\lim_{a \to 1} \int_{0}^{\infty} \left(-1 \frac{\partial}{\partial a} \right)^{N}e^{-ax}dx = \lim_{a \to 1} \int_{0}^{\infty} (-1)^{N} \left(\frac{\partial}{\partial a} \right)^{N}e^{-ax}dx$$
$$ = \lim_{a \to 1} \int_{0}^{\infty} (-1)^{N}(-1)^{N} x^{N}e^{-ax}dx$$
$$ = \lim_{a \to 1} \int_{0}^{\infty} x^{N}e^{-ax}dx$$
$$ = \int_{0}^{\infty} x^{N}e^{-x}dx$$

\textbf{(b)} This relation can be shown by evaluating the integral first.

$$\lim_{a \to 1} \left(-1 \frac{\partial}{\partial a} \right)^{N} \int_{0}^{\infty}e^{-ax}dx = \lim_{a \to 1} \left(-1\frac{\partial}{\partial a}\right)^{N} \frac{1}{a}$$
$$ = \lim_{a\to 1} (-1)^{N} \prod_{i=1}^{N}(-1)^{i}a^{-(i+1)}$$
$$ = \lim_{a\to 1}(-1)^{N}(-1)^{N}N! \prod_{i=1}^{N} a^{-(i+1)} = N!$$

\textbf{(c)} We define the substitution. Thus, let $x = N + y\sqrt{N}$, then, $dx = dy\sqrt{N}$. Now to get the bounds, as $x \to 0$ we see $y \to -\sqrt{N}$, and as $x \to \infty$, $y \to \infty$.

$$\int_{0}^{\infty} x^{N}e^{-x}dx = \int_{-\sqrt{N}}^{\infty} (N+y\sqrt{N})^{N}e^{-(N+y\sqrt{N})}\sqrt{N}dy$$
$$ = \int_{-\sqrt{N}}^{\infty} N^{N}\left(1+\frac{y\sqrt{N}}{N}\right)^{N}e^{-(N+y\sqrt{N})}\sqrt{N}dy$$
$$ =N^{N}\sqrt{N}e^{-N} \int_{-\sqrt{N}}^{\infty} \left(1+\frac{y\sqrt{N}}{N}\right)^{N}e^{-y\sqrt{N}}dy$$

We now use our approximation by replacing $1+\frac{y\sqrt{N}}{N}$ with $e^{\ln(1+\frac{y\sqrt{N}}{N})}$. Thus,

$$N^{N}\sqrt{N}e^{-N} \int_{-\sqrt{N}}^{\infty} \left(1+\frac{y\sqrt{N}}{N}\right)^{N}e^{-y\sqrt{N}}dy\approx N^{N}\sqrt{N}e^{-N} \int_{-\sqrt{N}}^{\infty} \left(e^{\frac{y\sqrt{N}}{N} - \frac{y^{2}}{2N}}\right)^{N}e^{-y\sqrt{N}}dy$$
$$ \approx N^{N}\sqrt{N}e^{-N} \int_{-\sqrt{N}}^{\infty} e^{- \frac{y^{2}}{2}}dy $$

Thus,

$$N^{N}\sqrt{N}e^{-N} \int_{-\sqrt{N}}^{\infty} e^{- \frac{y^{2}}{2}}dy \approx N!$$

\textbf{(d)} We now have an expression for $N!$, so let us try plugging that in and seeing where we go,

$$\ln(N!) \approx \ln(N^{N}\sqrt{N}e^{-N} \int_{-\sqrt{N}}^{\infty} e^{- \frac{y^{2}}{2}}dy)$$

For $N >> 1$, the lower integral bound approaches $-\infty$. With this in mind, we can compute the integral,

$$ \ln(N^{N}\sqrt{N}e^{-N} \int_{-\sqrt{N}}^{\infty} e^{- \frac{y^{2}}{2}}dy) \approx \ln(N^{N}\sqrt{N}e^{-N} \sqrt{2\pi N})$$
$$ \approx \ln(N^{N}) + \ln(e^{-N}) + \ln(\sqrt{2\pi N})$$

And thus we get,

$$\ln(N!) \approx N\ln(N) - N + \frac{1}{2}\ln(2\pi N)$$

\newpage

\textbf{Problem 2}

\textbf{(a)} This is a fairly simple integral to compute,

$$\int_{0}^{\infty} e^{-wx}dx = -\frac{1}{w}e^{-wx}  \bigg\rvert_{0}^{\infty} $$
$$ = -\frac{1}{w} (0 - 1) = \frac{1}{w} $$

\textbf{(b)} This problem is a bit tougher. The first step we take is to apply the geometric series closed form, 

$$\sum_{j=0}^{\infty} e^{-wj} = \sum_{j=0}^{\infty}\left(e^{-w}\right)^{j}$$
$$ = \frac{1}{1-e^{-w}} $$

Now we expand the taylor series for $e^{x}$ to get,

$$\frac{1}{1-e^{-w}} = \left(1 - \sum_{i=0}^{\infty}\frac{(-w)^{i}}{i!} \right)^{-1}$$

We pull the first term from this series since it is $1$,

$$ \left(1 - \sum_{i=0}^{\infty}\frac{(-w)^{i}}{i!} \right)^{-1} =  \left(\sum_{i=1}^{\infty}\frac{(-w)^{i}}{i!} \right)^{-1}$$

We pull another term for convenience and factor out a $w$,

$$\left(\sum_{i=1}^{\infty}\frac{(-w)^{i}}{i!} \right)^{-1} = \left(w - \sum_{i=2}^{\infty}\frac{(-w)^{i}}{i!} \right)^{-1}$$
$$= \frac{1}{w} \left(1 - \sum_{i=2}^{\infty}\frac{(-w)^{i-1}}{i!} \right)^{-1}$$

Recognize that we can again use a taylor expansion on this new argument, as it takes a similar form to $(1+x)^{-1}$,

$$ \frac{1}{w} \left(1 - \sum_{i=2}^{\infty}\frac{(-w)^{i-1}}{i!} \right)^{-1} = \frac{1}{w} \sum_{k=0}^{\infty} \frac{(-1)^{k}k!}{k!}\left(\sum_{i=2}^{\infty}(-1)^{i-1}\frac{(w)^{i-1}}{i!}\right)^{k}$$
$$ = \frac{1}{w}\sum_{k=0}^{\infty}(-1)^{k}\left(\sum_{i=2}^{\infty}(-1)^{i-1}\frac{w^{i-1}}{i!}\right)^{k}$$

Expanding this double series will result in the terms we are looking for. We show this explicitly for the first few terms.

$$ \frac{1}{w}\sum_{k=0}^{\infty}(-1)^{k}\left(\sum_{i=2}^{\infty}(-1)^{i-1}\frac{w^{i-1}}{i!}\right)^{k} = \frac{1}{w} \left( 1 - \left(-\frac{w}{2!} + \frac{w}{3!} - \dots \right) + \left(-\frac{w}{2!} + \frac{w}{3!} - \dots \right)^{2!} - \left(-\frac{w}{2!} + \frac{w}{3!} - \dots \right)^{3} + \dots \right)$$

$O(w)$ is exactly the first term in the first series, $\frac{w}{2!}$. Notice, if we expand the $\frac{1}{w}$ through, the first term matches our integral and the second term matches what we would expect, $\frac{1}{2}$.

Next, we look at terms of $O(w^{2})$ which results in the following expansion,

$$ \frac{1}{w}\sum_{k=0}^{\infty}(-1)^{k}\left(\sum_{i=2}^{\infty}(-1)^{i-1}\frac{w^{i-1}}{i!}\right)^{k} = \frac{1}{w} \left( 1 + \frac{w}{2!} - \frac{w^{2}}{6} + \frac{w^{2}}{4} + O(w^{3}) \right) $$
$$ =  \frac{1}{w} \left( 1 + \frac{w}{2!} + \frac{w^{2}}{12} + O(w^{3}) \right) $$

And we continue to expand for higher order terms,

$$ \frac{1}{w}\sum_{k=0}^{\infty}(-1)^{k}\left(\sum_{i=2}^{\infty}(-1)^{i-1}\frac{w^{i-1}}{i!}\right)^{k} =  \frac{1}{w} \left( 1 + \frac{w}{2!} + \frac{w^{2}}{12} + \frac{w^{3}}{24} - 2\frac{w^{3}}{2!3!} + \frac{w^{3}}{(2!)^{3}} + O(w^{4}) \right)$$
$$ =  \frac{1}{w} \left( 1 + \frac{w}{2!} + \frac{w^{2}}{12} + w^{3}\left(\frac{1}{24} - \frac{4}{24} + \frac{3}{24} \right)+ w^{4}\left(-\frac{1}{5!} - \frac{1}{24} + \frac{1}{36} - \frac{1}{24} + \frac{1}{16} \right)+ O(w^{5}) \right) $$
$$ = \frac{1}{w} \left(1 + \frac{w}{2} + \frac{w^{2}}{12} - \frac{w^{4}}{720} + w^{5}\left(\frac{1}{6!} - \frac{1}{45} + \frac{7}{96} - \frac{1}{12} + \frac{1}{2^{5}} \right) + O(w^{6})\right)$$
$$ = \frac{1}{w} \left(1 + \frac{w}{2} + \frac{w^{2}}{12} - \frac{w^{4}}{720} + w^{6} \left(-\frac{1}{7!} + \frac{17}{2880} - \frac{137}{4320} + \frac{1}{16} - \frac{5}{96} + \frac{1}{2^{6}} \right) + O(w^{7})\right)$$
$$ = \frac{1}{w} \left(1 + \frac{w}{2} + \frac{w^{2}}{12} - \frac{w^{4}}{720} + \frac{w^{6}}{30240} + O(w^{7})\right)$$

As required.

\end{document}
