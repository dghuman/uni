\documentclass[10pt]{article}
\usepackage[]{ragged2e}
\usepackage{fancyhdr,amsmath,amsthm,amssymb,bbm,graphicx,float}
\usepackage[utf8]{inputenc}
\usepackage[letterpaper,left=25mm,right=25mm]{geometry}

\setlength{\parskip}{1em}
\setlength{\parindent}{0em}

\newcommand{\Z}{\mathbb{Z}}
\newcommand{\R}{\mathbb{R}}
\newcommand{\Q}{\mathbb{Q}}
\newcommand{\C}{\mathbb{C}}

\DeclareMathOperator{\Ima}{Im}

\linespread{1.25}
\pagestyle{fancy}
\fancyhf{}
\lhead{PHYS 359 $|$  Assignment 4}

\rhead{Dilraj Ghuman $|$ 20564228}

\begin{document}
\textbf{4(c)}
\begin{figure}[H]
  \centering
  \includegraphics[scale=0.45]{"4c".png}
  \caption{This is the plot obtained when $C(T)$ is plotted against $T$, where $C(T)$ is in units of $k_{B}$.}
  \label{plot1}
\end{figure}
\textbf{(d)}
\begin{figure}[H]
  \centering
  \includegraphics[scale=0.45]{"4d".png}
  \caption{A plot of $C(T_{max})$ vs $g = g_{e}/g_{0}$ in red and $T_{max}$ vs $g$ in blue.}
  \label{plot2}
\end{figure}
\textbf{5(e)}
\begin{figure}[H]
  \centering
  \includegraphics[scale=0.5]{"5e".png}
  \caption{A plot of $C(T_{max})$ vs $g = g_{e}/g_{0}$ in red and $T_{max}$ vs $g$ in blue.}
  \label{plot2}
\end{figure}
\end{document}
