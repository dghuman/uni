\documentclass[12pt]{article}
\usepackage[]{ragged2e}
\usepackage{fancyhdr,amsmath,amsthm,amssymb,bbm}
\usepackage[utf8]{inputenc}
\usepackage[letterpaper,left=25mm,right=25mm]{geometry}

\setlength{\parskip}{1em}
\setlength{\parindent}{0em}

\newcommand{\Z}{\mathbb{Z}}
\newcommand{\R}{\mathbb{R}}
\newcommand{\Q}{\mathbb{Q}}
\newcommand{\C}{\mathbb{C}}

\DeclareMathOperator{\Ima}{Im}
\DeclareMathOperator{\Rea}{Re}
\DeclareMathOperator{\Arg}{Arg}




\linespread{1.25}
\pagestyle{fancy}
\fancyhf{}
\lhead{PMATH 332 $|$  Assignment 1}

\rhead{Dilraj Ghuman $|$ 20564228}

\begin{document}
\textbf{Problem 1}

\textbf{(a)}
We solve this using simple algebra,

\begin{equation}
  \begin{split}
    \frac{z}{1-z} = 2-3i\\
    z = (1-z)(2-3i)\\
    z + (z-1)(2-3i) = 0\\
    z + 2z - 3iz - 2 + 3i = 0\\
    z(3-3i) = 2-3i\\
  \end{split}
  \rightarrow
  \begin{split}
    z = \frac{2-3i}{3-3i}\\
    z = \frac{(2-3i)(3+3i)}{18}\\
    z = \frac{6+6i-9i+9}{18}\\
    z = \frac{15-3i}{18}\\
  \end{split}
\end{equation}

\textbf{(b)}
We apply the quadratic formula

$$z = \frac{(i-2) \pm \sqrt{(2-i)^{2} + 4(4)(8)}}{16}$$
$$z = \frac{i-2 \pm \sqrt{(4-4i-1) + 128}}{16}$$
$$z = \frac{i-2 \pm \sqrt{131-4i}}{16}$$

\textbf{Problem 2}

\textbf{(a)}
Assume $z = a+bi \in \C$, with $a,b\in\R$. $\Ima\{z\} = b$, and hence, $b > 0$.Furthermore,

$$\frac{1}{z} = \frac{\bar{z}}{z\bar{z}} = \frac{a-bi}{|z|^{2}} = \frac{a-bi}{a^{2}+b^{2}}$$

We aslo have, $\Ima\{\frac{1}{z}\} = -\frac{b}{a^{2}+b^{2}}$. However, $b>0$, thus, $\Ima \{\frac{1}{z}\} < 0$.

\textbf{(b)}
Assume $z\in \C$, $|z| = 1$ but $z \neq 1$. Take $z = a+bi$, $a,b \in \R$. In particular.

$$|z| = 1 \implies \sqrt{a^{2} + b^{2}} = 1 , a \neq 1$$
$$\frac{1}{1-z} = \frac{1-\bar{z}}{(1-z)(1-\bar{z})} = \frac{1-\bar{z}}{1-\bar{z}-z+|z|^{2}} = \frac{1 -(a-bi)}{1-(a-bi)-(a+bi)+a^{2}+b^{2}}$$
$$ = \frac{1-a+bi}{1-2a+a^{2}+b^{2}} $$

In particular, we notice $\Rea \{\frac{1}{1-z} \} = \frac{1-a}{1-2a+a^{2}+b^{2}}$, yet $a \neq 1$. Yet, we note $a^{2} + b^{2}  = 1^{2} = 1$, and thus

$$\frac{1-a+bi}{1-2a+a^{2}+b^{2}} = \frac{1-a}{2(1-a)} = \frac{1}{2} $$

\textbf{Problem 3}

\textbf{(a)}
Notice the real and imaginary part are both $3$, thus

$$\Arg(3+3i) = \theta = \frac{\pi}{4}$$

\textbf{(b)}
To compute $\arg (\frac{1+i}{2\sqrt{3} + 2i})$, we first rewrite it in standard form,

$$\frac{1+i}{2\sqrt{3} + 1} = \frac{(1+i)(2\sqrt{3} -2i)}{4(3)+4} = \frac{2\sqrt{3}-2i+2\sqrt{3}i + 2}{16} = \frac{2(\sqrt{3} + 1) - 2(1-\sqrt{3})i}{16} = \frac{\sqrt{3}+1(\sqrt{3}-1)i}{8}$$

Thus, we can conclude $\arg (\frac{1+i}{2\sqrt{3} + 2i}) = \arctan (\frac{\sqrt{3}-1}{\sqrt{3}+1}) + 2k\pi$.

\textbf{(c)}
We know $-\pi \in \R$, in particular we know it is on the negative side of the real number line, thus, $\Arg (-\pi) = -\pi$.

\newpage

\textbf{Problem 4}

\textbf{(a)}
We convert the right side of the equation to polar form. First find the norm,

$$|-1+\sqrt{3}i| = 2$$

We also have that $\Ima \{-1+\sqrt{3}i \} = \sqrt{3}$ and $\Rea \{-1+\sqrt{3}i \} = -1$, and thus $\theta = \frac{2\pi}{3}$. Hence,

$$z^{3} = -1+\sqrt{3}i = 2e^{i\frac{2\pi}{3}} \implies z = (2)^{\frac{1}{3}}e^{i\frac{2\pi}{9} + \frac{2k\pi}{3}}$$

Thus we have $k \in \{0,1,2\}$, and in particular, $z_{0} = (2)^{\frac{1}{3}}e^{i\frac{2\pi}{9}}$, $z_{1} = (2)^{\frac{1}{3}}e^{\frac{2\pi}{3}(\frac{i}{3} + 1)}$, $z_{2} = (2)^{\frac{1}{3}}e^{\frac{2\pi}{3}(\frac{i}{3} + 2)}$.

\textbf{(b)}
We first get the standard form,

$$z^{5} = \frac{-2i}{1+i} = \frac{-2i(1-i)}{(1+i)(1-i)} = \frac{-2i+2}{2} = 1-i$$

Now we rewrite it in polar from,

$$1-i \implies \theta = \frac{3\pi}{4}$$

and thus, $z^{5} = \sqrt{2}e^{i\frac{3\pi}{20}} \implies z = (2)^{\frac{1}{10}}e^{\frac{\pi}{5}(i\frac{3}{4} + 2k)}$ for $k \in \mathbb{N} , 0 \leq k \leq 4$.

\textbf{Problem 5}

\textbf{(a)}
To compute this argument, we change the complex value into polar form,

$$|\sqrt{3}-i| = \sqrt{3+1} = 2$$
$$\Arg (\sqrt{3}-i) = \frac{5\pi}{6}$$

and thus,

$$z = 2^{7}\left (\cos\left(\frac{35\pi}{6}\right) + i\sin\left(\frac{35\pi}{6}\right)\right)$$

\textbf{(b)}
We solve this series by rewriting it in terms of the exponential form,

$$\sum_{n=0}^{\infty}\frac{\sin\left(n\pi\right)}{2^{n}} = \sum_{n=0}^{\infty}\frac{\Im \{ e^{i\frac{n\pi}{40}} \}}{2^{n}} = \Im \left\{ \sum_{n=0}^{\infty} \frac{e^{i\frac{n\pi}{40}}}{2^{n}} \right\} = \Im \left \{ \sum_{0}^{\infty}\left( \frac{e^{i\frac{\pi}{40}}}{2}\right)^{n} \right \}$$

We recognize that this is exactly the geometric series, $a=1$, $r=\frac{e^{\frac{\pi}{40}}}{2}$, and thus,
$$\Im \left \{ \sum_{0}^{\infty}\left( \frac{e^{i\frac{\pi}{40}}}{2}\right)^{n} \right \} = \Im \left \{ \frac{1-\left(\frac{e^{\frac{i\pi}{40}}}{2}\right)^{21}}{1-\frac{e^{\frac{i\pi}{40}}{2}}{2}} \right \} =  \Im \left \{ \frac{1-\frac{e^{\frac{i21\pi}{40}}}{2^{21}}}{1-\frac{e^{\frac{i\pi}{40}}}{2}} \right \} = \Im \left \{ \frac{\left(1-\frac{e^{\frac{i21\pi}{40}}}{2^{21}}\right )\left(1-\frac{e^{\frac{-i\pi}{40}}}{2}\right )}{1-\frac{e^{\frac{-i\pi}{40}}}{2}-\frac{e^{\frac{i\pi}{40}}}{2} + \frac{1}{4}} \right \}$$
$$= \Im \left \{ \frac{1-\frac{e^{\frac{-i\pi}{40}}}{2}-\frac{e^{\frac{i21\pi}{40}}}{2^{21}} + \frac{1}{2^{22}}e^{i\frac{20\pi}{40}}}{\frac{5}{4}-\frac{1}{2}\left(e^{\frac{i\pi}{40}}+ e^{\frac{-i\pi}{40}}\right)} \right \} = \frac{1}{\frac{5}{4} - \cos\left(\frac{\pi}{40}\right)}\left(\frac{1}{2}\sin\left(\frac{\pi}{40}\right) - \frac{1}{2^{21}}\sin\left(\frac{21\pi}{40}\right) + \frac{1}{2^{22}}\sin\left(\frac{\pi}{2}\right)\right)$$

\textbf{(Bonus)}

\textbf{Problem 6}

Let $p(z)$ be a non-constant polynomial of degree $n \ge 1 \in \mathbb{N}$. Then, by the FTA, $\exists z_{0} \in \C $ s.t. $p(z_{0})=0$. Since $z \in \C$ and $\C$ is a field, then the ring formed by the polynomials over this field is a Euclidean domain, and we can use the Euclidean algorithm

$$p(z) = (z-z_{0})q_{0}(z) + r_{0}(z)$$

for $q_{i}(z), r_{i}(z) \in \mathbb{P}(\C)$, which is the ring of polynomials over $\C$. In particular, notice that,

$$p(z_{0}) = (z_{0}-z_{0})q_{0}(z_{0}) + r_{0}(z_{0})$$
$$0 = r_{0}(z_{0})$$

Thus, we can apply the Euclidean algorithm again to $r_{0}$,

$$r_{0} = (z - z_{0})q_{1} + r_{1}$$

First of all, by the algorithm, the degree of $r_{0}$ is necessarily less than the degree of $q_{0}$, which has a degree of $n-1$ by equality. Doing this recursively, we note after $n$ iterations, the remainder must reduce to $(z-z_{0})$. By the distributive property of rings, we can rewrite this as,

$$p(z) = (z-z_{0})h(z) \hspace{2cm} h(z) = \sum_{i=0}^{n-1}q_{i}$$

we notice $h(z) \in \mathbb{P}(\C)$, and thus by the FTA, $h(z_{1}) = 0$ for $z_{1} \in \C$. We notice that $\deg(h)=n-1$, and naturally we can recursively do this process again for $n-2$ more times. Thus,

$$p(z) = (z-z_{0})(z-z_{1})\dots (z-z_{n-1})$$

As required.

\newpage

\textbf{Problem 7}

We approach this problem in the same way we would any problem with complex numbers and powers, we rewrite it in polar form.

$$(z-2018)^{2n} + (z+2018)^{2n} = 0$$
$$\left( |z-2018|e^{i\Arg (z-2018)} \right)^{2n} + \left( |z+2018|e^{i\Arg (z+2018)} \right) = 0$$
$$|z-2018|^{2n}e^{i2n\Arg (z-2018)} + |z+2018|^{2n}e^{i2n\Arg (z+2018)} = 0$$
$$|z-2018|^{2n}e^{i2n\Arg (z-2018)} = -|z+2018|^{2n}e^{i2n\Arg (z+2018)}$$
$$|z-2018|^{2n}e^{i2n\Arg (z-2018)} = |z+2018|^{2n}e^{i\pi}e^{i2n\Arg (z+2018)}$$

We take $n \ge 1$ and an integer. Furthermore, by definition of equality,

$$|z-2018|^{2n} = |z+2018|^{2n} \hspace{2em} e^{i2n\Arg (z-2018)} = e^{i\pi}e^{i2n\Arg (z+2018)}$$

we also get,

$$i2n\Arg (z-2018) = i\pi + i2n\Arg (z+2018)$$
$$\Arg (z-2018) - \Arg (z+2018) = \frac{\pi}{2n}$$
$$\Arg \left( (z-2018)(\bar{z}+2018)\right) = \frac{\pi}{2n}$$
$$\Arg (|z|^{2} + 2018z - 2018\bar{z} - 2018^{2}) = \frac{\pi}{2n}$$

we let $z = a+bi$, for $a,b\in \R$. Then, expanding,

$$\Arg (a^{2} + b^{2} + 2018a + 2018bi - 2018a + 2018bi - 2018^{2}) = \frac{\pi}{2n}$$
$$\Arg (a^{2} + b^{2} -2018^{2} + (2)2018bi) = \frac{\pi}{2n}$$
$$\arctan \left( \frac{2(2018)b}{a^{2}+b^{2}-2018^{2}} \right) = \frac{\pi}{2n}$$
$$\frac{2(2018)b}{a^{2}+b^{2}-2018^{2}} = \tan\left(\frac{\pi}{2n}\right)$$

We notice that this implies that $a$ is identically zero and that $z$ is necessarily only imaginary.



\end{document}