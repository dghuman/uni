\documentclass[10pt]{article}
\usepackage[]{ragged2e}
\usepackage{fancyhdr,amsmath,amsthm,amssymb,bbm}
\usepackage[utf8]{inputenc}
\usepackage[letterpaper,left=25mm,right=25mm]{geometry}

\setlength{\parskip}{1em}
\setlength{\parindent}{0em}

\newcommand{\Z}{\mathbb{Z}}
\newcommand{\R}{\mathbb{R}}
\newcommand{\Q}{\mathbb{Q}}
\newcommand{\C}{\mathbb{C}}

\DeclareMathOperator{\Ima}{Im}

\linespread{1.25}
\pagestyle{fancy}
\fancyhf{}
\lhead{PMATH 332 $|$  Assignment 6}

\rhead{Dilraj Ghuman $|$ 20564228}

\begin{document}
\textbf{Problem 1}

\textbf{(a)}
We can define $g(z) = \frac{f(z)}{z-2}$. The first thing we notice is that $f(z)$ is the composition of analytic functions, so we can conclude that $f(z)$ must also be analytic. Further, we notice that $\Gamma$ is the \textit{unit} circle, this means that the point at which $g(z)$ is not analytic, $z_{0} = 2$ is not enclosed by this curve. Thus, we can take an open subset $b_{1.1}(0) \subset \C$, which is the open ball of radius $1.1$ centred at $0$. Clearly this set is simply connected and open, and since $f(z)$ is analytic on $\C$, it will be analytic on $b_{1.1}(0)$. The most important part is that now $g(z)$ is also analytic on this subset, and hence by Cauchy's Integral Theorem,
$$\frac{1}{2\pi i}\oint_{\Gamma}\frac{f(z)}{z -2}dz = \frac{1}{2\pi i}\oint_{\Gamma}g(z)dz = 0$$
\textbf{(b)}
We see that
$$f(2) = \exp(\sin(\cos^{2}(2)))$$
but we don't actually compute the value since we see that the last composed function is the exponential, and we know that this never will go to $0$ for any $z \in \C$, so it can't possibly agree with the non-zero value retrieved from $f(2)$.

(\textbf{Note:} Though complex sine and cosine aren't bounded, they are still finite $\forall z\in \C$, so the above argument still holds and the exponential will never be 0)

\textbf{(c)}
Recall that in order to compute the integral in \textbf{(a)} we used the fact that $g(z)$ was analytic over the open subset $b_{1.1}(0)$ and in particular it did not contain the point at which the integrand was not analytic, $z_{2}=2$. In order to use Cauchy's Integral Formula, we require that the point $z_{0} = 2$ be contained by $\Gamma$, but clearly it is not, so the theorem does not actually hold here and hence was not disproven. 

\textbf{Problem 2}

The first case we can consider is the trivial case, $f=g$, since then naturally we would see that $f^{\prime} = g^{\prime}$. The second case we consider is if we can write $f = g + C$ for some $C\in \C$, then we recall that the derivative over $\C$ will be linear, and further we know derivatives of constants vanish, so
$$f^{\prime} = (g + C)^{\prime} = g^{\prime}$$
and we can flip our definition be subtracting $C$ from both sides and get a similar relation for $g$.

% !!!!!!!!!!!!!!!!!! More cases?!!!!!!!!!!!!!!!!!!!!!!!!!!!!!!!!!!!!!!!!!
\newpage
\textbf{Problem 3}

We start with the left side, and we see that our conditions for $U$ allow for us to use Cauchy's Integral Formula,
$$\oint_{\Gamma} \frac{f^{\prime}(z)}{z-z_{0}}dz = 2\pi i f^{\prime}(z_{0})$$
but again we have an Integral formula for derivatives, so
$$2\pi i f^{\prime}(z_{0}) = 2\pi i \left(\frac{1}{2\pi i} \oint \frac{f(z)}{(z-z_{0})^{2}}dz\right) =  \oint \frac{f(z)}{(z-z_{0})^{2}}dz$$
as required.

\textbf{Problem 4}

By the Pok\'e-Lemma, we know that we can turn our radius two closed and simple contour into two independent contours that enclose $z = 1$ and $z = 0$, and since $U = \C \setminus \{0,1\}$ is open, we can always find contours that don't intersect. Let's call these two curves $\Gamma_{1}$ and $\Gamma_{0}$ (so that $\Gamma_{1}$ encloses $\{1\}$), and we see that
$$\oint_{\Gamma}f(z)dz = \oint_{\Gamma_{1}}f(z)dz + \oint_{\Gamma_{0}}f(z)dz$$
$$=\oint_{\Gamma_{1}}\frac{\sin(z)}{z^{2}(z-1)}dz + \oint_{\Gamma_{0}}\frac{\sin(z)}{z^{2}(z-1)}dz$$
we apply the Cauchy Integral formula for the first, and use the Cauchy Integral Formula for derivatives for the second. But, we know that the orientations are all clockwise, so we will end up with negatives
$$=-2\pi \i \frac{\sin(1)}{1^{2}} - 2\pi i \frac{\cos(0)}{-1}$$
$$=2\pi i(1 - \sin(1))$$

\textbf{Problem 5}

\textbf{(a)}
We have that
$$\sum_{k=0}^{\infty}\frac{z_{0}^{k}}{w^{k+1}} = \sum_{k=0}^{\infty}\frac{1}{w}\left(\frac{z_{0}}{w}\right)^{k} = \frac{1}{w} \sum_{k=0}^{\infty}\left(\frac{z_{0}}{w}\right)^{k}$$
where we know that $\frac{z_{0}}{w} < 1$ since $w$ lies on $\Gamma$ but $z_{0}$ lies inside the subset contained by $\Gamma$, which is a unit open ball. This is exactly the condition we need to use the geometric series,
$$\sum_{k=0}^{\infty}\frac{z_{0}^{k}}{w^{k+1}} = \frac{1}{w}\left(\frac{1}{1 - \frac{z_{0}}{w}}\right) = \frac{1}{w - z_{0}}$$
as required.

\textbf{(b)}
We apply what we prove in $\textbf{(a)}$,
$$\oint_{\Gamma}\frac{f(z)}{z-z_{0}}dz = \oint_{\Gamma}f(z)\left(\frac{1}{z-z_{0}}\right)dz = \oint_{\Gamma}f(z)\left(\sum_{k=0}^{\infty}\frac{z_{0}^{k}}{z^{k+1}}\right)dz = \oint_{\Gamma}\left(\sum_{k=0}^{\infty}\frac{f(z)}{z^{k+1}}z_{0}^{k}\right)dz$$

\textbf{(c)}
We sub in what we got for \textbf{(b)} to get,
$$\frac{1}{2\pi i}\oint_{\Gamma}\frac{f(z)}{z - z_{0}}dz = \frac{1}{2\pi i} \oint_{\Gamma}\left(\sum_{k=0}^{\infty}\frac{f(z)}{z^{k+1}}z_{0}^{k}\right)dz = \frac{1}{2\pi i}\sum_{k=0}^{\infty}z_{0}^{k+1}\left(\oint_{\Gamma}\frac{f(z)}{z^{k+1}}dz\right)$$
Apply Cauchy's Integral theorem for derivatives about the origin,
$$\frac{1}{2\pi i}\oint_{\Gamma}\frac{f(z)}{z - z_{0}}dz= \frac{1}{2\pi i}\sum_{k=0}^{\infty}z_{0}^{k}\left(2\pi i\frac{f^{(k)}(0)}{k!}\right) = \sum_{k=0}^{\infty}\frac{f^{(k)}(0)}{k!}z_{0}^{k}$$
but we recognize that this last term is the Taylor expansion of $f(z_{0})$ about $z = 0$, so
$$\frac{1}{2\pi i}\oint_{\Gamma}\frac{f(z)}{z - z_{0}}dz= \sum_{k=0}^{\infty}\frac{f^{(k)}(0)}{k!}z_{0}^{k} = f(z_{0})$$
as required.
\end{document}
