\documentclass[10pt]{article}
\usepackage[]{ragged2e}
\usepackage{fancyhdr,amsmath,amsthm,amssymb,bbm}
\usepackage[utf8]{inputenc}
\usepackage[letterpaper,left=25mm,right=25mm]{geometry}

\setlength{\parskip}{1em}
\setlength{\parindent}{0em}

\newcommand{\Z}{\mathbb{Z}}
\newcommand{\R}{\mathbb{R}}
\newcommand{\Q}{\mathbb{Q}}
\newcommand{\C}{\mathbb{C}}
\newcommand{\Arg}{\text{Arg}}

\DeclareMathOperator{\Ima}{Im}

\linespread{1.25}
\pagestyle{fancy}
\fancyhf{}
\lhead{PMATH 332 $|$  Assignment 3}

\rhead{Dilraj Ghuman $|$ 20564228}

\begin{document}
\textbf{Problem 1}

\textbf{(a)}
Assume $u$ and $v$ are continuous on $U$, and let $z_{0}\in U$. We apply the definition of continuity
$$\lim_{z\to z_{0}} f(z) = \lim_{z\to z_{0}} (u(z) + iv(z)) = \lim_{z\to z_{0}}u(z) + i \lim_{z\to z_{0}}v(z) = u(z_{0}) + iv(z_{0}) = f(z_{0})$$
hence the limit exists and is equal to the required value over all $z_{0} \in U$, thus $f$ is continuous.

\textbf{(b)}
Assume that $f$ is continuous over $U$, and let $z_{0} \in U$. Then, by definition of continuity,
$$\lim_{z\to z_{0}}(f(z)) = f(z_{0})$$
$$\lim_{z\to z_{0}}(u(z) + iv(z)) = u(z_{0}) + iv(z_{0})$$
$$\lim_{z\to z_{0}}u(z) + i\lim_{z \to z_{0}}v(z) = u(z_{0}) + iv(z_{0})$$
and by definition of equality in complex numbers we see,
$$\lim_{z\to z_{0}}u(z) = u(z_{0}) \hspace{1em} \& \hspace{1em} \lim_{z \to z_{0}}(v(z)) = v(z_{0})$$

\textbf{(c)}
To deduce the continuous functions being continuous, we decompose them into functions of $u$ and $v$. First, notice that $f = e^{z} = e^{a + ib} = e^{a}e^{ib} = e^{a}\cos(b) + ie^{a}\sin(b)$, where the real an imaginary components are continuous over $\R$. Similarly, $f(z) = \bar{z} = \overline{a + ib} = a - ib$, and $f(z) = |z| = |a+ib| = a^{2}-b^{2} + 2iab$, where the component functions are clearly continuous functions.

\textbf{Problem 2}

\textbf{(a)} Take $z_{0} \in \{z\in\C \setminus \{0\}: z\notin \R^{-}\}$. Then, I postulate that
$$\lim_{z\to z_{0}} \Arg(z) = \Arg(z_{0})$$
To see this, we apply the definition of a limit. First, notice that $\Arg(z) = \Arg(\frac{z}{|z|})$, so we can assume that $|z| = 1$ without loss of generality. In particular, let $\varepsilon > 0$ with the corresponding $\delta = \sqrt{2 - 2\cos(\varepsilon)}$. This choice of $\delta$ is motivated geometrically, since $\varepsilon$ will bound the angle between $z$ and $z_{0}$, which both have norm 1, we know that the difference between the two will be described by the cosine law, where $z \in \{z\in\C \setminus \{0\}: z\notin \R^{-}\}$.
$$|z - z_{0}| < \delta \hspace{1em} \to \hspace{1em} |z - z_{0}| < \sqrt{2 - 2\cos(\varepsilon)}$$
We see that, by the cosine law,
$$|\Arg(z) - \Arg(z_{0})| = | \arccos \left(\frac{|z|^{2} + |z_{0}|^{2} - |z-z_{0}|^{2}}{2|z||z_{0}|}\right) | = |\arccos \left(\frac{2 - |z-z_{0}|^{2}}{2}\right)|$$
$$=|\arccos \left(1 - \frac{1}{2}|z - z_{0}|^{2}\right)| < |\arccos\left(1 - \frac{1}{2}\delta^{2}\right)| = |\arccos\left(1-\frac{1}{2}(\sqrt{2 -2\cos(\varepsilon)})^{2}\right)| = |\varepsilon| = \varepsilon$$
Hence the limit exists and is $\Arg(z_{0})$. Notice that this is only well defined since $\varepsilon < 2\pi$ due to the limited codomain. This argument would not hold if the codomain was any larger.

\textbf{(b)} To see that $\Arg$ is continuous on the negative real line $\R^{-}$, we consider the fact that $\Arg(z)$ where $z \in \R^{-}$ is the constant function $\Arg(z) = \pi$. Yet the constant function is trivially continuous. Thus $\Arg$ is continuous over $\R^{-}$.

\textbf{Problem 3}

\textbf{(a)}
First we write the function $e^{z}$ in the form $u + iv$. Since $z\in\C$, $\exists x,y\in\R$ such that $z= x+iy$. Then,
$$e^{z} = e^{x+yi} = e^{x}e^{yi} = e^{x}(\cos(y) + i\sin(y)) = e^{x}\cos(y) + ie^{x}\sin(y)$$
Taking the components and differentiating,
$$u_{x} = \frac{\partial}{\partial x}(e^{x}\cos(y)) = e^{x}\cos(y) \hspace{1em} \& \hspace{1em} v_{y} = \frac{\partial}{\partial y}(e^{x}\sin(y)) = e^{x}\cos(y)$$
$$u_{y} = \frac{\partial}{\partial y}(e^{x}\cos(y)) = -e^{x}\sin(y) \hspace{1em} \& \hspace{1em} v_{x} = \frac{\partial}{\partial x}(e^{x}\sin(y)) = e^{x}\sin(y)$$
We see that $u_{x} = v_{y}$ and $u_{y} = -v_{x}$ which satisfy the Cauchy-Riemann equations. Next, we know that $|z| = \sqrt{z\bar{z}}$, and since we already have that $z = x+iy$, we see
$$|z| = \sqrt{z\bar{z}} = \sqrt{(x+iy)(x-iy)} = \sqrt{x^{2}+y^{2}}$$
and further
$$u_{x} = \frac{\partial}{\partial x}(\sqrt{x^{2} + y^{2}}) = x(x^{2} + y^{2})^{-\frac{1}{2}} \hspace{1em} \& \hspace{1em} v_{y} = \frac{\partial}{\partial y}(0) = 0$$
We stop here since $u_{x} \neq v_{y}$ which already disagrees with the Cauchy-Riemann equations.

\textbf{(b)}
I claim that $|z|$ is analytic over the real and imaginary lines sans $\{0\}$, individually. This forms four portions of $\C$ over which the modulus is continuous, $\R^{+}, \R^{-}, \mathbb{I}^{+}$ and $\mathbb{I}^{-}$, where $\mathbb{I}^{\pm} = \{ z\in\C |z = ai, a \in \R^{\pm}\}$. First, take $z\in \R^{+}$, then,
$$|z| = z$$
which we know is differentiable over the reals. Similarly, if $z_{1}\in\R^{-}, z_{2}\in\mathbb{I}^{+}$ and $z_{3}\in\mathbb{I}^{-}$, then we see that, with $z_{2} = ai, z_{3} = bi$,
$$|z_{1}| = -z \hspace{1em} |z_{2}| = a \hspace{1em} |z_{3}| = -b$$
which are all the same identity function with modified domains, but equivalent to the real valued one, and hence differentiable,
\newpage
\textbf{Problem 4}

\textbf{(a)}
Consider the function $f = e^{x}\cos(y) + ie^{x}\sin(y)$. We showed in \textbf{Problem 3} that the Cauchy-Riemann equations are satisfied by this function, and further, we note these are satisfied over all of $\C$ with the partials being continuous, and hence being continuous over all of $\C$, which is open.Thus, $u_{x},u_{y},v_{x}$ and $v_{y}$ will be continuous over some open ball $b_{r}(z_{0})$ $\forall z_{0} \in \C$ and $r>0$. By theorem, $f$ is differentiable.

Consider $g = e^{x}\sin(y) + ie^{x}\cos(y)$, which swaps the $u$ and $v$ from $f$. I propose this is not differentiable. To see this, consider the complex number, $z \in \C$, with zero as the imaginary component, $z = x \in \R$. We take two limits, one vertically and one horizontally. Let $h\in \R$, then,
$$\lim_{h\to 0}\left(\frac{g(z + h) - g(z)}{h}\right) = \lim_{h\to 0}\left(\frac{e^{x+h}(0) + ie^{x+h}(1) - e^{x}(0) -ie^{x}(1) }{h}\right) = \lim_{h\to 0}\left(\frac{0}{h}\right) = 0$$
$$\lim_{h\to 0}\left(\frac{g(z + ih) - g(z)}{ih}\right) =\lim_{h\to 0}\left(\frac{e^{x}\sin(h) + ie^{x}\cos(h) - e^{x}(0) -ie^{x}(1) }{ih}\right) = -ie^{x}\lim_{h\to 0}\left(\frac{\sin(h)}{h}\right) + e^{x}\lim_{h\to 0}\left(\frac{\cos(h)-1}{h}\right)$$
We recognize that the last two limits are in indeterminant form, thus we apply L'H\^opital's rule and get,
$$-ie^{x}\lim_{h\to 0}\left(\frac{\sin(h)}{h}\right) + e^{x}\lim_{h\to 0}\left(\frac{\cos(h)-1}{h}\right) = -ie^{x}\lim_{h\to 0}\left(\frac{\cos(h)}{1}\right) + e^{x}\lim_{h\to 0}\left(\frac{-\sin(h)}{1}\right) = -ie^{x}(1) + e^{x}(0) = -ie^{x}$$
Clearly the limits do not agree, hence $g$ is not differentiable over $\C$.

\textbf{(b)}
Since $f$ is differentiable, then by theorem, since $f = u + iv$ we have that the partials must exist and satisfy the C-R equations, hence,
$$u_{x} = v_{y} \hspace{1em} \& \hspace{1em} u_{y} = -v_{x}$$
we notice that for $h = v - iu$, that $u^{\prime} = v$ and $v^{\prime} = -u$, then,
$$u^{\prime}_{x} = v_{x} = -u_{y} = (-u)_{y} = v^{\prime}_{y} \hspace{1em} \& \hspace{1em}u^{\prime}_{y} = v_{y} = u_{x} = -(-u)_{x} = -v^{\prime}_{x}$$
which satisfy the C-R equations. Next, due to the differentiability of $f$, we have that the partials $u_{x},u_{y}, v_{x}$ and $v_{y}$ must exist, and hence so do the partials for $h$ over all of $\C$. Now all we require is that they are continuous over some $b_{r}(z_{0})$ for $z_{0}\in\C$ and $r>0$.

Since $f$ is differentiable over $\C$, it is also analytic and holomorphic. Then, $f^{\prime}$ is also anlaytic, and hence holomorphic. Yet this implies that the partials of the components must also be continuous over some open ball around all $z_{0} \in \C$.

Hence, by theorem, we have that $h$ is differentiable.
\newpage
\textbf{Bonus}

\textbf{Problem 5}

Consider the real valued function $f(x) = x^{\frac{23}{3}}$ at the point $x_{0} = 0$. We see that the function is indeed differentiable at $x_{0}$, and can be shown quiet easily by definition,
$$\lim_{h\to 0}\frac{f(0 + h) - f(0)}{h} = \lim_{h\to 0}\frac{h^{\frac{23}{3}} - 0}{h} = \lim_{h\to 0}h^{\frac{20}{3}}= 0$$
Clearly this can be continued 6 more times, since the power of the function will be greater than 1. In particular,
$$\lim_{h\to 0}\frac{f(0 + h) - f(0)}{h} = \lim_{h\to 0}\frac{h^{\frac{20}{3}} - 0}{h} = \lim_{h\to 0}h^{\frac{17}{3}}= 0$$
$$\vdots$$
$$\lim_{h\to 0}\frac{f(0 + h) - f(0)}{h} = \lim_{h\to 0}\frac{h^{\frac{5}{3}} - 0}{h} = \lim_{h\to 0}h^{\frac{2}{3}}= 0$$
Now, however, we again try to differentiate at this point,
$$\lim_{h\to 0}\frac{f(0 + h) - f(0)}{h} = \lim_{h\to 0}\frac{h^{\frac{2}{3}} - 0}{h} = \lim_{h\to 0}h^{-\frac{1}{3}}$$
This is undefined, as it blows up to $\infty$. Hence, $f$ is a real function that is $7^{th}$ order differentiable at $x_{0}$ but not $8^{th}$ order differentiable.

\textbf{Problem 6}

To approach this problem, we first realize it is easier to work with the right side than it is the left. This will become apparent as we work. Hence, consider the right hand side of the equation, where we apply the definition of differentiability,
$$\frac{f^{\prime}(z_{0})}{g^{\prime}(z_{0})} = \frac{\lim_{h\to 0}\frac{f(z_{0} + h) - f(z_{0})}{h}}{\lim_{h\to 0}\frac{g(z_{0} + h) - g(z_{0})}{h}}$$
where $h\in \C$. By supposition we know that $g^{\prime}(z_{0}) \neq 0$, so we can apply our limit property,
$$\frac{\lim_{h\to 0}\frac{f(z_{0} + h) - f(z_{0})}{h}}{\lim_{h\to 0}\frac{g(z_{0} + h) - g(z_{0})}{h}}=\lim_{h\to 0}\frac{\frac{f(z_{0} + h) - f(z_{0})}{h}}{\frac{g(z_{0} + h) - g(z_{0})}{h}} = \lim_{h\to 0}\frac{f(z_{0}+h) -f(z_{0})}{g(z_{0} + h) - g(z_{0})}$$
again, applying the limit property,
$$\lim_{h\to 0}\frac{f(z_{0}+h) -f(z_{0})}{g(z_{0} + h) - g(z_{0})} = \frac{\lim_{h\to 0}f(z_{0}+h) -f(z_{0})}{\lim_{h\to 0}g(z_{0} + h) - g(z_{0})} = \frac{f(z_{0}) - f(z_{0})}{g(z_{0}) - g(z_{0})} = \frac{0}{0}$$
yet, again by supposition, we have $\lim_{z\to z_{0}}f(z) = \lim_{z\to z_{0}}g(x) = 0$, thus,
$$\frac{f(z_{0}) - f(z_{0})}{g(z_{0}) - g(z_{0})} = \frac{0}{0} = \frac{\lim_{z\to z_{0}}f(z)}{\lim_{z\to z_{0}}g(z)} = \lim_{z\to z_{0}}\frac{f(z)}{g(z)}$$
as required.

\end{document}
