\documentclass[10pt]{article}
\usepackage[]{ragged2e}
\usepackage{fancyhdr,amsmath,amsthm,amssymb,bbm}
\usepackage[utf8]{inputenc}
\usepackage[letterpaper,left=25mm,right=25mm]{geometry}

\setlength{\parskip}{1em}
\setlength{\parindent}{0em}

\newcommand{\Z}{\mathbb{Z}}
\newcommand{\R}{\mathbb{R}}
\newcommand{\Q}{\mathbb{Q}}
\newcommand{\C}{\mathbb{C}}
\newcommand{\Arg}{\text{Arg}}

\DeclareMathOperator{\Ima}{Im}

\linespread{1.25}
\pagestyle{fancy}
\fancyhf{}
\lhead{PMATH 332 $|$  Assignment 4}

\rhead{Dilraj Ghuman $|$ 20564228}

\begin{document}
\textbf{Problem 1}

\textbf{(1)}
We apply the definition of $\sin(z)$ in $\C$,
$$\sin(z) = 2i \hspace{1em} \to \hspace{1em} \frac{e^{iz}-e^{-iz}}{2i} = 2i$$
$$e^{iz}-e^{-iz} = (2i)(2i)$$
Now since $z\in \C$, we know $\exists$ $a,b\in\R$ such that $z = a+bi$, thus,
$$e^{i(a + bi)} - e^{-i(a+bi)} = -4$$
$$e^{ia}e^{-b} - e^{-ia}e^{b} = -4$$
We apply the definition of the complex exponential,
$$e^{-b}(\cos(a) + i\sin(a)) - e^{b}(\cos(a) - i\sin(a)) = -4$$
$$\left(e^{-b}-e^{b}\right)\cos(a) + i\left(e^{-b} + e^{b}\right)\sin(a) = -4$$
by equality, we notice that $\sin(a)$ must be $0$, since $-4$ has no imaginary component. Thus we conclude $a = n\pi$ where $n \in \Z$. With this known, we have,
$$\left(e^{-b}-e^{b}\right)\cos(n\pi) = -4 \hspace{1em} \to \hspace{1em}\left(e^{-b}-e^{b}\right)(-1)^{n} = -4$$
$$e^{-b} - e^{b} = (-1)^{n}(-1)(4)$$
$$e^{-b} - e^{b} = (-1)^{n+1}4$$
$$e^{2b} + 4(-1)^{n+1}e^{b} - 1 = 0$$
which is a quadratic in $e^{b}$. Applying the quadratic formula,
$$e^{b} = \frac{4(-1)^{n} \pm \sqrt{16 - 4(-1)}}{2}$$
$$e^{b} = \frac{4(-1)^{n} \pm \sqrt{20}}{2}$$
$$e^{b} = 2(-1)^{n} \pm \sqrt{5}$$
Notice that the solution to this is dependent on $n$. Thus, we see that the final solution is $z = a+ib$ where, for $n\in \Z$, $a = n\pi$ and if $n = 0$ mod $2$, $b = \ln(2 + \sqrt{5})$, otherwise $b = \ln(\sqrt{5} - 2)$.

\newpage
\textbf{(2)}
Apply the definition of $\cos(z)$,
$$\cos(z) = \sqrt{2} \hspace{1em} \to \hspace{1em} \frac{e^{iz} + e^{-iz}}{2} = \sqrt{2}$$
Again, let $z = a+bi$, we get,
$$e^{i(a+bi)} + e^{-i(a+bi)} = 2\sqrt{2}$$
$$e^{-b}(\cos(a) + i\sin(b)) + e^{b}(\cos(a) - i\sin(a)) = 2\sqrt{2}$$
$$\left(e^{-b} + e^{b}\right)\cos(a) + i\left(e^{-b} - e^{b}\right)\sin(a) = 2\sqrt{2}$$
again, no imaginary component implies $a = n\pi$ for $n \in \Z$. Hence,
$$\left(e^{-b}+e^{b}\right)\cos(n\pi) = 2\sqrt{2} \hspace{1em} \to \hspace{1em} \left(e^{-b}+e^{b}\right)(-1)^{n} = 2\sqrt{2} $$
$$1 + e^{2b} = (-1)^{n}2\sqrt{2}e^{b}$$
$$e^{2b} + (-1)^{n+1}2\sqrt{2}e^{b} + 1 = 0$$
Again, a quadratic in $e^{b}$, thus we see
$$e^{b} = \frac{(-1)^{n}2\sqrt{2} \pm \sqrt{(4)(2) - 4}}{2}$$
$$e^{b} = \frac{(-1)^{n}2\sqrt{2} \pm \sqrt{4}}{2}$$
$$e^{b} = (-1)^{n}\sqrt{2} \pm 1$$
In this case we notice this is only a solution for $n=0$ mod $2$. Thus, we have that,
$$a = 2n\pi \hspace{2em} b = \ln(\sqrt{2} - 1) \hspace{1em} \text{or} \hspace{1em} b = \ln(\sqrt{2} + 1)$$
where $n \in \Z$.

\textbf{(3)}
We apply the definition of both $\cos(z)$ and $\sin(z)$ in $\C$,
$$cos(z)=i\sin(z) \hspace{1em} \to \hspace{1em} \frac{e^{iz} + e^{-iz}}{2} = i\left(\frac{e^{iz} - e^{-iz}}{2i}\right)$$
$$e^{iz} + e^{-iz} = e^{iz} - e^{-iz} \hspace{1em} \to \hspace{1em}2e^{-iz} = 0$$
We can assume $z = a+ib$, hence,
$$e^{-iz} = 0 \hspace{1em} \to \hspace{1em} e^{-i(a +ib)} = 0$$
$$e^{b}e^{-ia} = 0$$
Since $e^{b} \neq 0$ $\forall b\in\R$, we get that,
$$e^{-ia} = 0 \hspace{1em} \to \hspace{1em} \cos(a) - i\sin(a) = 0$$
yet, this implies that
$$\cos(a) = 0 \hspace{1em} \& \hspace{1em} \sin(a) = 0$$
which is not possible for any $a \in \R$. Thus there are no solutions.

\newpage
\textbf{Problem 2}

\textbf{(a)}
Start with the assumption that $z = a +ib$ for some $a,b\in \R$. By assumption, $\Im{z} \geq 0$ implies that $b \geq 0$. Then,
$$|e^{iz}| = |e^{i(a+ib)}| = |e^{-b}e^{ia}| = |e^{-b}||e^{ia}|$$
but since $a\in \R$, we know that $|e^{ia}| \leq 1$, and hence,
$$|e^{-b}||e^{ia}| \leq |e^{-b}|$$
but we see that by assumption $b\geq 0$, and hence it is necessary that,
$$|e^{-b}| \leq 1$$

\textbf{(b)}
In the previous assignment, we proved that the L'h\^opital's rule still holds for complex functions. In particular, notice
$$\lim_{z \to 0}\sin(z) = 0 \hspace{2em} \& \hspace{2em} \lim_{z\to 0} z = 0$$
hence we have an indeterminant form. Since both of the functions are analytic, we apply L'h\^opital's rule
$$\lim_{z\to 0}\frac{(\sin(z))^{\prime}}{(z)^{\prime}} = \lim_{z\to 0} \frac{\cos(z)}{1} = \lim_{z\to 0}\cos(z) = 0$$

\textbf{(c)}
We use part \textbf{(b)} to deduce that this limit diverges. In particular
$$\lim_{z\to 0}\frac{\sin(z)}{z^{2}} = \left(\lim_{z\to 0}\frac{\sin(z)}{z}\right)\left(\lim_{z\to 0}\frac{1}{z}\right) = (1)\left(\lim_{z\to 0}\frac{1}{z}\right)$$
to see that the remaining limit diverges, we take some $M >0$. Then, let $\delta = \frac{1}{M}$, and we see that if $z\in b_{\delta}(0)\setminus 0$, then
$$|z| \leq \frac{1}{M} \implies M \leq \frac{1}{|z|}$$
as required.

\textbf{Problem 3}

\textbf{(a)}
To check the Cauchy-Riemann equations we first need to separate the Log into two functions. This follows naturally from definition
$$Log(z) = \ln(|z|) + i\Arg(z) = \ln(\sqrt{x^{2} + y^{2}}) + i\Arg(x+iy)$$

\newpage
so we check directly,
$$u_{x} = \frac{\partial}{\partial x} \ln(\sqrt{x^{2} + y^{2}}) = \frac{1}{\sqrt{x^{2} + y^{2}}}\left(\frac{x}{\sqrt{x^{2} + y^{2}}}\right) = \frac{x}{x^{2} + y^{2}}$$
$$u_{y} = \frac{\partial}{\partial y} \ln(\sqrt{x^{2} + y^{2}}) = \frac{1}{\sqrt{x^{2} + y^{2}}}\left(\frac{y}{\sqrt{x^{2} + y^{2}}}\right) = \frac{y}{x^{2} + y^{2}}$$
$$v_{x} = \frac{\partial}{\partial x}\Arg(x + iy) = \frac{\partial}{\partial x}\arctan\left(\frac{y}{x}\right) = \frac{1}{1 + \left(\frac{y}{x}\right)^{2}}\frac{-y}{x^{2}} = \frac{-y}{x^{2} + y^{2}}$$
$$v_{y} = \frac{\partial}{\partial y}\Arg(x + iy) = \frac{\partial}{\partial y}\arctan\left(\frac{y}{x}\right) = \frac{1}{1 + \left(\frac{y}{x}\right)^{2}}\frac{1}{x} = \frac{x}{x^{2} + y^{2}}$$
Clearly we see that,
$$u_{x} = v_{y} \hspace{2em} \& \hspace{2em} u_{y} = -v_{x}$$
which satisfy the Cauchy-Riemann equations.

\textbf{(b)}
We have that the partials $u_{x},u_{y},v_{x}$ and $v_{y}$ exist on $\C\setminus (\{0\} \cup \R^{-})$. Further, we recognize that the partials are continuous on all of $\C\setminus (\{0\}\cup \R^{-})$, since the discontinuity is excluded, namely $\{0\}\cup \R^{-}$. Thus, we have that the partials exist, are continuous on some finite open ball around every point, and satisfy the Cauchy-Riemann equations. By theorem, $Log(z)$ is analytic.

\textbf{(c)}
We showed in \textbf{(a)} what the partials to this complex function were. Since the Cauchy-Riemann equations have been satisfied, we can use the partials to compute the full derivative. We use the $x$ partials, though the choice is arbitrary
$$\frac{d}{dz}Log(z) = u_{x} + iv_{x} =  \frac{x}{x^{2} + y^{2}} + \frac{-iy}{x^{2} + y^{2}} = \frac{x -iy}{x^{2} + y^{2}} = \frac{x-iy}{(x+iy)(x-iy)} = \frac{1}{x+iy} = \frac{1}{z}$$
as we would expect.

\textbf{(d)}
We know that the exponential function is analytic and we just showed that the complex logarithm is analytic over the principle branch. Furthermore, we have that composition of analytic functions will be analytic, albeit, over the intersection of the domains. Thus, the function $z^{\alpha} = e^{\alpha Log(z)}$, which is exactly the composition of these two functions, will be analytic over the principle branch.

\textbf{(e)}
We apply the chain rule. In particular, if we let $f(z) = \alpha Log(z)$, then we have that,
$$\frac{d}{dz}z^{\alpha} = \frac{d}{dz}e^{\alpha Log(z)} = \frac{d}{dz}e^{f(z)} = f^{\prime}(z)e^{f(z)} = (\alpha Log(z))^{\prime}e^{\alpha Log(z)}$$
but we just computed the derivative in \textbf{(c)}. Hence, we see,
$$ (\alpha Log(z))^{\prime}e^{\alpha Log(z)} = \frac{1}{z}e^{\alpha Log(z)} = \frac{1}{z}z^{\alpha}= z^{\alpha -1}$$
as required.

\newpage
\textbf{(BONUS)}

The error in the proof occurs when we initially set up the definition of the limit for the composition of the functions. In particular, if $f$ was a constant function in $z$, then the limit
$$\lim_{z\to z_{0}}\frac{h(f(z)) - h(f(z_{0}))}{z-z_{0}}$$
would not make sense since the function $h$ would not be varying over any neighbourhood; $h$ would be fixed to a single point. Since differentiability is meaningless unless it is defined over some neighbourhood, or open ball, this limit is not sufficient for the proof.

Assume that $f:U\to\C$ is analytic an $h:W\to\C$ is analytic as well, where $U,W\subset\C$ and $f(U)\subset W$. Then, we consider the differentiability of the composition of the two functions at a fixed point $z_{0}\in U$. By definition, for $k\in \C$
$$\lim_{k\to 0}\frac{h(f(z_{0}) + k) - h(f(z_{0}))}{k}$$
now we notice that this limit allows for $f$ to be constant, since there will still be variance in $h$ around the point $f(z_{0})$. In particular, since $f$ is differentiable, it is also continuous and thus
$$\lim_{k\to 0}\frac{h(f(z_{0}) + k) - h(f(z_{0}))}{k} = \lim_{k\to 0}\frac{h(f(z_{0}) + k) - h(f(z_{0}))}{f(z_{0} + k) - f(z_{0})}\frac{f(z_{0} + k) - f(z_{0})}{k}$$
where we apply the definition of differentiability to both parts and receive what we expect.
$$\lim_{k\to 0}\frac{h(f(z_{0}) + k) - h(f(z_{0}))}{f(z_{0} + k) - f(z_{0})}\frac{f(z_{0} + k) - f(z_{0})}{k} = \left(\lim_{k\to 0}\frac{h(f(z_{0}) + k) - h(f(z_{0}))}{f(z_{0} + k) - f(z_{0})}\right)\left(\lim_{k\to 0}\frac{f(z_{0} + k) - f(z_{0})}{k}\right)$$
$$ = h^{\prime}(f(z_{0}))f^{\prime}(z_{0})$$
due to the differentiability of both $f$ and $h$.

\end{document}
