\documentclass[10pt]{article}
\usepackage[]{ragged2e}
\usepackage{fancyhdr,amsmath,amsthm,amssymb,bbm}
\usepackage[utf8]{inputenc}
\usepackage[letterpaper,left=25mm,right=25mm]{geometry}

\setlength{\parskip}{1em}
\setlength{\parindent}{0em}

\newcommand{\Z}{\mathbb{Z}}
\newcommand{\R}{\mathbb{R}}
\newcommand{\Q}{\mathbb{Q}}
\newcommand{\C}{\mathbb{C}}

\DeclareMathOperator{\Ima}{Im}

\linespread{1.25}
\pagestyle{fancy}
\fancyhf{}
\lhead{AMATH 353 $|$  Assignment 5}

\rhead{Dilraj Ghuman $|$ 20564228}

\begin{document}
\textbf{Problem 1}

\textbf{(a)}
Assume that $u(x,y,t) = T(t)S(x,y)$, then our PDE becomes
$$u_{t} = k(u_{xx} + u_{yy}) \hspace{1em} \implies \hspace{1em} T^{\prime}S = k(TS_{xx} + TS_{yy})$$
$$\frac{T^{\prime}}{T} = \frac{k(S_{xx} + S_{yy})}{S}$$
where we let $\lambda$ be our separation constant, and hence
$$T^{\prime} + \lambda T = 0 \hspace{2em} \& \hspace{2em} S_{xx} + S_{yy} = -\frac{\lambda}{k}S$$
Now let $S(x,y) = X(x)Y(y)$, so we get
$$S_{xx} + S_{yy} = -\frac{\lambda}{k}S \hspace{1em} \implies \hspace{1em} X^{\prime \prime}Y + XY^{\prime \prime} = -\frac{\lambda}{k}XY$$
$$\frac{X^{\prime \prime}}{X} + \frac{Y^{\prime \prime}}{Y} = -\frac{\lambda}{k}$$
$$\frac{X^{\prime \prime}}{X} = -\frac{\lambda}{k} - \frac{Y^{\prime \prime}}{Y}$$
where we let $\mu^{2}$ be the separation constant here, and we get
$$X^{\prime \prime} = -\mu^{2} X \hspace{2em} \& \hspace{2em} Y^{\prime \prime} = -\left(\frac{\lambda}{k} - \mu_{2}\right)Y$$
We look at the boundary and initial conditions, and we see that for non-trivial spatial solutions we get
$$u(x,y,0) = X(x)Y(y)T(0) = f(x,y) \hspace{1em} \implies \hspace{1em} T(0) = f(x,y)$$
$$u(0,y,t) = u(l,y,t) = 0 = X(l)Y(y)T(t) = X(0)Y(y)T(t) \hspace{1em} \implies \hspace{1em}  0 = X(l) = X(0)$$
$$u_{y}(x,0,t) = u_{y}(x,l,t) = 0 = X(x)Y(l)T(t) = X(x)Y(0)T(t) \hspace{1em} \implies \hspace{1em}  0 = Y^{\prime}(l) = Y^{\prime}(0)$$

\textbf{(b)}
We solve the $X$ ODE first. By inspection we know that
$$X(x) = A\cos(\mu x) + B\sin(\mu x)$$
and applying the first BC gives
$$X(0) = A = 0$$
The second BC gives
$$X(l) = B\sin(\mu l) = 0 \hspace{1em} \implies \hspace{1em} \mu_{n} = \frac{n\pi}{l}$$
So our first spatial solution is
$$X_{n}(x) = \sin\left(\frac{n\pi x}{l}\right)$$
Next, we solve the $Y$ ODE, where again by inspection
$$Y(y) = A\cos\left({\sqrt{\frac{\lambda}{k} - \mu^{2}}y}\right) + B\sin\left({\sqrt{\frac{\lambda}{k} - \mu^{2}}y}\right)$$
and the BC's give
$$Y^{\prime}(0) = B\sqrt{\frac{\lambda}{k} - \mu^{2}} = 0 \hspace{1em} \implies \hspace{1em} B = 0$$
$$Y^{\prime}(l) = -A\sqrt{\frac{\lambda}{k} - \mu^{2}}\sin\left(\sqrt{\frac{\lambda}{k} - \mu^{2}}l\right) = 0$$
We see that then that
$$\sqrt{\frac{\lambda}{k} - \mu^{2}}l = m\pi \hspace{2em} m\in\mathbb{N}\cup\{0\}$$
$$\lambda_{m,n} = \frac{m^{2}\pi^{2}k}{l^{2}} - \mu^{2}_{n}k$$
Plugging this into our eigenfunction we get
$$Y_{m}(y) = \cos\left(\frac{m\pi y}{l}\right)$$
and hence we have the eignefunctions for our spatial solutions.

\textbf{(c)}
By inspection, we see that the solution is
$$T_{m,n}(t) = Ae^{-\lambda_{m,n}t}$$

\textbf{(d)}
We know that the final solution should be the product of these individual solutions, in particular
$$u(x,y,t) = \sum_{m=0}^{\infty}\sum_{n=1}^{\infty}T_{n,m}(t)Y_{m}(y)X_{n}(x) =\sum_{m=0}^{\infty}\sum_{n=1}^{\infty} Ae^{-\lambda_{m,n}t}\left(\sin\left(\frac{n\pi x}{l}\right)\right)\left(\cos\left(\frac{m\pi y}{l}\right)\right) $$

\textbf{(e)}
We use the temporal initial condition to solve for the Fourier coefficient. In particular
$$u(x,y,0) = f(x,y)= \sum_{m=0}^{\infty}\sum_{n=1}^{\infty} A\left(\sin\left(\frac{n\pi x}{l}\right)\right)\left(\cos\left(\frac{m\pi y}{l}\right)\right) $$
Since we have a double sum, we will need to project the sum over two bases. We see
$$A_{m,n} = \frac{\left(\cos\left(\frac{m\pi y}{l}\right),\left(\sin\left(\frac{n\pi x}{l}\right),f(x,y)\right)\right)}{\left(Y_{m}(y),Y_{m}(y)\right)\left(X_{n}(x),X_{n}(x)\right)}$$

\newpage
\textbf{Problem 2}

\textbf{(a)}
Assume the solution is of the form $u(x,y,t) = T(t)M(x,y)$, then our PDE becomes
$$T^{\prime\prime}M = Tc^{2}(M_{xx} + M_{yy})$$
Apply the separation constant $\lambda$ and we get
$$T^{\prime\prime} + \lambda T = 0 \hspace{1em} \& \hspace{1em} M_{xx} + M_{yy} = -\frac{\lambda}{c^{2}}M$$
Let $M(x,y) = X(x)Y(y)$, then
$$\frac{X^{\prime\prime}}{X} + \frac{Y^{\prime\prime}}{Y} = -\frac{\lambda}{c^{2}}$$
where we use another separation constant $\mu^{2}$ to get
$$X^{\prime\prime}+\mu^{2}X = 0 \hspace{2em} \& \hspace{2em} Y^{\prime\prime} + \left(\frac{\lambda}{c^{2}} - \mu^{2}\right)Y = 0$$
We first solve the spatial problem. Solving first for $X(x)$, we see that
$$X(x) = A\cos(\mu x) + B\sin(\mu x)$$
but our BC says that $X(0) = X(L) = 0$, so
$$X(0) = A = 0 \hspace{2em} \& \hspace{2em} X(L) = B\sin(\mu L) = 0$$
$$\implies \mu_{n} = \frac{n\pi}{L} \hspace{2em} n\in\mathbb{N}\setminus\{0\}$$
Now for $Y(y)$,
$$Y(y) = A\cos\left(\sqrt{\frac{\lambda}{c^{2}}-\mu^{2}}y\right) + B\sin\left(\sqrt{\frac{\lambda}{c^{2}}-\mu^{2}}y\right)$$
and the BC tells us $Y(0) = Y(H) = 0$ which gives us
$$Y(0) = A = 0 \hspace{2em} \& \hspace{2em} Y(H) = B\sin\left(\sqrt{\frac{\lambda}{c^{2}}-\mu^{2}}H\right)$$
$$\implies \sqrt{\frac{\lambda}{c^{2}}-\mu^{2}}H = m\pi \hspace{2em} m\in\mathbb{N}\setminus\{0\}$$
$$\lambda_{m,n} = \frac{m^{2}\pi^{2}c^{2}}{H^{2}} + \mu^{2}c^{2}$$
Finally, our temporal solution will be
$$T(t) = A\cos(\sqrt{\lambda_{m,n}}t) + B\sin(\sqrt{\lambda_{m,n}}t)$$
and so our solution that satisfies the BCs will be
$$u(x,y,t) = \sum_{m=1}^{\infty}\sum_{n=1}^{\infty}T_{m,n}(t)Y_{m}(y)X_{n}(x) = \sum_{m=1}^{\infty}\sum_{n=1}^{\infty}\left( A\cos(\sqrt{\lambda_{m,n}}t) + B\sin(\sqrt{\lambda_{m,n}}t)\right)\left(\sin\left(\frac{m\pi y}{H}\right)\right)\left(\sin\left(\frac{n\pi x}{L}\right)\right)$$
which with our first initial condition gives
$$u(x,y,0) = f(x,y) =  \sum_{m=1}^{\infty}\sum_{n=1}^{\infty}AY_{m}(y)X_{n}(x)$$
$$\implies A_{m,n} = \frac{\left(X_{n}(x),\left(Y_{m}(y),f(x,y)\right)\right)}{(Y_{m}(y),Y_{m}(y))(X_{n}(x),X_{n}(x))}$$
Similarly, the second initial condition gives us
$$u_{t}(x,y,0) = g(x,y) = \sum_{m=1}^{\infty}\sum_{n=1}^{\infty}B\sqrt{\lambda_{m,n}}Y_{m}(y)X_{n}(x)$$
$$\implies B_{m,n} = \frac{\left(X_{n}(x),\left(Y_{m}(y),g(x,y)\right)\right)}{\lambda_{m,n}(Y_{m}(y),Y_{m}(y))(X_{n}(x),X_{n}(x))}$$
and so we have completely solved the PDE.

\textbf{(b)}
Refer to attached plots.

\textbf{(c)}
We have that
$$\sin\left(\frac{\pi y}{H}\right)\sin\left(\frac{2\pi x}{L}\right) + \sin\left(\frac{2\pi y}{H}\right)\sin\left(\frac{\pi x}{L}\right) = 2\sin\left(\frac{\pi y}{H}\right)\sin\left(\frac{\pi x}{L}\right)\cos\left(\frac{\pi x}{L}\right) + 2\sin\left(\frac{\pi x}{L}\right)\sin\left(\frac{\pi y}{H}\right)\cos\left(\frac{\pi y}{H}\right)$$
$$= 2\sin\left(\frac{\pi y}{H}\right)\sin\left(\frac{\pi x}{L}\right)\left(\cos\left(\frac{\pi x}{L}\right) + \cos\left(\frac{\pi y}{H}\right)\right)$$
The nodal lines will be the values of $(x,y)$ over which these functions will vanish
$$\sin\left(\frac{\pi y}{H}\right)\sin\left(\frac{\pi x}{L}\right)\left(\cos\left(\frac{\pi x}{L}\right) + \cos\left(\frac{\pi y}{H}\right)\right) = 0$$
$$\sin\left(\frac{\pi y}{H}\right)\sin\left(\frac{\pi x}{L}\right) = 0 \hspace{2em} \& \hspace{2em} \cos\left(\frac{\pi x}{L}\right) + \cos\left(\frac{\pi y}{H}\right) = 0$$
The sine terms give us $(x,y) = (0,0),(L,0),(0,H)$ and $(L,H)$, all of which lie on the corners. The cosines give us that
$$\cos\left(\frac{\pi x}{L}\right) =- \cos\left(\frac{\pi y}{H}\right)$$
we notice that since both of the cosines only have half a period over the domain, we expect to be able to find a continuous set of solutions to this transcendental equation. Since the entire function itself is smooth (a product of smooth functions) we expect that the roots form a continuous set. Then, since we require that the boundary points be the corners, we end up with lines between the corners.

\newpage
\textbf{Problem 3}

We first recall what the laplacian in polar coordinates is and see that
$$\nabla^{2}u = \frac{1}{r}\frac{\partial}{\partial r}\left(r\frac{\partial u}{\partial r}\right) + \frac{1}{r^{2}}\frac{\partial^{2}u}{\partial \theta^{2}} = 0$$
We approach this as we normally would, so let $u(r,\theta) = R(r)\Theta(\theta)$, and get
$$\frac{\Theta}{r}\left(rR^{\prime\prime} + R^{\prime}\right) + \frac{R}{r^{2}}\Theta^{\prime\prime} = 0 \hspace{2em} \implies \hspace{2em}\frac{r}{R}(rR^{\prime\prime} + R^{\prime})+ \frac{\Theta^{\prime\prime}}{\Theta} = 0$$
$$\frac{r}{R}(rR^{\prime\prime} + R^{\prime})=-\frac{\Theta^{\prime\prime}}{\Theta}$$
Let $\lambda^{2}$ be the separation constant, which gives us
$$\frac{r}{R}(rR^{\prime\prime} + R^{\prime}) = \lambda^{2} \hspace{2em} \& \hspace{2em} -\frac{\Theta^{\prime\prime}}{\Theta}=\lambda^{2}$$
The angular ODE gives us that
$$\Theta(\theta) = A\cos(\lambda \theta) + B\sin(\lambda \theta)$$
It is safe to assume continuity, so $\Theta(0) = \Theta(2\pi)$ and $\Theta^{\prime}(0) = \Theta^{\prime}(2\pi)$.  Then
$$A = A\cos(2\pi \lambda) + B\sin(2\pi \lambda) \hspace{2em} \& \hspace{2em} \lambda B = -A\lambda\sin(\lambda \theta)+B\lambda\cos(\lambda \theta)$$
which solving results in the usual result
$$\Theta(\theta) = A_{n}\cos(n\theta) + B_{n}\sin(n\theta) \hspace{2em} n\in\mathbb{N}\cup \{0\}$$
We look to the radial ODE and notice that the solution must be of the form
$$R(r) = Dr^{n} + Er^{-n}$$
but we require that $u(r,\theta)$ be bounded as $r \to \infty$, so we can assume that $D = 0$, and hence
$$R_{n}(r) = E_{n}r^{-n}$$
and thus our solution becomes
$$u(r,\theta) = \sum_{n=0}^{\infty}\left(A_{n}\cos(n\theta) + B_{n}\sin(n\theta)\right)r^{-n}$$
but we have that
$$u(a,\theta) = f(\theta) = \sum_{n=0}^{\infty}\left(A_{n}\cos(n\theta) + B_{n}\sin(n\theta)\right)a^{-n} $$
$$A_{n} = a^{n}\frac{(\cos(n\theta),f(\theta))}{(\cos(n\theta),\cos(n\theta))} \hspace{2em} \& \hspace{2em} B_{n} =  a^{n}\frac{(\sin(n\theta),f(\theta))}{(\sin(n\theta),\sin(n\theta))}$$
and so our problem is solved.

\newpage
\textbf{Problem 4}

\textbf{(a)}
First we let $u(r,\theta,t) = T(t)M(r,\theta)$. Then,
$$MT^{\prime} = DT\nabla^{2}M - kTM$$
$$\frac{T^{\prime}}{T} = \frac{D\nabla^{2}M}{M} - k$$
$$T^{\prime} + \lambda T = 0 \hspace{2em} \& \hspace{2em} \nabla^{2}M = \frac{k - \lambda}{D}M$$
where $\lambda$ is our separating constant. Now we let $M(r,\theta) = R(r)\Theta(\theta)$, and hence
$$\frac{\Theta}{r}\left(rR^{\prime\prime} + R^{\prime}\right) + \frac{R}{r^{2}}\Theta^{\prime\prime} = \frac{k-\lambda}{D}R\Theta$$
$$\frac{r}{R}\left(rR^{\prime\prime} + R^{\prime}\right) + \frac{\Theta^{\prime\prime}}{\Theta} = \frac{k-\lambda}{D}r^{2}$$
$$\frac{r}{R}\left(rR^{\prime\prime} + R^{\prime}\right) - \frac{k-\lambda}{D}r^{2} = \mu^{2} \hspace{2em} \& \hspace{2em} -\frac{\Theta^{\prime\prime}}{\Theta} = \mu^{2}$$
We recognize the solution to the angular ODE, and further, we recognize this to be exactly the same as in \textbf{Problem 3}, and hence we will find that
$$\Theta_{n}(\theta) = A_{n}\cos(n\theta) + B_{n}\sin(n\theta) \hspace{2em} \mu_{n}=n\in\mathbb{N}\cup\{0\}$$
and for the radial equation we see that
$$r^{2}R^{\prime\prime} + rR^{\prime} + \left(\frac{\lambda - k}{D}r^{2} - n^{2}\right) = 0$$
which we recognize to be the Bessel DE, and hence the solution will be of the form
$$R(r) = D_{n}J_{n}\left(\frac{\lambda - k}{D}r\right) + E_{n}Y_{n}\left(\frac{\lambda-k}{D}r\right)$$
which are the Bessel Functions of the first and second kind. However, we need that $R(l) = 0$, so we can get conditions on $\lambda$ since if the solution is bounded at the centre, $r=0$ we require that $E_{n} = 0$, so
$$R(l) = D_{n}J_{n}\left(\frac{\lambda - k}{D}l\right) = 0$$ 
which will give us $\lambda_{n,m}$. Once we have that, we can find the temporal solution
$$T_{n,m}(t) = Ce^{-\lambda_{n,m}t}$$
and the final solution will the product of all three over the double sum
$$u(r,\theta,t) = \sum_{n=0}^{\infty}\sum_{m=1}^{\infty}T_{n,m}(t)R_{n,m}(r)\Theta_{n}(\theta)$$

\textbf{(b)}
If we let $k=0$ and $D\neq 0$, we can see that our angular solution remains unchanged, and so does our temporal solution (excluding that $\lambda$ is changing). The real change occurs with the Bessel Functions in the Radial equation. We notice that for $k=0$ we get exactly what we would get for the 2D case in a circularly symmetric system under polar coordinates. This naturally will be similar to the 1D case since we have a symmetry in the other degree of freedom.

\newpage
\textbf{Problem 5}

\textbf{(a)}
Assume that $u = M(x,y)N(z)$, then we have that
$$M_{xx}N + M_{yy}N + MN^{\prime\prime} = 0$$
$$\frac{M_{xx} + M_{yy}}{M} + \frac{N^{\prime\prime}}{N} = 0$$
$$M_{xx} + M_{yy} + \lambda^{2}M = 0 \hspace{2em} \& \hspace{2em} N^{\prime\prime} = \lambda^{2}N$$
which are the corresponding ODEs for $M$ and $N$ for the separation constant $\lambda^{2}$.

\textbf{(b)}
Let $M(x,y) = X(x)Y(y)$, then we have that
$$X^{\prime\prime}Y + XY^{\prime\prime} + \lambda^{2}XY = 0$$
$$\frac{X^{\prime\prime}}{X} + \frac{Y^{\prime\prime}}{Y} + \lambda^{2} = 0$$
$$X^{\prime\prime} + (\lambda^{2}-\mu^{2})X = 0 \hspace{2em} \& \hspace{2em} Y^{\prime\prime} + \mu^{2}Y = 0$$
where $\mu^{2}$ is another separation constant. We see that for $Y$ we get
$$Y(y) = A\cos(\mu y) + B\sin(\mu y)$$
and the BC's give us
$$Y(0) = A = 0 \hspace{2em} \to \hspace{2em} Y(H) = B\sin(\mu H) = 0$$
$$\implies \mu_{n} = \frac{n\pi}{H} \hspace{2em} n\in\mathbb{N}\setminus\{0\}$$
For $X$ we get
$$X(x) = D\cos\left(\sqrt{\lambda^{2} - \mu^{2}}x\right) + E\sin\left(\sqrt{\lambda^{2} - \mu^{2}}x\right)$$
which in the BC's gives
$$X(0) = D = 0 \hspace{2em} \to \hspace{2em} X(L) = E\sin\left(\sqrt{\lambda^{2}-\mu^{2}}L\right) = 0$$
$$\implies \sqrt{\lambda^{2}-\mu^{2}}L = m\pi$$
$$\lambda_{n,m} = \sqrt{\frac{m^{2}\pi^{2}}{L^{2}} + \mu^{2}}$$
Finally we can solve the $N$ component
$$N_{n,m}(z) = A_{n,m}e^{\lambda_{n,m}z} + B_{n,m}e^{-\lambda_{n,m}z}$$
and the solution will simply be plugging all of this into our original assumption
$$u(x,y,z) = \sum_{n=1}^{\infty}\sum_{m=1}^{\infty}N_{n,m}(z)X_{m}(x)Y_{n}(y)$$
$$u(x,y,z) = \sum_{n=1}^{\infty}\sum_{m=1}^{\infty}\left(A_{n,m}e^{\lambda_{n,m}z} + B_{n,m}e^{-\lambda_{n,m}z}\right)\sin\left(\frac{m\pi x}{L}\right)\sin\left(\frac{n\pi y}{H}\right)$$
where we now apply the $z$ BC's,
$$u_{y}(x,y,0) = 0 = \sum_{n=1}^{\infty}\sum_{m=1}^{\infty}\left(A_{n,m}\lambda_{n,m}e^{\lambda_{n,m}z} - B_{n,m}\lambda_{n,m}e^{-\lambda_{n,m}z}\right)\sin\left(\frac{m\pi x}{L}\right)\sin\left(\frac{n\pi y}{H}\right)$$
But this is the zero function in an inner product space, so we can assume $A_{n,m} = B_{n,m}$. The second $z$ condition will give us
$$u(x,y,H) = f(x,y) =  \sum_{n=1}^{\infty}\sum_{m=1}^{\infty}A_{n,m}\left(e^{\lambda_{n,m}H} +e^{-\lambda_{n,m}H}\right)\sin\left(\frac{m\pi x}{L}\right)\sin\left(\frac{n\pi y}{H}\right)$$
Taking inner products with the orthogonal basis in both $m$ and $n$ we see
$$A_{n,m} = \frac{\left(\sin\left(\frac{n\pi y}{H}\right),\left(\sin\left(\frac{m\pi y}{L}\right),f(x,y)\right)\right)}{\left(\sin\left(\frac{m\pi y}{L}\right),\sin\left(\frac{m\pi y}{L}\right)\right)\left(\sin\left(\frac{n\pi y}{L}\right),\sin\left(\frac{n\pi y}{L}\right)\right)\left(e^{\lambda_{n,m}H} +e^{-\lambda_{n,m}H}\right)}$$
as required.
\end{document}
