\documentclass[10pt]{article}
\usepackage[]{ragged2e}
\usepackage{fancyhdr,amsmath,amsthm,amssymb,bbm}
\usepackage[utf8]{inputenc}
\usepackage[letterpaper,left=25mm,right=25mm]{geometry}

\setlength{\parskip}{1em}
\setlength{\parindent}{0em}

\newcommand{\Z}{\mathbb{Z}}
\newcommand{\R}{\mathbb{R}}
\newcommand{\Q}{\mathbb{Q}}
\newcommand{\C}{\mathbb{C}}
\newcommand{\N}{\mathbb{N}}
\newcommand{\Sp}{\mathbb{S}}
\newcommand{\Pro}{\mathbb{P}}
\newcommand{\di}[2][]{\frac{\partial #1}{\partial #2}}
\newcommand{\del}[2][]{\frac{d #1}{d #2}}

\DeclareMathOperator{\Ima}{Im}

\linespread{1.25}
\pagestyle{fancy}
\fancyhf{}
\lhead{PMATH 465 $|$  Assignment 6}

\rhead{Dilraj Ghuman $|$ 20564228}

\begin{document}

\textbf{Question 1}

\textbf{(a)}

\textbf{i.} We first need to verify that $D$ is involutive. To see this, consider the Lie Bracket of the two vector fields:
\[ [X,Y] = (y-0)\di{x} + (0-x)\di{y} + (0-0)\di{z} = y\di{x} - x\di{y}. \]
If we suppose $p\in U$ with $p = (x,y,z)$, then we get
\[ [X,Y]_{p} =  y\di{x} - x\di{y}. \]
now, we need to find $a,b\in \R$ such that
\[ [X,Y]_{p} = aX + bY \iff y\di{x} - x\di{y} = bz\di{x} - az\di{y} + (ay - bx)\di{z} \]
\[ \implies az = y \hspace{2em} bz = x \hspace{2em} ay-bx = 0 \]
\[ \implies a = \frac{y}{z} \hspace{2em} b = \frac{x}{z} \]
So, we see that $[X,Y]_{p}\in D_{p}$ $\forall p\in U$, and hence by definition $D$ is involutive over $U$.

\textbf{ii.} Start by finding the integral curves generated by each of these vector fields. First, with $X$ we see that we need to solve the following ODEs,
\[ \del[x]{t} = 0 \hspace{2em} \del[y]{t} = -z \hspace{2em} \del[z]{t} = y \]
we first note that $x(t) = a$ for some $a\in \R$. The other two components are a coupled ODE, which we can solve with a simple substitution:
\[ \frac{d^{2}y}{dt^{2}} = -\del[z]{t} = -y \implies y(t) = A\cos(t) + B\sin(t) \]
\[ \implies z(t) = -\del[y]{t} = A\sin(t) - B\cos(t).\]
Thus, we have that our integral curves are
\[ \gamma_{X}(t) = (a, A\cos(t) + B\sin(t), A\sin(t) - B\cos(t)) \]
where the coeffiecents are determined depending on the point through which these curves pass at $t=0$. For $Y$ we see that we need to solve the ODEs
\[ \del[x]{t} = z \hspace{2em} \del[y]{t} = 0 \hspace{2em} \del[z]{t} = -x \]
which we see gives $y(t) = c$ for some constant $c$. Furthermore
\[ \frac{d^{2}x}{dt^{2}} = \del[z]{t} = -x \implies x(t) = C\cos(t) + D\sin(t) \]
\[ \implies z = \del[x]{t} = -C\sin(t) + D\cos(t) \]
and so we have that the integral curves all take the form of
\[ \gamma_{Y}(t) = (C\cos(t) + D\sin(t),c,-C\sin(t) + D\cos(t)). \]
To see the comparison between the two curves, suppose we considered a point $p = (p_{1},p_{2},p_{3}) \in \R^{3}$ that these curves pass through, both at $t=0$, for simplicity. Then
\[ \gamma_{X}(0) = (p_{1},p_{2},p_{3}) = (a,A,-B) \hspace{2em} \gamma_{Y}(0) = (p_{1},p_{2},p_{3}) = (C,c,D) \]
and our new integral curves become
\[ \gamma_{X}(t) = (p_{1},p_{2}\cos(t) -p_{3}\sin(t),p_{2}\sin(t) + p_{3}\cos(t)) \hspace{2em} \gamma_{Y}(t) =  (p_{1}\cos(t) +p_{3}\sin(t),p_{2},-p_{1}\sin(t) + p_{3}\cos(t)) \]
which are just circles in the $yz$-plane and $xz$-plane respectively. If all such integral curves on our integral submanifold for the distribution $D$ are of this form, then we can conclude that the integral submanifold

% FINISH ME =============================================================================

\textbf{(b)}

\textbf{i.} We approach this by finding an integral curve through the origin such that the tangent to the curve lies in this distribution. Doing this for both $X$ and $Y$ will give us a good idea of what we can get for an integral submanifold.

For $X$, we know we just need to solve the ODE's
\[ \del[x]{t} = 1 \hspace{2em} \del[y]{t} = 0 \hspace{2em} \del[z]{t} = yz \]
from which we immediatly get
\[ x(t) = t + b \hspace{2em} y(t) = c \]
for $b,c\in\R$. Then, we see that
\[ \del[z]{t} = yz = cz \implies z(t) = de^{ct} \]
But, we want our integral curve to go through the origin, so
\[ \gamma(0) = (x(0),y(0),z(0)) = (b,c,d) \implies \gamma(t) = (t,0,0). \]
On the other hand, for $Y$, we see that we need to solve
\[ \del[x]{t} = 0 \hspace{2em} \del[y]{t} = 1 \hspace{2em} \del[z]{t} = 0 \]
which we see gives us
\[ x(t) = a \hspace{2em} y(t) = t + c \hspace{2em} z(t) = d \]
for $a,c,d\in \R$. Then, we need this to also go through the origin, so
\[ \gamma(0) = (x(0),y(0),z(0)) = (a,c,d) \implies \gamma(t) = (0,t,0) \]
So our integral curves from the basis vectors are the basis vectors on the $x-y$ plane in $\R^{3}$. Thus, we can conclude that an integral submanifold of $D$ at the origin is the $x-y$ plane.

\textbf{ii.}

% FINISH ME ====================================================================================

\newpage
\textbf{Question 2}

First we show that $\textbf{(a)}\iff\textbf{(b)}$. To see this, suppose a local chart $(U,\varphi(x_{1},\dots,x_{n})$ of $M$. We need to show that the smoothness of $\omega$ as a map is the same as smoothness of the component functions locally, $\omega^{i}$. We need, then, to consider how the two are related. Notice, we already know that the cotangent bundle is a smooth manifold, so we can use the associated chart on that; set $\tilde{U} = \tilde{\pi}^{-1}(U)\subset T^{*}M$, with 
\[ \tilde{\varphi}: \tilde{U}\subset T^{*}M \to \varphi(U)\times \R^{n} \subset \R^{n}\times\R^{n} \]
where
\[\tilde{\varphi}(p, \omega_{p} = \sum_{k=1}^{n}\omega_{p}^{k}d_{p}x_{k}) = (\varphi(p),(\omega_{p}^{1},\dots,\omega_{p}^{n}))\]
with $p\in U$. Consider some $a\in \varphi(u)$, then
\[ \tilde{\varphi}\circ\omega\circ \varphi^{-1}(a) = \tilde{\varphi}(\omega_{\varphi^{-1}(a)}) \]
and we can unpack this further,
\[ \omega_{\varphi^{-1}(a)} = \sum_{k=1}^{n} \omega_{\varphi^{-1}(a)}^{k}d_{\varphi^{-1}(a)}x_{k} = \sum_{k=1}^{n} \omega^{k}(\varphi^{-1}(a))d_{\varphi^{-1}(a)}x_{k}\]
so we can see that
\[ \tilde{\varphi}(\omega_{\varphi^{-1}(a)}) = (\varphi(\varphi^{-1}(a)), (\omega^{1}\circ \varphi^{-1}(a), \dots, \omega^{n}\circ \varphi^{-1}(a))). \]
With this, we see that smoothness in $\omega$ will hold if and only if the component functions are also smooth $\omega^{i}$ $\forall i \in \{1,\dots,n\}$.

Now, consider some smooth atlas $\{(U_{\alpha},\varphi_{\alpha})\}_{\alpha}$, giving the smooth structure to $M$. Then
\[ \tilde{\varphi_{\alpha}}\circ \omega \circ \varphi_{\beta}^{-1} = \tilde{\varphi_{\alpha}}\circ \tilde{\varphi}^{-1}\circ\tilde{\varphi}\circ\omega \circ\varphi^{-1}\circ\varphi\circ \varphi_{\beta}^{-1}\]
which is still smooth since each part is smoothly compatible.

Now, we show that $\textbf{(b)}\iff\textbf{(c)}$. Suppose first that $\omega(Y)\in C^{\infty}(M)$ for any $Y\in \mathfrak{X}(M)$. Then, if $(U,\varphi = (x_{1},\dots,x_{n})$ is a local chart of $M$, we get
\[ \omega^{i} = \omega\left(\del{x_{i}}\right) \in C^{\infty}(M) \]
since $\del{x_{i}}\in \mathfrak{X}(M)$.

Conversely, we suppose the component functions smooth, in particular $\omega^{i}\in C^{\infty}(U_{\alpha})$, $\forall U_{\alpha}$ in our smooth atlas $\{(U_{\alpha},\varphi_{\alpha})\}$. Suppose $a\in \R^{n}$ and $Y\in \mathfrak{X}(M)$



%\[ \omega(Y)(p) = \omega_{p}(Y_{p}) = \sum_{k=1}^{n}\omega^{k}(p)d_{p}x_{k}(Y) \]
\end{document}
