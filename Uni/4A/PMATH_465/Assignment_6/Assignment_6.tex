\documentclass[10pt]{article}
\usepackage[]{ragged2e}
\usepackage{fancyhdr,amsmath,amsthm,amssymb,bbm}
\usepackage[utf8]{inputenc}
\usepackage[letterpaper,left=25mm,right=25mm]{geometry}

\setlength{\parskip}{1em}
\setlength{\parindent}{0em}

\newcommand{\Z}{\mathbb{Z}}
\newcommand{\R}{\mathbb{R}}
\newcommand{\Q}{\mathbb{Q}}
\newcommand{\C}{\mathbb{C}}
\newcommand{\N}{\mathbb{N}}
\newcommand{\Sp}{\mathbb{S}}
\newcommand{\Pro}{\mathbb{P}}
\newcommand{\di}[2][]{\frac{\partial #1}{\partial #2}}
\newcommand{\del}[2][]{\frac{d #1}{d #2}}

\DeclareMathOperator{\Ima}{Im}

\linespread{1.25}
\pagestyle{fancy}
\fancyhf{}
\lhead{PMATH 465 $|$  Assignment 6}

\rhead{Dilraj Ghuman $|$ 20564228}

\begin{document}

\textbf{Question 1}

\textbf{(a)}

\textbf{i.} We first need to verify that $D$ is involutive. To see this, consider the Lie Bracket of the two vector fields:
\[ [X,Y] = (y-0)\di{x} + (0-x)\di{y} + (0-0)\di{z} = y\di{x} - x\di{y}. \]
If we suppose $p\in U$ with $p = (x,y,z)$, then we get
\[ [X,Y]_{p} =  y\di{x} - x\di{y}. \]
now, we need to find $a,b\in \R$ such that
\[ [X,Y]_{p} = aX + bY \iff y\di{x} - x\di{y} = bz\di{x} - az\di{y} + (ay - bx)\di{z} \]
\[ \implies az = y \hspace{2em} bz = x \hspace{2em} ay-bx = 0 \]
\[ \implies a = \frac{y}{z} \hspace{2em} b = \frac{x}{z} \]
So, we see that $[X,Y]_{p}\in D_{p}$ $\forall p\in U$, and hence by definition $D$ is involutive over $U$.

\textbf{ii.} Start by finding the integral curves generated by each of these vector fields. First, with $X$ we see that we need to solve the following ODEs,
\[ \del[x]{t} = 0 \hspace{2em} \del[y]{t} = -z \hspace{2em} \del[z]{t} = y \]
we first note that $x(t) = a$ for some $a\in \R$. The other two components are a coupled ODE, which we can solve with a simple substitution:
\[ \frac{d^{2}y}{dt^{2}} = -\del[z]{t} = -y \implies y(t) = A\cos(t) + B\sin(t) \]
\[ \implies z(t) = -\del[y]{t} = A\sin(t) - B\cos(t).\]
Thus, we have that our integral curves are
\[ \gamma_{X}(t) = (a, A\cos(t) + B\sin(t), A\sin(t) - B\cos(t)) \]
where the coeffiecents are determined depending on the point through which these curves pass at $t=0$. For $Y$ we see that we need to solve the ODEs
\[ \del[x]{t} = z \hspace{2em} \del[y]{t} = 0 \hspace{2em} \del[z]{t} = -x \]
which we see gives $y(t) = c$ for some constant $c$. Furthermore
\[ \frac{d^{2}x}{dt^{2}} = \del[z]{t} = -x \implies x(t) = C\cos(t) + D\sin(t) \]
\[ \implies z = \del[x]{t} = -C\sin(t) + D\cos(t) \]
and so we have that the integral curves all take the form of
\[ \gamma_{Y}(t) = (C\cos(t) + D\sin(t),c,-C\sin(t) + D\cos(t)). \]
To see the comparison between the two curves, suppose we considered a point $p = (p_{1},p_{2},p_{3}) \in \R^{3}$ that these curves pass through, both at $t=0$, for simplicity. Then
\[ \gamma_{X}(0) = (p_{1},p_{2},p_{3}) = (a,A,-B) \hspace{2em} \gamma_{Y}(0) = (p_{1},p_{2},p_{3}) = (C,c,D) \]
and our new integral curves become
\[ \gamma_{X}(t) = (p_{1},p_{2}\cos(t) -p_{3}\sin(t),p_{2}\sin(t) + p_{3}\cos(t)) \hspace{2em} \gamma_{Y}(t) =  (p_{1}\cos(t) +p_{3}\sin(t),p_{2},-p_{1}\sin(t) + p_{3}\cos(t)) \]
which are just circles in the $yz$-plane and $xz$-plane respectively. If all such integral curves on our integral submanifold for the distribution $D$ are of this form, then we can conclude that the integral submanifold

% FINISH ME =============================================================================

\textbf{(b)}

\textbf{i.} We approach this by finding an integral curve through the origin such that the tangent to the curve lies in this distribution. Doing this for both $X$ and $Y$ will give us a good idea of what we can get for an integral submanifold.

For $X$, we know we just need to solve the ODE's
\[ \del[x]{t} = 1 \hspace{2em} \del[y]{t} = 0 \hspace{2em} \del[z]{t} = yz \]
from which we immediatly get
\[ x(t) = t + b \hspace{2em} y(t) = c \]
for $b,c\in\R$. Then, we see that
\[ \del[z]{t} = yz = cz \implies z(t) = de^{ct} \]
But, we want our integral curve to go through the origin, so
\[ \gamma(0) = (x(0),y(0),z(0)) = (b,c,d) \implies \gamma(t) = (t,0,0). \]
On the other hand, for $Y$, we see that we need to solve
\[ \del[x]{t} = 0 \hspace{2em} \del[y]{t} = 1 \hspace{2em} \del[z]{t} = 0 \]
which we see gives us
\[ x(t) = a \hspace{2em} y(t) = t + c \hspace{2em} z(t) = d \]
for $a,c,d\in \R$. Then, we need this to also go through the origin, so
\[ \gamma(0) = (x(0),y(0),z(0)) = (a,c,d) \implies \gamma(t) = (0,t,0) \]
So our integral curves from the basis vectors are the basis vectors on the $x-y$ plane in $\R^{3}$. Thus, we can conclude that an integral submanifold of $D$ at the origin is the $x-y$ plane.

\textbf{ii.}

% FINISH ME ====================================================================================

\newpage
\textbf{Question 2}

First we show that $\textbf{(a)}\iff\textbf{(b)}$. To see this, suppose a local chart $(U,\varphi(x_{1},\dots,x_{n})$ of $M$. We need to show that the smoothness of $\omega$ as a map is the same as smoothness of the component functions locally, $\omega^{i}$. We need, then, to consider how the two are related. Notice, we already know that the cotangent bundle is a smooth manifold, so we can use the associated chart on that; set $\tilde{U} = \tilde{\pi}^{-1}(U)\subset T^{*}M$, with 
\[ \tilde{\varphi}: \tilde{U}\subset T^{*}M \to \varphi(U)\times \R^{n} \subset \R^{n}\times\R^{n} \]
where
\[\tilde{\varphi}(p, \omega_{p} = \sum_{k=1}^{n}\omega_{p}^{k}d_{p}x_{k}) = (\varphi(p),(\omega_{p}^{1},\dots,\omega_{p}^{n}))\]
with $p\in U$. Consider some $a\in \varphi(u)$, then
\[ \tilde{\varphi}\circ\omega\circ \varphi^{-1}(a) = \tilde{\varphi}(\omega_{\varphi^{-1}(a)}) \]
and we can unpack this further,
\[ \omega_{\varphi^{-1}(a)} = \sum_{k=1}^{n} \omega_{\varphi^{-1}(a)}^{k}d_{\varphi^{-1}(a)}x_{k} = \sum_{k=1}^{n} \omega^{k}(\varphi^{-1}(a))d_{\varphi^{-1}(a)}x_{k}\]
so we can see that
\[ \tilde{\varphi}(\omega_{\varphi^{-1}(a)}) = (\varphi(\varphi^{-1}(a)), (\omega^{1}\circ \varphi^{-1}(a), \dots, \omega^{n}\circ \varphi^{-1}(a))). \]
With this, we see that smoothness in $\omega$ will hold if and only if the component functions are also smooth $\omega^{i}$ $\forall i \in \{1,\dots,n\}$.

Now, consider some smooth atlas $\{(U_{\alpha},\varphi_{\alpha})\}_{\alpha}$, giving the smooth structure to $M$. Then
\[ \tilde{\varphi_{\alpha}}\circ \omega \circ \varphi_{\beta}^{-1} = \tilde{\varphi_{\alpha}}\circ \tilde{\varphi}^{-1}\circ\tilde{\varphi}\circ\omega \circ\varphi^{-1}\circ\varphi\circ \varphi_{\beta}^{-1}\]
which is still smooth since each part is smoothly compatible.

Now, we show that $\textbf{(b)}\iff\textbf{(c)}$. Suppose first that $\omega(Y)\in C^{\infty}(M)$ for any $Y\in \mathfrak{X}(M)$. Then, if $(U,\varphi = (x_{1},\dots,x_{n})$ is a local chart of $M$, we get
\[ \omega^{i} = \omega\left(\del{x_{i}}\right) \in C^{\infty}(M) \]
since $\del{x_{i}}\in \mathfrak{X}(M)$.

Conversely, we suppose the component functions smooth, in particular $\omega^{i}\in C^{\infty}(U_{\alpha})$, $\forall U_{\alpha}$ in our smooth atlas $\{(U_{\alpha},\varphi_{\alpha})\}$. Suppose $a\in \R^{n}$ and $Y\in \mathfrak{X}(M)$, then
\[ \omega(Y) \circ \varphi^{-1}(a) : \varphi(U)\subset \R \to \R \]
\[ \omega(Y) \circ \varphi^{-1}(a) = \omega_{\varphi^{-1}(a)}(Y_{\varphi^{-1}(a)}) = \sum_{i=1}^{\infty}\omega^{i}(\varphi^{-1}(a))d_{\varphi^{-1}(a)}x_{i}(Y).\]
By supposition, we already know that $\omega^{i}$ is smooth, so all we need to show being smooth is $d_{\varphi^{-1}(a)}x_{i}(Y)$ $\forall i\in \{1,\dots,n\}$. We see that
\[ d_{\varphi^{-1}(a)}x_{i}(Y) = Y_{\varphi^{-1}(a)}(x_{i})\]
but $Y \in \mathfrak{X}(M)$, so we know that $Y$ is a smooth vector field, and hence, $Y^{i} = Y(x_{i})$ is a smooth function, and thus $d_{\varphi^{-1}(a)}x_{i}(Y)$ is a smooth function. Thus, we can conclude that $\omega(Y)\in C^{\infty}(M)$ for any $Y\in \mathfrak{X}(M)$.

\newpage
\textbf{Question 3}

Suppose $M,N$ smooth manifolds and $F:M\to N$ a smooth map. Further, we will need , $f\in C^{\infty}(N)$ and $\omega \in \Lambda^{1}(N)$.

\textbf{(a)} Let $X \in T_{p}M$ for some $p\in M$, $\omega,\eta \in \Lambda^{1}(N)$ and $a,b\in \R$. Then, we see that
\[ F^{*}(a\omega + b\eta)(X) = (a\omega+b\eta)(F_{*,p}(X)) = a\omega(F_{*,p}(X)) + b\eta(F_{*,p}(X)) = aF^{*}(\omega)(X) + bF^{*}(\eta)(X) \]
and hence we have $\R$-linearity in the pullback.

\textbf{(b)} Let $X \in T_{p}M$ for some $p\in M$ and $f\in C^{\infty}(N)$. Then
\[ F^{*}(df)(X) = df(F_{*,p}(X)) = d_{F(p)}f(F_{*,p}(X)) = F_{*,p}(X)(f) = X_{p}(f\circ F) = d_{F(p)}(f\circ f)(X) = d(f\circ F)(X) \]
as required.

\textbf{(c)} Let $X \in T_{p}M$ for some $p\in M$, $f\in C^{\infty}(N)$ and $\omega \in \Lambda^{1}(N)$. Then,
\[ F^{*}(f\omega)(X) = (F_{*,p})^{*}(f(F(p))\omega_{F(p)})(X) = f(F(p))(F_{*,p})^{*}(\omega_{F(p)})(X) = (f\circ F)F^{*}(\omega)(X). \]

\textbf{(d)} Let $X \in T_{p}M$ for some $p\in M$, $(U,\varphi = (y_{1},\dots,y_{n})$ local chart of $N$, $\omega\in \Lambda^{1}(N)$ where $\omega = \sum_{i=1}^{n}\omega^{i}dy_{i}$ and $F = (F_{1},\dots,F_{n})$. Then,
\[ F^{*}(\omega)(X) = (F_{*,p})^{*}(\omega_{F(p)})(X) = \omega_{F(p)}(F_{*,p}(X)) = \sum_{i=1}^{n}\omega_{F(p)}^{i}d_{F(p)}(y_{i})(F_{*,p}(X)). \]
Notice that
\[ d_{F(p)}(y_{i})(F_{*,p}(X)) = (F_{*,p}(X))_{F(p)}(y_{i}) = X_{p}(y_{i}\circ F) = X_{p}(F_{i}) = d_{F(p)}F_{i}(X)\]
which then gives us that
\[ F^{*}(\omega)(X) = \sum_{i=1}^{n}\omega_{F(p)}^{i}d_{F(p)}(y_{i})(F_{*,p}(X)) = \sum_{i=1}^{n}(\omega^{i}\circ F)(p)d_{p}F_{i}(X)\]
as expected.

\textbf{(e)} We use the previous result. Supposing the same conditions as \textbf{(d)}, we see that we know
\[ F^{*}(\omega) = \sum_{i=1}^{n}(\omega^{i}\circ F)dF_{i} \]
but we also know that $\omega$ is smooth, and thus $\omega^{i}$ is smooth and $(\omega^{i}\circ F)\in C^{\infty}(M)$ since $F$ is also smooth. Moreover, since $y_{i}$ is smooth as well, we get that $dF_{i}\in \Omega^{1}(M)$. Thus, since the right hand side of the equations is just a sum of smooth 1-forms on $M$, we get that
\[ F^{*}(\omega) = \sum_{i=1}^{n}(\omega^{i}\circ F)dF_{i} \in \Omega^{1}(M) \]
as expected.

\newpage
\textbf{Question 4}

We already know that $H$, $N$, and $G$ must all be groups, so now we need to show that $G$ is a smooth manifold. To see how this quoteint group forms a smooth manifold, we need to see how the quotient group is constructed. We recall that the quotient group by a normal subgroup, which $N$ is, is defined as the set of left-cosets, which is equivalent to the set of right cosets. Then,
\[ G = H/N = \{hN: h\in H\} \]
and we see that if $n\in N$ and $h\in H$
\[ h =
\begin{bmatrix}
  1 & a & b \\
  0 & 1 & c \\
  0 & 0 & 1 \\
\end{bmatrix}
\hspace{2em}
n =
\begin{bmatrix}
  1 & 0 & m \\
  0 & 1 & 0 \\
  0 & 0 & 1 \\
\end{bmatrix}
\]
where $a,b,c,m \in \R$. Then
\[ hn =
\begin{bmatrix}
  1 & a & b+m \\
  0 & 1 & c \\
  0 & 0 & 1 \\
\end{bmatrix}
.\]
Then, we see that the elements of $H/N$ are cosets, and hence is the set of matrices $\forall m \in \R$. Furthermore, we see that $[h] = hN$, and $h \sim h^{\prime}$, where
\[ h =
\begin{bmatrix}
  1 & a & b \\
  0 & 1 & c \\
  0 & 0 & 1 \\
\end{bmatrix}
\hspace{2em}
h^{\prime} =
\begin{bmatrix}
  1 & a^{\prime} & b^{\prime} \\
  0 & 1 & c^{\prime} \\
  0 & 0 & 1 \\
\end{bmatrix}
\iff a = a^{\prime} \hspace{1em} \& \hspace{1em} c = c^{\prime}.
\]
Finally, this shows us that we can associate each element of the group with an element of $\R^{2}$, in particular $h\mapsto (a,c)\in \R^{2}$ as a bijection! The bijection follows quite naturally, since if $[h]\in G$, then it can be defined uniquely by $a,c \in \R$, and thus uniquely by $(a,c)\in \R^{2}$. For surjection, we see that any $(a,c)\in \R^{2}$ has a class defined by any $h$, as long as the components match accordingly, and then the preimage becomes $hN = [h]$.

Even more natural is the product of two elements. If we call the association map $f: G \to \R^{2}$, where $f(h) = (a,c)$ for $h\in G$, then we can even see that the operation of multiplication is sent nicely, since if we take the $h,h^{\prime}$ from earlier, we get
\[ hh^{\prime} =
\begin{bmatrix}
  1 & a^{\prime}+ a & b^{\prime}+c^{\prime}a + b \\
  0 & 1 & c^{\prime} + c \\
  0 & 0 & 1 \\
\end{bmatrix}
\]
and so $f(hh^{\prime}) = (a^{\prime} + a,c^{\prime} + c)$. Notice this works for the group $G$ since we take $h$ and $h^{\prime}$ as representatives of their respective cosets.

We also have that $f$ is smooth with smooth inverese, since it is just a polynomial map, and a very simply polynomial map at that. Thus, $f$ has given us a way of associating $G$ with $\R^{2}$ smoothly, and hence we can use the natural topology of $\R^{2}$ with the standard atlas to get a smooth structure on $G$. Therefore, $G$ is a smooth manifold.

To completely show that $G$ is a Lie group, we need that the multiplication map is also smooth and so is the inverse map. Of course, this follow quite easily from our association with $\R^{2}$, since multiplication is just addition of vectors in $\R^{2}$, we can use the smoothness of addition to guarantee that the multiplication map is smooth. The inverse map is also quite straightforward, as it better agree with multiplication being addition of vectors in $\R^{2}$, and hence is just the map that flips the signs of the components of the vector in $\R^{2}$, and hence is also smooth.

We can conclude that $G$ is a Lie Group.

The Lie Algebra structure is isomorphic to the tangent plane of the Lie group at the identity. In particular, our Lie algebra will be $T_{[e]}G$, where $e$ is the identity matrix in $G$. Notice that $e$ is just a representative of the identy element of $G$, as the identity is the coset $eN = N$. The correspining element in $\R^{2}$ is, as we expect, $(0,0)$, so we know that the tangent space will also correspond, thus $T_{[e]}G \simeq T_{(0,0)}\R^{2} \simeq \R^{2}$, where '$\simeq$' is vector space isomorphism. Hence, the Lie algebra of this Lie group is isomorphic as a vector space to $\R^{2}$.

\newpage
\textbf{Question 5}

\textbf{(a)} Let $X\in \mathfrak{X}(M)$, $(x_{1},\dots,x_{n})$ be local coordinates and let $\gamma:I\to M$ be a smooth curve.

Assume that $X$ is parallel along $\gamma(t)$, then we know that if $X = \sum_{i=1}^{n}X^{i}\di{x_{i}}$,
\[ D_{\gamma^{\prime}(t)}X = \sum_{i=1}^{n}\gamma^{\prime}(t)(X^{i})\di{x_{i}} = 0 \]
but we notice that we must have
\[ \gamma^{\prime}(t)(X^{i}) = \gamma_{*,\gamma(t)}\left(\del{t}\right)(X^{i}) = \del{t}(X^{i}\circ \gamma(t)) = 0 \]
$\forall i \in \{1,\dots,n\}$, which implies $X^{i}\circ\gamma(t) = a^{i}$ for some $a^{i}\in \R$. Thus, we see that
\[  X_{\gamma(t)} = \sum_{i=1}^{n}X^{i}(\gamma(t))\di{x_{i}}\bigg|_{\gamma(t)} = \sum_{i=1}^{n}X^{i}_{\gamma(t)}\di{x_{i}}\bigg|_{\gamma(t)} = \sum_{i=1}^{n}a^{i}\di{x_{i}}\bigg|_{\gamma(t)}\]
and $X$ is constant along $\gamma(t)$.

Conversely, suppose that we have $a_{1},\dots,a_{n}\in \R$ such that
\[ X_{\gamma(t)} = \sum_{i=1}^{n}a_{i}\di{x_{i}}\bigg|_{\gamma(t)} \]
then we see that
\[ D_{\gamma^{\prime}(t)}X = \sum_{i=1}^{n}\gamma^{\prime}(t)(X^{i})\di{x_{i}}\bigg|_{\gamma(t)} = \sum_{i=1}^{n}\gamma_{*,\gamma(t)}(\del{t})(X^{i})\di{x_{i}}\bigg|_{\gamma(t)} \]
\[= \sum_{i=1}^{n}\del{t}(X^{i}(\gamma(t)))\di{x_{i}}\bigg|_{\gamma(t)} = \sum_{i=1}^{n}\del{t}(a_{i})\di{x_{i}}\bigg|_{\gamma(t)} = 0\]
and so we have that $X$ is parallel along $\gamma(t)$.

From what we see above, if we replace $X$ with $\gamma^{\prime}(t)$, we see that if $\gamma(t)$ were a geodesic,
then
\[ D_{\gamma^{\prime}(t)}\gamma^{\prime}(t) = 0 \iff \gamma^{\prime}(t)_{\gamma(t)} = \sum_{i=1}^{n}a_{i}\del{x_{i}}\bigg|_{\gamma(t)} \]
and so we see that $\gamma^{\prime}(t)$ are constant tangents, which implies that $\gamma(t)$ is linear, and hence $\gamma(t)$ is linear and corresponds with a line (segment) in $\R^{n}$

\textbf{(b)}

\textbf{(c)} We first see that $\R$-linearity in each component is satisfied immediatly since the standard inner product is bilinear, and we have already shown that the Euclidean Connection is bilinear. Suppose $f\in C^{\infty}(S^{2})$, $p\in S^{2}$ and $X,Y\in \mathfrak{X}(S^{2})$, then
\[ \nabla_{fX}Y(p) = D_{fX}Y(p) - (D_{fX}Y(p) \cdot p)p = fD_{fX}Y(p) - f(D_{X}Y(p) \cdot p)p = f\nabla_{X}Y(p) \]
and hence we have $C^{\infty}(S^{2})$-linearity in the first component. Now we check the Leibniz Rule,
\[ \nabla_{X}(fY)(p) = D_{X}(fY)(p) - (D_{X}(fY)(p) \cdot p)p\]
\[ = (X(f)Y + fD_{X}Y)(p) - (((X(f)Y + fD_{X}Y))(p) \cdot p)(p)\]
\[ = X(f)Y(p) + fD_{X}Y(p) - (X(f)Y(p)\cdot p)(p) + (fD_{X}Y(p) \cdot p)(p)\]
\[= X(f)Y(p) + f(D_{X}Y(p) - (D_{X}(Y)(p) \cdot p)p) = X(f)Y(p) + f\nabla_{X}(Y) \]
as expected. So, we see that this is indeed an affine connection.
\end{document}
