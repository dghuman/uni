\documentclass[10pt]{article}
\usepackage[]{ragged2e}
\usepackage{fancyhdr,amsmath,amsthm,amssymb,bbm}
\usepackage[utf8]{inputenc}
\usepackage[letterpaper,left=25mm,right=25mm]{geometry}

\setlength{\parskip}{1em}
\setlength{\parindent}{0em}

\newcommand{\Z}{\mathbb{Z}}
\newcommand{\R}{\mathbb{R}}
\newcommand{\Q}{\mathbb{Q}}
\newcommand{\C}{\mathbb{C}}
\newcommand{\Sp}{\mathbb{S}}
\newcommand{\Pro}{\mathbb{P}}

\DeclareMathOperator{\Ima}{Im}

\linespread{1.25}
\pagestyle{fancy}
\fancyhf{}
\lhead{PMATH 465 $|$  Assignment 2}

\rhead{Dilraj Ghuman $|$ 20564228}

\begin{document}

\textbf{Question 1}

Let $X = \{(x,y)\in\R^{2} | y=\pm 1\}$, and we define the equivalence relation $\sim$ where $(x,1) \sim (x,-1)$ for $x \neq 0$. Then $Y = X/\sim$ is the set of equivalence classes of $x$ values and two zeros, $(0,-1)$ and $(0,+1)$. 

We first see that $X$ is a topological space by the relative topology. To see that $X$ is also Hausdorff, consider $(x_{1},y_{1}), (x_{2},y_{2}) \in X$ distinct. Then, suppose $U_{1}, U_{2} \subset X$ as open sets. By the relative topology of $\R^{2}$, we can always choose open balls in $\R^{2}$ small enough so that the intersection of $X$ with that open set is not disjoint. In particular, this means that the open set will be an open interval on either the line $y=1$ or $y=-1$.

Then, if $y_{1} \neq y_{2}$, we have that $U_{1}\cap U_{2} = \emptyset$ by construction. Suppose that $y_{1} = y_{2}$, then we need to consider only the distinct points $x_{1}$ and $x_{2}$ on $\R$, but since the relative topology here will just be the standard topology of $\R$, we get Hausdorff for free. So $X$ is Hausdorff for sure.

Now consider $Y$. In particular, to see that $Y$ is not Hausdorff, we choose the distinct points $(0,-1)$ and $(0,+1)$. We note that open sets are defined in $Y$ by the quotient topology, and we use open sets in $X$ under the equivalence relation to get our topology, $\bar{\tau}$.

In particular, suppose again we can pick open balls small enough in $\R^{2}$ around the two points so that the intersection of the open ball with $X$ is restricted to a single line. Call these two open sets $U_{1} = B_{r}((0,1))\cap X$ and $U_{2} = B_{r}((0,-1))\cap X$, where $r < 2$. Suppose that $p_{1} \in U_{1}$, then $\exists x \in \R\setminus \{0\}$ such that $p_{1} = (x,1) \sim (x,-1) \in U_{2}$. So, in $Y$, the open sets will never be disjoint around $(0,1)$ and $(0,-1)$.

Thus $Y$ is not Hausdorff.

\newpage

\textbf{Question 2}

\newpage

\textbf{Question 3}

\textbf{(a)} Suppose $x\in \Sp^{n}\setminus \{N\}$, then we can parameterize the line connecting $x$ and $N$ by the following equation,
$$y = t(x-N)+ N \hspace{1em} t\in \R$$
In particular, we can find the $t$ for which this line intersects the subspace defined by $x_{n+1} = 0 \subset \R^{n+1}$,
$$(y_{1}, \dots, y_{n}, 0) = tx + N(1-t) = (tx_{1}, \dots, tx_{n}, tx_{n+1} + 1 - t)$$
$$ \implies t(x_{n+1} - 1) + 1 = 0 \iff t = \frac{1}{1-x_{n+1}}$$
Hence, we see that the intersection of the line connecting $N$ and $x$ with the subspace setting $x_{n+!} = 0$ is just,
$$\frac{(x_{1},\dots,x_{n},0)}{1-x_{n+1}} = (u,0) = (\sigma(x),0) $$
as expected. We do the same for $S$, and we see that
$$y = t(x-S) + S$$
and again, for the subspace defined by $x_{n+1}= 0$, we get,
$$ (y_{1}, \dots, y_{n}, 0) = (tx_{1},\dots,tx_{n},tx_{n+1} -1 +t)$$
$$ \implies tx_{n+1} -1 +t = 0 \iff t = \frac{1}{1 + x_{n+1}}$$
Hence, the intersection point will be,
$$\frac{(x_{1},\dots,x_{n},0)}{1+x_{n+1}} = -\frac{(-x_{1},\dots,-x_{n},0)}{1-(-x_{n+1})} = (-\sigma(-x),0) = (\tilde{\sigma}(x),0)$$

again as we would expect.

\textbf{(b)} First we show injection. To see this, suppose that $x,y \in \Sp^{n}\setminus \{N\}$. Then, as we showed in $\textbf{(a)}$, we can relate the image under $\sigma$ of these points to the intersection of the line connecting the point and $N$ with the subspace that sets the $n^{th}+1$ component to 0. That is to say, if $\sigma(x) = \sigma(y)$, then $(\sigma(x),0) = (\sigma(y),0)$, which is to say that the two lines would intersect at that point in the plane defined by setting the $n^{th}+1$ component to 0. Yet, these two lines necessarily intersect at $N$ aswell, so they must be the same line.

Then, we have that this line intersects $\Sp^{n}$ at $N$, by necessity, and both $x$ and $y$. But this is impossible, since a straight line in $R^{n+1}$ will only intersect a sphere at most twice. Further, by supposition, $x \neq N$ and $y \neq N$, and hence it must be that $x=y$. Thus we have that $\sigma$ is injective.

Now we check surjection. The image lies in $\R^{n}$, so suppose $u\in\R^{n}$. In particular, we can again use the fact that $\sigma$ is simply the intersection of the line connecting $N$ and a point on $\Sp^{n}$ with the plane defined by setting the last component to 0. In particular, we then associate $u$ with $(u,0)$. Then, the line connecting this point and $N$ is defined by,
$$y = t((u,0) - N) + N = (tu,0) + N(1-t) = (tu,1-t)$$
for $t\in\R$. To find the point $y\in \Sp^{n}\setminus \{N\}$ with which this line intersects, we use the definition of $\Sp^{n}$
$$1 = \sum_{i=1}^{n+1}y_{i}^{2} = \sum_{i=1}^{n}(tu_{i}) + (1-t)^{2}$$
$$1 - (1 -2t +t^{2}) = t^{2}|u|^{2}$$
$$2 - t = t|u|^{2} \iff t = \frac{2}{|u|^{2} +1}$$
Thus, we have that
$$y = \left(\frac{2u}{|u|^{2} + 1},1 - \frac{2}{|u|^{2} + 1}\right) = \frac{(2u_{i},\dots,2u_{n},|u|^{2}-1)}{|u|^{2} + 1}$$
Hence we have that every $u\in\R^{n}$ has a preimage on $\Sp^{n}\setminus \{N\}$, and thus $\sigma$ is surjective.

We can now conclude that $\sigma$ is a bijection, with the inverse stated above.

\textbf{(c)} We note that $\tilde{\sigma} \circ \sigma^{-1}: \R^{n} \to \R^{n}$, since both stereographic projections are bijective. Then, suppose that $u\in\R^{n}$, and from the inverse computed in \textbf{(b)} we see
$$\tilde{\sigma}(\sigma^{-1}(u)) = \tilde{\sigma}\left(\frac{(2u_{i},\dots,2u_{n},|u|^{2}-1)}{|u|^{2} + 1}\right)$$
and by definition, we recall that $\tilde{\sigma}(x) = -\sigma(-x)$, thus,
$$\tilde{\sigma}(\sigma^{-1}(u)) = -\sigma\left(-\frac{(2u_{i},\dots,2u_{n},|u|^{2}-1)}{|u|^{2} + 1}\right) = -\sigma\left(\frac{(-2u_{i},\dots,-2u_{n},-|u|^{2}+1)}{|u|^{2} + 1}\right)$$
and applying the definition of $\sigma$ we see,
$$\tilde{\sigma}(\sigma^{-1}(u)) = -\sigma\left(\frac{(-2u_{i},\dots,-2u_{n},-|u|^{2}+1)}{|u|^{2} + 1}\right) = -\frac{(-2u_{i},\dots,-2u_{n})}{(|u|^{2} + 1)(1 + |u|^{2} - 1)} = \frac{(2u_{i},\dots,2u_{n})}{(|u|^{2} + 1)|u|^{2}}$$
Which is smooth except at the origin, which makes sense considering the stereographic projections get the origin from opposite poles. Further, we recall that both $\sigma$ and $\tilde{\sigma}$ are invertible, and inverse will thus also be smooth. Then, we have a diffeomorphism and hence the two charts are smoothly compatible, and hence the atlas $\{(\sigma,\Sp^{n}\setminus \{N\}),(\tilde{\sigma},\Sp^{n}\setminus \{S\})\}$ is a smooth atlas and gives a smooth structure on $\Sp^{n}$.

\textbf{(d)} To see how this smooth structure is the same as the standard smooth structure on $\Sp^{n}$ we consider the charts $(U_{i}^{\pm},\varphi_{i}^{\pm})$ and note that these are exactly our charts when $i = n+1$. Further, since every smooth atlas has a maximal atlas, we can see that the standard smooth structure then must contain the atlas we made above. Hence we have the same smooth structure on the two manifolds.

\newpage

\textbf{Question 4}

\textbf{(a)} Denote $\pi_{1} : \R^{n+1} \to \Pro^{n}$ and $\pi_{2} : \Sp^{n} \to \Sp^{n}/\sim$ the respective projection maps. We know that these two maps are homeomorphisms, so we know to expect their composition to also be a homeomorphism. In particular, consider the map
$$f = \pi_{1} \circ\pi_{2}^{-1}: \Sp^{n}/\sim \to \pi_{1}(\Sp^{n})\subset \Pro^{n}$$
which is a homeomorphism due to $\pi_{1}$ and $\pi_{2}$ being homeomorphisms, and further is a homeomorphism from $\Sp^{n}/\sim$ to a subset of $\Pro^{n}$.

Now all we need to do is convince ourselves that this subset is actually all of $\Pro^{n}$. To see this, suppose that $p\in \Pro^{n}$, then $\pi_{1}^{-1}(p) \subset \R^{n+1}$ is a real line through the origin such that all those points are sent to $p$ under $\pi_{1}$. But the intersection of this line with $S^{n}$ will correspond with two antipodal points. In particular, $\pi_{2}\left(\pi_{1}^{-1}(p)\cap S^{n}\right)\subset \Sp^{n}/\sim$, and hence every point in $\Pro^{n}$ has a preimage in $\Sp^{n}/\sim$.

\textbf{(b)} To see that $\Pro^{n}$ is compact and connected, we use the fact that it is homeomorphic to $\Sp^{n}/\sim$, in particular we know that $\Sp^{n}/\sim$ is the quotient of the compact and connected set $\Sp^{n}\subset \R^{n+1}$. Then, $\Sp^{n}/\sim$ must also be compact and connected, and hence $\Pro^{n}$ is compact and connected since it is homeomorphic to a compact and connected set.

\textbf{(c)} We will start with the sets and corresponding maps $\{U_{i}, \varphi_{i}\}_{i}$.

First, we note that $\varphi_{i}$ is injective by construction. Suppose $[x],[y]\in U_{i}\subset\Pro^{n}$ and $\varphi_{i}([x]) = \varphi_{i}([y])$, but then by definition of $\varphi_{i}$ we must have that $x_{j} = y_{j}$ $\forall j= 1,\dots,n+1$ and hence $\varphi_{i}$ is injective.

\textbf{(i)}
We need that $\varphi_{i}(U_{i})$ is open in $\R^{n}$. To see this, we recall the projection map $\pi: \R^{n+1}\setminus\{0\} \to \Pro^{n}$ and denote the trivially continuous map $\iota: \R^{n} \to \R^{n+1}\setminus\{0\}$ where $\iota(x_{1},\dots,x_{i-1},x_{i+1},\dots,x_{n+1}) = (x_{1},\dots,x_{i-1},1,x_{i+1},\dots,x_{n+1})$. Then, we notice that $\varphi^{-1} = \pi \circ \iota$, and since both $\iota$ and $\pi$ are continuous, so is $\varphi^{-1}$. Finally, we know that $U_{i}$ is open in $\Pro^{n}$ and hence $\varphi_{i}(U_{i})$ is open.

\textbf{(ii)}
Fix $\alpha,\beta\in\{1,\dots,n+1\}$, then consider $\varphi_{\alpha}(U_{\alpha}\cap U_{\beta})$ and $\varphi_{\beta}(U_{\alpha}\cap U_{\beta})$: we wish to show these are open in $\R^{n}$. To see this, we only need that $U_{\alpha}\cap U_{\beta}$ is open in $\Pro^{n}$, but this follows from the quotient topology of $\Pro^{n}$. Then, since $U_{\alpha}\cap U_{\beta} \subset U_{\alpha}$, and we already showed prior that $\varphi_{i}^{-1}$ is continuous, we get that $\varphi_{\alpha}(U_{\alpha}\cap U_{\beta})$ is open. A similar argument with $\alpha$ swapped for $\beta$ shows the same for $\varphi_{\beta}(U_{\alpha}\cap U_{\beta})$.

\newpage

\textbf{(iii)}
Suppose that $U_{\alpha}\cap U_{\beta} \neq \emptyset$. We want
$$\varphi_{\alpha}\circ\varphi_{\beta}^{-1}: \varphi_{\beta}(U_{\alpha}\cap U_{\beta}) \subset \R^{n} \to \varphi_{\alpha}(U_{\alpha}\cap U_{\beta})\subset \R^{n}$$
to be a diffeomorphism. To see this, suppose that $p\in U_{\alpha}\cap U_{\beta}$, then if $p = [x_{1}, \dots, x_{n+1}]$, we know that $\varphi_{\beta}(p) = \left(\frac{x_{1}}{x_{\beta}},\dots,\frac{x_{\beta -1}}{x_{\beta}},\frac{x_{\beta+1}}{x_{\beta}},\dots,\frac{x_{n+1}}{x_{\beta}}\right)\in \R^{n}$. Applying our map to this, we see,
$$\varphi_{\alpha}(\varphi_{\beta}^{-1}(\varphi_{\beta}(p))) = \varphi_{\alpha}\left(\varphi_{\beta}^{-1}\left(\left(\frac{x_{1}}{x_{\beta}},\dots,\frac{x_{\beta -1}}{x_{\beta}},\frac{x_{\beta+1}}{x_{\beta}},\dots,\frac{x_{n+1}}{x_{\beta}}\right)\right)\right)$$
$$ = \left(\frac{x_{1}}{x_{\alpha}},\dots,\frac{x_{\alpha -1}}{x_{\alpha}},\frac{x_{\alpha+1}}{x_{\alpha}},\dots,\frac{x_{n+1}}{x_{\alpha}}\right)$$
and since necessarily we have $x_{\alpha} \neq 0$ and $x_{\beta} \neq 0$, we have that this is indeed a diffeomorphism. Indeed the inverse would carry the same property of being smooth, and the bijective properties come from each $\varphi_{i}$ being a homeomorphism.

\textbf{(iv)}
We need next that $U_{i}$ covers $\Pro^{n}$ for countably many $U_{i}$. This follows directly from the construction of the problem. In particular, we only need the set $\{U_{i}\}_{i}$ where $i = 1,\dots,n+1$. To see this, suppose a point $p\in \Pro^{n}$, then since we recall that $\Pro^{n} = \R^{n+1}\setminus\{0\}/\sim$ where $x \sim y \iff x = ky$ for $k\in \R\setminus \{0\}$, we can conclude that $[0]\notin \Pro^{n}$. Hence, $p$ must have atleast one non-zero element, suppose it is the $i^{th}$ element, then we know that $p\in U_{i}$. Hence all of $\Pro^{n}$ is covered by this open cover, and it is countable since it is finite.

\textbf{(v)} Finally suppose $p,q\in \Pro^{n}$ distinct. Then, without loss of generality, suppose that the $i^{th}$ element of $p$ is zero. Further assume it is non-zero for $q$, then we can see that $\exists j = 1,\dots,i-1,i+1,\dots,n+1$ such that this component of $p$ is non-zero, and hence $p\in U_{j}$ and $q\in U_{i}$ but $p\notin U_{i}$. Now assume that $p$ and $q$ are non-zero in all of the same components, then we have that necessarily $p,q\in U_{i}$ for all such non-zero components.

Hence we have that $\{U_{i},\varphi_{i}\}_{i}$ induces a smooth structure on $\Pro^{n}$.

Now we can look at the second set of maps and sets, $\{V_{i},\psi_{i}\}_{i}$.
\end{document}
