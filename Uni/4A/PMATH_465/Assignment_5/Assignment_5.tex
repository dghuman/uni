\documentclass[10pt]{article}
\usepackage[]{ragged2e}
\usepackage{fancyhdr,amsmath,amsthm,amssymb,bbm}
\usepackage[utf8]{inputenc}
\usepackage[letterpaper,left=25mm,right=25mm]{geometry}

\setlength{\parskip}{1em}
\setlength{\parindent}{0em}

\newcommand{\Z}{\mathbb{Z}}
\newcommand{\R}{\mathbb{R}}
\newcommand{\Q}{\mathbb{Q}}
\newcommand{\C}{\mathbb{C}}
\newcommand{\N}{\mathbb{N}}
\newcommand{\Sp}{\mathbb{S}}
\newcommand{\Pro}{\mathbb{P}}
\newcommand{\di}[2][]{\frac{\partial #1}{\partial #2}}
\newcommand{\del}[2][]{\frac{d #1}{d #2}}

\DeclareMathOperator{\Ima}{Im}

\linespread{1.25}
\pagestyle{fancy}
\fancyhf{}
\lhead{PMATH 465 $|$  Assignment 3}

\rhead{Dilraj Ghuman $|$ 20564228}

\begin{document}

\textbf{Question 1}

\textbf{(A) Problem 9.3 in Lee}

\textbf{(a)} We consider the vector field $V = y\di{x} + \di{y}$ on $\R^{2}$. Suppose $p = (p_{1},p_{2})\in\R^{2}$, and say we wish to find the flow $\varphi_{t}(p) = \gamma_{p}(t)$. We find the maximal integral curve in the standard way; suppose $\gamma_{p}(t) = (x(t),y(t))$ and $\gamma_{p}(0) = p$, then
\[ \gamma^{\prime}_{p}(t) = \gamma_{*}\left(\del{t}\bigg|_{p}\right) = \del[x]{t}\di{x} + \del[y]{t}\di{y} = y\di{x} + \di{y} = V  \implies \del[x]{t} = y \hspace{1em} \& \hspace{1em} \del[y]{t} = 1\]
\[ y(t) = t + c \hspace{2em} x(t) = \frac{1}{2}t^{2}+ct+d \]
but we have our initial condition, which gives
\[ \gamma_{p}(0) = (d,c) = (p_{1},p_{2}) \implies \varphi_{t}((p_{1},p_{2})) = (\frac{1}{2}t^{2}+p_{2}t+p_{1},t + p_{2}) \]
where $t\in \R$, and we see this is our flow.

\textbf{(b)} Let $W = x\di{x} + 2y\di{y}$, and we can let $\varphi_{t}(p) = \gamma_{p}(t)$, where $p = (p_{1},p_{2})\in \R^{2}$. Then, if $\gamma_{p}(0) = p$ and $\gamma_{p}(t) = (x(t),y(t))$, we get
\[ \gamma^{\prime}_{p}(t) = \gamma_{*}\left(\del{t}\bigg|_{p}\right) = \del[x]{t}\di{x} + \del[y]{t}\di{y} = x\di{x} + 2y\di{y} = W \implies \del[x]{t} = x \hspace{1em} \& \hspace{1em} \del[y]{t} = 2y\]
\[ x(t) = Ae^{t} \hspace{2em} y(t) = Be^{2t}. \]
Applying our initial condition,
\[ \gamma_{p}(0) = (A,B) = (p_{1},p_{2}) \implies A = p_{1} \hspace{1em} \& \hspace{1em} B = p_{2} \]
which then gives us that
\[ \varphi_{t}((p_{1},p_{2})) = (p_{1}e^{t},p_{2}e^{2t}) \]
where $t\in \R$, and hence we have our flow.

\textbf{(c)} We see that $X = x\di{x} - y\di{y}$, and we let $\varphi_{t}(p) = \gamma_{p}(t)$ for $p=(p_{1},p_{2})\in \R^{2}$, $\gamma_{p}(t) = (x(t),y(t))$ and $\gamma_{p}(0) = p$. Then,
\[ \gamma^{\prime}_{p}(t) = \gamma_{*}\left(\del{t}\bigg|_{p}\right) = \del[x]{t}\di{x} + \del[y]{t}\di{y} = x\di{x} - y\di{y} = X \implies \del[x]{t} = x \hspace{1em} \& \hspace{1em} \del[y]{t} = -y \]
\[ x(t) = Ae^{t} \hspace{2em} y(t) = Be^{-t}. \]
Applying our initial condition,
\[ \gamma_{p}(0) = (A,B) = (p_{1},p_{2}) \implies A = p_{1} \hspace{1em} \& \hspace{1em} B = p_{2} \]
so that
\[ \varphi_{t}((p_{1},p_{2})) = (p_{1}e^{t},p_{2}e^{-t}) \]
where $t\in \R$ and thus we have our flow.

\textbf{(d)} We let $Y = x\di{y} + y\di{x}$, and further let $\varphi_{t}(p) = \gamma_{p}(t)$, where $p=(p_{1},p_{2})\in \R^{2}$, $\gamma_{p}(t) = (x(t),y(t))$, and $\gamma_{p}(0) = p$, then
\[ \gamma^{\prime}_{p}(t) = \gamma_{*}\left(\del{t}\bigg|_{p}\right) = \del[x]{t}\di{x} + \del[y]{t}\di{y} = x\di{y} + +y\di{x} = Y \implies \del[x]{t} = y \hspace{1em} \& \hspace{1em} \del[y]{t} = x \]
which is a pair of coupled ODEs. Suppose the following set-up
\[ \vec{x} = 
\begin{bmatrix}
  x \\
  y \\
\end{bmatrix}
\hspace{1em}
A =
\begin{bmatrix}
  0 & 1 \\
  1 & 0 \\
\end{bmatrix}
\implies \del{t}\vec{x} = A\vec{x}
\]
so that our system of ODE's becomes a vector ODE. Suppose a solution of the form $e^{\lambda t}\vec{v}$ where $\vec{v}\in \R^{2}$, then
\[ \del{t}\vec{x} = A\vec{x} \implies \lambda e^{\lambda t}\vec{v} = Ae^{\lambda t}\vec{v} \implies  \lambda \vec{v}= A\vec{v} \]
hence, we need to find the eigenvectors and eigen values for our solution. Thus,
\[ \text{det}(A - \lambda \mathcal{I}) =
\begin{vmatrix}
  -\lambda & 1 \\
  1 & -\lambda \\
\end{vmatrix}
= \lambda^{2} - 1 \implies \lambda = \pm 1 
\]
Furthermore, plugging in the eigenvalues, we can find that the corresponding eigenvectors to be
\[ \vec{v_{1}} = 
\begin{bmatrix}
  1 \\
  1
\end{bmatrix}
\hspace{2em} \& \hspace{2em} \vec{v_{2}} =
\begin{bmatrix}
  -1 \\
  1 \\
\end{bmatrix}
\]
With this, we see that our solution becomes
\[ \gamma_{p}(t) = (Ae^{t} - Be^{-t}, Ae^{t} + Be^{-t}) \implies \gamma_{p}(0) = (p_{1},p_{2}) = (A-B,A+B) \]
\[ \implies A = \frac{p_{1}+p_{2}}{2} \hspace{2em} \& \hspace{2em} B = \frac{p_{2} - p_{1}}{2} \]
and hence the flow is
\[ \varphi_{t}(p) = \left(\frac{p_{1}+p_{2}}{2}e^{t} - \frac{p_{2} - p_{1}}{2}e^{-t}, \frac{p_{1}+p_{2}}{2}e^{t} + \frac{p_{2} - p_{1}}{2}e^{-t}\right) \hspace{1em} \forall t\in \R\]
as required.

\textbf{(B) Problem 9.18 from Lee} 

We compute the flows of the two vector fields using the same methodology we used in the previous question. For $X$ we have that
\[ \gamma^{\prime}_{p}(t) = \gamma_{*}\left(\del{t}\bigg|_{p}\right) = \del[x]{t}\di{x} + \del[y]{t}\di{y} = x\di{x} - y\di{y} = X  \implies \del[x]{t} = x \hspace{1em} \& \hspace{1em} \del[y]{t} = -y\]
which we know will give us
\[ x = Ae^{t} \hspace{2em} \& \hspace{2em} y = Be^{-t}. \]
If we suppose $p = (p_{1},p_{2})\in M$, we get,
\[ \gamma_{p}(0) = (p_{1},p_{2}) = (A,B) \implies A = p_{1} \hspace{2em} B = p_{2}\]
hence we have that
\[ \theta_{t}((p_{1},p_{2})) = (p_{1}e^{t},p_{2}e^{-t}) \]
as our flow for $X$. Next, we consider the flow for $Y$. Again, we have that
\[ \gamma^{\prime}_{p}(t) = \gamma_{*}\left(\del{t}\bigg|_{p}\right) = \del[x]{t}\di{x} + \del[y]{t}\di{y} = x\di{y} + y\di{x} = Y  \implies \del[y]{t} = x \hspace{1em} \& \hspace{1em} \del[x]{t} = y.\]
We recall from question $\textbf{(A)}$ that the solution to this will give us a flow of 
\[ \psi_{t}((p_{1},p_{2})) = \left(\frac{p_{1}+p_{2}}{2}e^{t} - \frac{p_{2} - p_{1}}{2}e^{-t}, \frac{p_{1}+p_{2}}{2}e^{t} + \frac{p_{2} - p_{1}}{2}e^{-t}\right) \hspace{1em} \forall t\in \R\]
for $Y$.

We first compute the composition of the maps. Thus, if $J,K \subset \R$ are open with $0\in J$ and $0\in K$, not defined explicitly just yet,  we can let $(s,t) \in J\times K$. Then
\[ \theta_{s}\circ \psi_{t} ((p_{1},p_{2})) = \theta_{s}\left(\frac{p_{1}+p_{2}}{2}e^{t} - \frac{p_{2} - p_{1}}{2}e^{-t}, \frac{p_{1}+p_{2}}{2}e^{t} + \frac{p_{2} - p_{1}}{2}e^{-t}\right)\]
\[= \left(\frac{p_{1}+p_{2}}{2}e^{t+s} - \frac{p_{2} - p_{1}}{2}e^{s-t}, \frac{p_{1}+p_{2}}{2}e^{t-s} + \frac{p_{2} - p_{1}}{2}e^{-s-t}\right)\]
\[ \psi_{t}\circ\theta_{s} ((p_{1},p_{2})) = \psi_{t}(p_{1}e^{s},p_{2}e^{-s}) = \left(\frac{p_{1}e^{s}+p_{2}e^{-s}}{2}e^{t} - \frac{p_{2}e^{-s} - p_{1}e^{s}}{2}e^{-t}, \frac{p_{1}e^{s}+p_{2}e^{-s}}{2}e^{t} + \frac{p_{2}e^{-s} - p_{1}e^{s}}{2}e^{-t}\right) \]

%finish me

\textbf{(C)} We have that $Z = \di{x} \in \mathcal{X}(\R^{2})$ and $V = y\di{x} + \di{y}$. Then, by definition of the Lie bracket,
\[ [V,Z] = (V(Z^{x}) - Z(V^{x}))\del{x} + (V(Z^{y}) - Z(V^{y}))\del{y} = (V(1) - Z(y))\del{x} + (V(0) - Z(1))\del{y}\]
\[ [V,Z] = (0 - 0)\del{x} + (0 - 0)\del{y} = 0 \]

as expected.

% add diagram

\newpage
\textbf{Question 2: Problem 9.5 from Lee} 

By supposition, we know that $M$ has a nowhere vanishing vector field. Call this vector field $X\in \mathcal{X}(M)$. Furthermore, since $M$ is compact, we know that $M$ is complete and that the maximal integral curves admitted by this vector field will be over all of $\R$. Thus, we get a global flow for our manifold $M$. I propose that the homotopic map will be naturally induced by the flow of this manifold.

To see this, consider $p\in M$, and suppose $\gamma_{p}(t) : I\subset \R^{n} \to M$ an integral curve , with dim$(M) = n$, $0\in I$, $\gamma_{p}(0) = p \in M$ and $X_{\gamma_{p}(t)} = \gamma_{p}^{\prime}(t)$. However, this integral curve clearly gives us our flow aswell; we simply associate $\varphi_{t}(p) = \gamma_{p}(t)$. Furthermore, since $M$ is compact, $I = \R$, and our flow is global. The homotopy comes from this flow. To see this, we fix a map using the flow, in particular, let $F_{1}: M \to M$ such that $F_{1}(p) = \varphi_{1}(p)$, and we see that neccessarily $id: M \to M$ is just $id(p) = p = \varphi_{0}(p)$. So, since $\varphi_{t}(p)$ is a \textit{smooth}, global flow, we have that it is also continuous, and we get
\[ \varphi_{0}(p) = id(p) \hspace{2em} \& \hspace{2em} \varphi_{1}(p) = F_{1}(p) \]
where we know $\varphi: \R \times M \to M$. Thus, we have that the identity is homotopic with $F_{1}$, where $F_{1}$ is smooth since the flow is smooth.

To finish off the proof, we need that the homotopy yields no stationary points. This follows from the vector field that originally admitted the global flow; the vector feild is non-vanishing. Suppose we did have a stationary point, say $q\in M$ under the homotopy, then we see
\[ \varphi_{0}(q) = \varphi_{1}(q) \implies \gamma_{q}(0) = \gamma_{q}(1) \implies \gamma_{q}^{\prime}(t) = 0 \implies X_{\gamma_{q}(t)} = 0 \]
which is a contradiction, since we assumed that $X$ is non-vanishing. Therefore we have no stationary points.

\newpage
\textbf{Problem 3}

\textbf{(a)} We get the smoothness of $D_{X}Y$ from the components. In particular, notice that since $Y^{i}$ is smooth $\forall i = 1,\dots,n$, and $X \in \mathcal{X}(M)$, we know compositions of smooth maps will be smooth as well, and hence $X(Y^{i}) \in \C^{\infty}(M)$. Then, by lemma, since the component functions are smooth, $D_{X}Y$ is smooth.

First, we need that the map is $\R$-bilinear. Suppose $X,Y,Z \in \mathcal{X}(M)$ and $a,b\in\R$, then
\[ D(aX,bY) = D_{aX}bY = \sum_{i=1}^{n}aX(bY^{i})\di{x_{i}} = ab\sum_{i=1}^{n}X(Y^{i})\di{x_{i}} = abD_{X}Y = abD(X,Y)\]
\[ D(X, Y + Z) = D_{X}(Y + Z) = \sum_{i=1}^{n}X((Y + Z)^{i})\di{x_{i}} = \sum_{i=1}^{n}X((Y+Z)(x_{i}))\di{x_{i}}\]
\[= \sum_{i=1}^{n}X(Y(x_{i})+Z(x_{i}))\di{x_{i}} = \sum_{i=1}^{n}(X(Y^{i}) + X(Z^{i}))\di{x_{i}}\]
\[= \sum_{i=1}^{n}X(Y^{i})\di{x_{i}} + \sum_{i=1}^{n}X(Z^{i})\di{x_{i}} = D_{X}Y + D_{X}Z = D(X,Y) + D(X,Z) \]
\[ D(X+Y,Z) = D_{X+Y}Z = \sum_{i=1}^{n}(X+Z)(Y^{i})\di{x_{i}} = \sum_{i=1}^{n}(X(Y^{i})\di{x_{i}} + Z(Y^{i})\di{x_{i}})\]
\[= D_{X}Y + D_{Z}Y = D(X,Y) + D(Z,Y) \]
So we see that we have $\R$-bilinearity in this map. Now we need that it is $C^{\infty}(M)$-linear in $X$. Let $f\in\C^{\infty}(M)$, then,
\[ D(fX,Y) = D_{fX}Y = \sum_{i=1}^{n}(fX)(Y^{i})\di{x_{i}} \]
however, we note that since $f$ is just a smooth function on $M$, and $\mathcal{X}(M)$ is a vector field closed under multiplication by $C^{\infty}(M)$ fuctions. To see this explicitly, we need this to act on a smooth function at a point; let $g\in C^{\infty}(M)$ and $p\in M$, then
\[ D(fX,Y)(f(p)) = \sum_{i=1}^{n}(fX)(Y^{i})(p)\di[g(p)]{x_{i}} = \sum_{i=1}^{n}(fX)_{p}(Y^{i})\di[g(p)]{x_{i}} \]
\[ = \sum_{i=1}^{n}(f(p)X(p))(Y_{i})\di[g(i)]{x_{i}} = f(p)\sum_{i=1}^{n}X_{p}(Y^{i})\di[g(p)]{x_{i}}\]
\[ = f(p)(D_{X}Y)(p) = (fD(X,Y))(p) \]
as required!

Now we check the Leibniz rule. We have,
\[ D(X,fY)(g(p)) = D_{X}(fY)(g(p)) = \sum_{i=1}^{n}X(fY)^{i}(p)\di[g(p)]{x_{i}} \]
\[ = \sum_{i=1}^{n}X_{p}(fY)(x_{i})\di[g(p)]{x_{i}} = \sum_{i=1}^{n}X_{p}(fY^{i})\di[g(p)]{x_{i}}\]
\[ = \sum_{i=1}^{n}(X_{p}(f)Y^{i}_{p} + f(p)X_{p}(Y^{i}))\di[g(p)]{x_{i}} = \sum_{i=1}^{n}X_{p}(f)Y^{i}_{p}\di[g(p)]{x_{i}} + \sum_{i=1}^{n}f(p)X_{p}(Y^{i})\di[g(p)]{x_{i}} \]
\[ = X_{p}(f)Y_{p}(g) + f(p)D_{X}(Y)(g(p)) \]
\[ \implies D(X,fY) = X(f)Y + fD(X,Y) \]
as required. Thus, we have that the directional derivative is indeed an affine connection.

Now we look at $D + \Gamma$. Since $\Gamma$ is a $C^{\infty}(\R^{n})$-bilinear map it is also $\R$-bilinear, and hence we have that the sum of two $\R$-bilinear maps better be $\R$-bilinear. Furthermore, since $\Gamma$ is $C^{\infty}(\R^{n}$-bilinear, it automatically satisfies the being $C^{\infty}(\R^{n})$-linear in $X$. Thus, we get that the sum will also be $C^{\infty}(\R^{n})$-linear. Then, all we need is that it also satisfies the Leibniz rule. We see
\[(D+\Gamma)(X,fY) = D(X,fY) + \Gamma(X,fY) = X(f)Y + fD(X,Y) + f\Gamma(X,Y) = X(f)Y + f(D+\Gamma)(X,Y)\]
as required. Thus, we see that this sum is indeed an affine connection.

To see that every affine connection is just the sum of the directional derivative, $D$ and some $C^{\infty}(\R^{n})$-bilinear map $\Gamma$, we note that we need to find $\Gamma$ such that this holds, since we already have $D$ and $\nabla$, our affine connection. Then, notice
\[ \nabla  = D + \Gamma \implies \Gamma = \nabla - D. \]
So, all we need is that $\nabla - D$ satisfies the conditions for it being like $\Gamma$. In particular, we need to check that it is $C^{\infty}(\R^{n})$-bilinear. Suppose $g,f\in C^{\infty}(\R^{n})$, and $X,Y\in \mathcal{X}(\R^{n})$, then

\end{document}
