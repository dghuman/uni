\documentclass[10pt]{article}
\usepackage[]{ragged2e}
\usepackage{fancyhdr,amsmath,amsthm,amssymb,bbm}
\usepackage[utf8]{inputenc}
\usepackage[letterpaper,left=25mm,right=25mm]{geometry}

\setlength{\parskip}{1em}
\setlength{\parindent}{0em}

\newcommand{\Z}{\mathbb{Z}}
\newcommand{\R}{\mathbb{R}}
\newcommand{\Q}{\mathbb{Q}}
\newcommand{\C}{\mathbb{C}}
\newcommand{\N}{\mathbb{N}}
\newcommand{\Sp}{\mathbb{S}}
\newcommand{\Pro}{\mathbb{P}}

\DeclareMathOperator{\Ima}{Im}

\linespread{1.25}
\pagestyle{fancy}
\fancyhf{}
\lhead{PMATH 465 $|$  Assignment 4}

\rhead{Dilraj Ghuman $|$ 20564228}

\begin{document}

\textbf{Question 1}

This question motivates the use of partitions of unity. So, we first build an open cover. We note that since $A$ is closed, we have for free that $M\setminus A$ is open. Now we need an open set around $A$. Since $Y$ is a smooth vector field on $A$, then it is smooth $\forall p \in A$, and in particular, must be smooth for some open neighbourhood $W_{p}$ with $p\in W_{p}$. This follows from the local coordinate definition. Say $(U,\psi)$ is a patch with $p\in U$ and coordinates $(x_{1},\dots,x_{n})$, then we know
\[ Y_{p} = Y(p) = \sum_{n=1}^{\infty}Y^{i}(p)\frac{\partial}{\partial x_{i}}|_{p} \]
where $Y^{i}$ is smooth. Hence, we see that $Y$ will be smooth over some open neighbourhood, which in this case is $U$, and hence we can let $W_{p} = U$. We let the new open set be the union of all such open neighbourhoods, $\cup_{p\in A}W_{p}$, and can suppose $\cup_{p\in A}W_{p}\subseteq U$, for arbitrary open set $U \supset A$, since if it is not, we can take the intersection.

Now, our open cover is $\{M\setminus A,\cup_{p\in A}W_{p}\}$, and hence $\exists$ a partition of unity subordinate to this cover, say $\{\varphi\}\cup \{\varphi_{p}\}_{p\in A}$ respectively. Further, define the extension of $Y$ to $W_{p}$ by $\hat{Y}$. Then, we let
\[ \tilde{Y} = \sum_{p\in A}\varphi_{p}\hat{Y} \]
where the operation between the maps is the standard product, since both $\hat{Y}$ and $\varphi_{2}$ take points $p\in M$. Now we need to show that this is a smooth vector field. Let $q\in A$, then clearly
\[ \tilde{Y}_{q} = \tilde{Y}(q) = \underbrace{\sum_{p\in A}\varphi_{p}(q)}_{1}\hat{Y}(q) = \hat{Y}(p) = Y_{p}\]
since $\text{supp}(\varphi)\subset M\setminus A$, so $\varphi(q) = 0 \implies \sum_{p\in A}\varphi_{p} = 1$. So, we have that $\tilde{Y}|_{A} = Y$ and is a smooth vector field since $Y$ is a smooth vector field.

Next, we suppose $q \in M\setminus \cup_{p\in A}W_{p}$. Then we have that $\sum_{p\in A}\varphi_{p} = 0$, since $\text{supp}(\varphi_{p})\subset W_{p}$. So,
\[ \tilde{Y}_{q} = \tilde{Y}(q) = \sum_{p\in A}\varphi_{p}(q)\hat{Y}(q) = 0 \]
which is to say the vector field vanishes outside of $U$. This is trivially a smooth vector field.

Finally, we wish to have $\tilde{Y}$ be smooth over all of $M$, so we need to consider the missing patch between the two closed sets we have already shown, namely we need to consider the open set $U \setminus A$. To see how this will be smooth, we need to rewrite our $\tilde{Y}$ in a slightly different, but equivalent way. Suppose $q\in U\setminus A$, then,
\[ \tilde{Y}_{q} = \tilde{Y}(q) = \sum_{p\in A}\varphi_{p}(q)\hat{Y}(q) = \sum_{p\in A}\varphi_{p}(q)\hat{Y}(q) + \varphi(q)\hat{Y}(q) - \varphi(q)\hat{Y}(q)\]
\[ = \left(\sum_{p\in A}\varphi_{p}(q) + \varphi(q)\right)\hat{Y}(q) - \varphi(q)\hat{Y}(q) = (1 - \varphi(q))\hat{Y}(q) \]
Further, since $\text{supp}(\varphi)\cap\left(\cup_{p\in A}\text{supp}(\varphi_{p})\right) \neq \emptyset$, since otherwise the partition of unity would vanish at some point and contradict the definition, and the partition of unity uses smooth functions, we must have that $1 - \varphi(q)$ must go smoothly between $0$ and $1$ while switching between $\text{supp}(\varphi)$ and $\cup_{p\in A}\text{supp}(\varphi_{p})$.

This actually makes the check much easier, since $q$ must be in the support of either $\varphi$ or $\cup_{p\in A}\varphi_{p}$. If it is in one exclusively, then we get the above cases. The only remaining case is that it is in both, since it can't be in neither. Then, we see that $(1-\varphi(q)) \in (0,1)\subset \R$ and hence
\[ \tilde{Y}_{q} = (1-\varphi(q))\hat{Y}_{q} \]
which is smooth since $\hat{Y}_{q}$ is smooth at this $q$, which is in some $W_{p}$.

Hence, we have shown that for some smooth vector field $Y$ defined over the closed set $A\subset M$, for any open $U\subset M$ such that $A\subset U$, $\exists \tilde{Y}$ such that $\tilde{Y}$ is a smooth vector field over $M$ and $\tilde{Y}|_{A} = Y$, as required.

Further, since a singleton point is also a closed set, we can extend this finding to individual points. In particular, $\forall p \in M$, and $X \in T_{p}M$, since $X$ is a smooth vector field at $p$, a closed set, by the above proof, we can conclude that indeed $\exists$ a smooth vector field, $Y$ over $M$, such that $Y|_{p} = X$.

\newpage
\textbf{Question 2}

First, suppose that $F$ is a constant map. Let $f\in C^{\infty}(N)$, $p\in M$, and $X \in T_{p}M$. Then, by definition we have that
\[ F_{*}\circ X\circ f (p) = X(f\circ F)(p) = X(\underbrace{f(F(p))}_{\text{constant}}) = 0 \]
since the derivation of a constant is zero. This must hold $\forall p\in M$, hence we have that the pushforward of a constant map is everywhere vanishing.

Now we consider the opposite implication. Suppose that $F_{*}$ is the trivial push forward and everything vanishes. We pick a coordinate representation; let $(U,\varphi)$ be a local patch with coordinates $\varphi = (x_{1},\dots,x_{n})$ and $p\in U$. Then, we have by definition of the pushforward,
\[ F_{*}\circ X\circ f(p) = X(f\circ F(p)) = \sum_{i=1}^{n}X^{i}(p)\frac{\partial f}{\partial x_{i}}(F(p)) \]
but we assumed that this is the everywhere vanishing pushforward, then we see
\[ \sum_{i=1}^{n}X^{i}(p)\frac{\partial f}{\partial x_{i}}(F(p)) = 0 \]
where we know that $X^{i}(p)$ is smooth. Thus, we see that this is saying that the partials are everywhere vanishing at $F(p)$, which implies that
\[ F(p) = C \]
for some constant $C$. This is true for the entire subset $U$, and to see that this is the same constant everywhere, we know that we can repeat the above steps with another patch, say $(V,\psi)$, and reach a similar conclusion with another constant, $C^{\prime}$. But, since $M$ is connected, we know that there exists a set of overlapping patches that will connect the two open sets $U$ and $V$, which would force $C = C^{\prime}$. Thus, we can conclude that
\[ F(p) = C \hspace{1em} \forall p\in M \]
as required.

\newpage
\textbf{Question 3}

\textbf{(a)} To show that this pushforward map is indeed an isomorphism, we will need to show that it is a bijection and preserves the vector space properties of the tangent space. From the linearity property of the pushforward, we get scalar multiplication and addition of vectors preserved for free, which is all we need.

Now we just need that this map is a bijection. To see injection, we just need to work through the definition. Suppose $X,Y\in T_{(p,q)}(M\times N)$, for $(p,q)\in M\times N$, and $f\in C^{\infty}(M)$, $g\in C^{\infty}(N)$, such that
\[ (\pi_{1*},\pi_{2*})\circ X \circ (f,g) (p,q) = (\pi_{1*},\pi_{2*})\circ Y \circ (f,g) (p,q) \]
Then, by definition,
\[ (\pi_{1*},\pi_{2*})\circ X \circ (f,g) (p,q) = (\pi_{1*}X, \pi_{2*}X)\circ (f,g)(p,q) = (\pi_{1*}X(f), \pi_{2*}X(g))(p,q) \]
\[ (\pi_{1*},\pi_{2*})\circ X \circ (f,g) (p,q) = (X(f\circ \pi_{1}(p,q)), X(g\circ \pi_{2}(p,q))) = (X(f(p)),X(g(q)))\]
Through a similar argument, we get
\[  (\pi_{1*},\pi_{2*})\circ Y \circ (f,g) (p,q) = (Y(f(p)),Y(g(q))) \]
and then by assumption we get,
\[ Y(f(p)) = X(f(p)) \hspace{2em} Y(g(q)) = X(g(q)) \]
which must hold $\forall p\in M, q\in N, f\in C^{\infty}(M)$ and $g\in C^{\infty}(N)$. In particular, if we let $(U,\varphi)$ and $(V,\psi)$ be local patches of $M$ and $N$ respectively, with $p\in U$, $q\in V$ and coordinates $\varphi = (x_{1},\dots,x_{m})$, $\psi = (y_{1},\dots,y_{n})$, then this result says that
\[ \sum_{i=1}^{m}Y^{i}(p) \frac{\partial f}{\partial x_{i}}(p) = \sum_{j=1}^{m}X^{j}(p) \frac{\partial f}{\partial x_{j}}(p) \hspace{1em} \implies \hspace{1em} Y^{i}(p) = X^{i}(p)\]
and a similar argument in $V$ says that $Y^{i}(q) = X^{i}(q)$ $\forall p\in M,n\in N$. Thus, $Y=X$ as expected.

To see surjection, we notice the fact that the tangent spaces form vector spaces over $\R$. Further, we know that the dimension of the two spaces is equivalent, so if we have injection, we must have surjection. Thus, we have a bijection that preserves the vector space operations, and hence we have an isomorphism.

\textbf{(b)} Suppose $X\in T_{p}M$ and $Y\in T_{q}N$ for $p\in M$ and $q\in N$. Then, by the surjection, $\exists Z\in T_{(p,q)}(M\times N)$ such that $(\pi_{1*},\pi_{2*})(Z) = (X,Y)\in T_{p}M\times T_{q}N$. Further, by lemma $\exists \gamma: I\subset \R \to M\times N$ such that $\gamma(t_{0}) = (p,q)$ and $\gamma^{\prime}(t_{0}) = Z$. Hence, suppose $f \in C^{\infty}(M)$ and $g\in C^{\infty}(N)$, then
\[(\pi_{1*},\pi_{2*})\circ \gamma^{\prime}(t_{0})(f,g) = (\pi_{1*},\pi_{2*})\circ Z(f,g)(p,q) = (X(f(p)),Y(g(q))) \]
\[ \implies (\pi_{1*},\pi_{2*})\circ \gamma^{\prime}(t_{0}) = (X|_{p},Y|_{q}) \]
as required.

\newpage
\textbf{Question 4}

\newpage
\textbf{Question 5}

\newpage
\textbf{Question 6}

\textbf{(a)} Let $f\in C^{\infty}(M)$. Then, from the definition of $F$-related, we get that, if $p\in M$,
\[ Z_{F(p)} = F_{*,p}(Y_{p})\]
where from definition,
\[ Z_{F(p)}(f) = Z(F(p))(f) = Z(f)\circ F(p) \hspace{2em} F_{*,p}(Y_{p})(f) = F_{*}(Y(p))(f) = Y(f\circ F(p))\]
\[\implies Z(f)\circ F(p) = Y(f\circ F(p)) \hspace{2em} \forall p\in M\]
and since it holds for all $p$, we can conclude
\[ Z(f)\circ F = Y(f\circ F)\]
as required.

\textbf{(b)} Suppose $\mathcal{A}_{M}, \mathcal{A}_{N}$ smooth atlases for $M$ and $N$ respectively. Then, we know that $\mathcal{A}_{M}\times \mathcal{A}_{N}$ will be a smooth atlas for $M\times N$. In particular, let $(U,\varphi)\in \mathcal{A}_{M}$, $(V,\psi)\in \mathcal{A}_{N}$, $p\in U$, $q\in V$ with $U = (x_{1},\dots,x_{m})$ and $V = (y_{1},\dots,y_{n})$ local representations where $m = \text{dim}(M)$ and $n = \text{dim}(N)$. Then, $(U\times V,\varphi\times\psi)\in \mathcal{A}_{M}\times\mathcal{A}_{N}$, $(p,q)\in (U\times V)\subset M\times N$, with local representation $\varphi\times \psi = (x_{1},\dots,x_{m},y_{1},\dots,y_{n})$.

To build a smooth vector field on $M\times N$ with $Y$, we use the $\pi_{1}$-related condition, where $f\in C^{\infty}(M)$, then we want that
\[ Y(f) \circ \pi_{1} = \tilde{Y}(f\circ \pi_{1}). \]
Looking at each part in local coordinates,
\[ Y(f)\circ \pi_{1}(p,q) = Y(f(p)) = \sum_{i=1}^{m}Y^{i}(p)\frac{\partial f(p)}{\partial x_{i}} \hspace{2em} \tilde{Y}(f\circ \pi_{1})(p,q) = \tilde{Y}_{(p,q)}(f(p)) = \sum_{i=1}^{m}\tilde{Y}^{i}((p,q))\frac{\partial f(p)}{\partial x_{i}} + \sum_{j=1}^{n}\tilde{Y}^{j+m}((p,q))\frac{\partial f(p)}{\partial y_{j}}\]
We notice that in order for $\tilde{Y}$ to be $\pi_{1}$-related to $Y$, we must have that, for $i\in [1,m+n]\subset \Z$
\[ \tilde{Y}^{i}((p,q)) = Y^{i}(p) \hspace{1em} 0 < i \leq m \hspace{2em}\& \hspace{2em} \tilde{Y}^{i}((p,q)) = 0 \hspace{1em}i > m \]
$\forall (p,q)\in M\times N$. We know that since $Y\in \mathcal{X}(M)$, $Y^{i}\in C^{\infty}(M)$ and we can extend this to a smooth function on $M\times N$ by making the new components zero. This is exactly what $\tilde{Y}$ is by construction; the components that do not match with $Y^{i}$ vanish, and the resulting vector field is still smooth $\forall (p,q)\in U\times V$. 

Suppose $\exists X\in \mathcal{X}(M\times N)$ such that $X$ is $\pi_{1}$-related to $Y$. Then, from the same argumena bove, we get that for $i\in \Z$,
\[ X^{i}(p,q) = Y^{i}(p) = \tilde{Y}^{i}(p,q) \hspace{1em} 0 < i \leq m \hspace{2em} \&\hspace{2em} X^{i}(p,q) = 0 = \tilde{Y}^{i}(p,q) \hspace{1em} m < i \leq m+n\]
$\forall (p,q)\in U\times V$. So, $\tilde{Y} = X$, and hence $\tilde{Y}$ is unique.

Thus, we have constructed a smooth vector field by looking component-wise at $Y$, that is unique, and by definition of being $\pi_{1}$-related by construction, we have that $Y = \pi_{1*}(\tilde{Y})$, as required.

The exact same steps can be taken for $Z$ but the first $m$ components would vanish in $\tilde{Z}^{i}$ instead. So, we can conclude the same result for $Z$ and $\tilde{Z}$.

\newpage
\textbf{Question 7}

\end{document}
