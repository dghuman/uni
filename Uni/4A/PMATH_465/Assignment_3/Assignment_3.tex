\documentclass[10pt]{article}
\usepackage[]{ragged2e}
\usepackage{fancyhdr,amsmath,amsthm,amssymb,bbm}
\usepackage[utf8]{inputenc}
\usepackage[letterpaper,left=25mm,right=25mm]{geometry}

\setlength{\parskip}{1em}
\setlength{\parindent}{0em}

\newcommand{\Z}{\mathbb{Z}}
\newcommand{\R}{\mathbb{R}}
\newcommand{\Q}{\mathbb{Q}}
\newcommand{\C}{\mathbb{C}}
\newcommand{\N}{\mathbb{N}}
\newcommand{\Sp}{\mathbb{S}}
\newcommand{\Pro}{\mathbb{P}}

\DeclareMathOperator{\Ima}{Im}

\linespread{1.25}
\pagestyle{fancy}
\fancyhf{}
\lhead{PMATH 465 $|$  Assignment 3}

\rhead{Dilraj Ghuman $|$ 20564228}

\begin{document}

\textbf{Question 1}

\textbf{(a)} To see this, consider the open cover $\{M\setminus A,M \setminus B\}$. Then by theorem, we have a partition of unity subordinate to this cover denoted by $\{\varphi_{1}:M\setminus A \to \R\}\cup\{\varphi_{2}:M\setminus B\to \R\}$. In particular, note that we can let
$$f = \varphi_{1}$$
and such an $f$ will satisfy our conditions. To see this, first note that $supp(f) = supp(\varphi_{1})\subset M\setminus A$, thus $f(q) = 0$ $\forall q \in A$, as we want. Next, note that since this is a partition of unity, $0 \leq f(x) \leq 1$ $\forall x \in M$ by definition. Finally, notice that $\forall p\in B$,
$$1 = \sum_{i=1}^{2}\varphi_{i}(p) = \varphi_{1}(p) + \varphi_{2}(p) = \varphi_{1}(p) = f(p)$$
since $supp(\varphi_{2}) \subset M\setminus B$, and is smooth by theorem (there always exists smooth partition of unity). Thus, such an $f$ satisfies our function.

\textbf{(b)} We need to construct a function that is only 1 on the closed set $B\subset M$ and is only 0 on the closed set $A\subset M$ where $A\cap B = \emptyset$ so that the inverse is well defined. To do this, we first need to construct an open cover that seperates our three sets of interest, namely $A, B$ and the rest of $M$.

First we build open sets around $A$ and $B$ such that the intersection of the two is empty. We use the Hausdorff Property of the smooth manifold to do so. First, we can build an open set around $B$ by taking the union of open neighbourhoods around each point, $W_{p}$ where $p\in W_{p}$. Similarly, we can do the same for $A$ with open sets $V_{q}$ where $q\in V_{q}$. But, by the Hausdorff property of manifolds, we can see that if $V_{q}\cap W_{p} \neq \emptyset$, then $\exists \tilde{V}_{q},\tilde{W}_{p}$ such that $\tilde{V}_{q}\cap\tilde{W}_{p} = \emptyset$. Thus,
$$U_{A} = \cup_{q\in A}\tilde{V}_{q} \hspace{2em} U_{B} = \cup_{p \in B}\tilde{W}_{p} \hspace{2em} U_{A}\cap U_{B} = \emptyset$$

Now, we can consider the open cover $\{U_{A},U_{B},M\setminus\{A\cup B\}\}$, and by the existence of partitions of unity, we can suppose such a partition with maps $\{\varphi_{A},\varphi_{B},\psi\}$. Now, we need a function such that it is $1$ only on $B$ and $0$ only on $A$. To build this, we use the Level Sets of Smooth Functions theorem.

By this theorem, for $A$ and $B$ $\exists  f_{A},f_{B}\in C^{\infty}(M)$ respectively, such that $f_{A}^{-1}(0) = A$ and $f_{B}^{-1}(0) = B$. Thus, we can use the function
$$f = \frac{1}{2} + \frac{f_{A}}{2(f_{A} + \phi_{A})} - \frac{f_{B}}{2(f_{B} + \phi_{B})}.$$
First, we note this function is smooth, and is bounded between $0$ and $1$. Next, notice that if $p\in A$, then $f(p) = 0$, if $p \in B$ then $f(p) = 1$.

\textbf{(c)} We construct our partition of unity $\{\varphi_{1}: U \to \R,\varphi_{2}: M\setminus A \to \R\}$ subordinate to the open cover $\{U,M\setminus A\}$, where $M\setminus A$ is open since $A$ is closed. Then, we see that we can simply let $f = \varphi_{1}$.

First, $\forall p \in A$, we see that
$$1 = \sum_{i=1}^{2}\varphi_{i}(p) = \varphi_{1}(p) + \varphi_{2}(p) = \varphi_{1}(p) = f(p)$$
since $supp(\varphi_{2}) \subset M\setminus A$. Next, we see that since $\varphi_{1}$ is a partition of unity,
$$0 \leq f(x) = \varphi(1)(x) \leq 1 \hspace{1em} \forall x \in M$$
by definition. Finally, $supp(f) = supp(\varphi_{1}) \subset U$ as required.

Thus $f$ satisfies our conditions.

\textbf{(d)} 

\newpage

\textbf{Question 2}

\textbf{(a)} We have already shown in class that $\R^{n}$ is indeed a smooth manifold. Further, vectors under addition in $\R^{n}$ trivially form a group, as it follows from component addition since the vector space is defined over a field. All we need then is that the inverse and addition form a smooth map.

First we consider the group law, $f: \R^{n} \times \R^{n} \to \R^{n}$ which we must show is smooth. We note that $\R^{n}\times \R^{n}$ is a smooth manifold as well, since finite products of smooth manifolds are smooth manifolds where the underlying smooth structure is induced by the smooth structure of each smooth manifold. Then, we see that
$$f(v,w) = v+w \hspace{1em} \forall v,w\in \R^{n}$$
is the explicit map for the group law. Further, we know that a smooth atlas of $\R^{n}$ is just the identity map $\{id: \R^{n} \to \R^{n}\}$, and similarly for $\R^{n}\times\R^{n}$, since $\R^{n}\times\R^{n} \simeq \R^{2n}$. Then,
$$id\circ f \circ id^{-1}: \R^{n}\times\R^{n} \simeq \R^{2n} \to \R^{n}$$
and in particular, $\forall v,w\in \R^{n}$,
$$id\circ f\circ id^{-1}(v,w) = id(f(v,w)) = id(v + w) = v+w$$
which is smooth.

Next we look at the inverse map, $\iota: \R^{n} \to \R^{n}$, where $\iota(v) = -v$ $\forall v\in \R^{n}$. Then, again we use the identity map as our patch map,
$$id\circ \iota \circ id^{-1}(v) = id(\iota(v)) = id(-v) = -v \hspace{1em} \forall v\in \R^{n}$$
which is smooth.

Thus, we have that $\R^{n}$ is a Lie group.

\textbf{(b)} First we note that $\R^{*}$ is indeed a group under multiplication. This follows from the component multiplication being multiplication in $\R$, which is associative, has identity, and has multiplicative inverses. Further the product is closed in the group with the zero vector removed. 

We note that the set $\{0\}$ is closed, hence the complement, $\R^{*}$ is open in $\R$, and hence the ambient standard topology of $\R$ equips $\R^{*}$ with a smooth atlas, and hence makes $\R^{*}$ a smooth manifold.

Next we show that the group law, $f: \R^{*}\times \R^{*} \to \R^{*}$ where $f(a,b) = ab$ for $a,b\in \R{*}$ is a smooth map. We know that $\R^{*}\times \R^{*}$ is a smooth manifold, and we use the identity map as our surface patch. Then,
$$id \circ f \circ id^{-1}: \R^{*}\times\R^{*} \to \R^{*}$$
where for $a,b\in \R^{*}$,
$$id \circ f \circ id^{-1}(a,b) = id(f(a,b)) = id(ab) = ab$$
is smooth.

Now we consider the inverse map $\iota : \R^{*} \to \R^{*}$ where $\iota(a) = \frac{1}{a}$ for $a\in \R^{*}$. Then we see that
$$id \circ \iota id^{-1}: \R^{*} \to \R^{*}$$
where for $a \in \R^{*}$,
$$id \circ \iota \circ id^{-1}(a) = id(f(a)) = id\left(\frac{1}{a}\right) = \frac{1}{a}$$
which is smooth since $0 \notin \R^{*}$.

Hence we have that $\R^{*}$ is a Lie Group.

\textbf{(c)} We already know that $S^{1}$ is a smooth manifold since $\C$ can be associated with $\R^{2}$. Further, we get that the product manifold is also smooth, and $S^{1}$ is a group under multiplication, since we just get rotations on the circle in $\C$, inverses exist under angle addition, the identity is an angle of 0, or multiplication by 1, and associativity follows from real addition.

First we need that the group operation is a smooth map. To see this, we consider the smooth atlas $\mathcal{A} = \{\varphi,U\}\cup\{\psi,V\}$ where
$$\varphi: U = S^{1}\setminus \{1\} \to (0,2\pi)\subset \R  \hspace{2em} \varphi(z = e^{i\theta}) = \theta \in \varphi(U)$$
$$\psi: V = S^{1}\setminus \{-1\} \to (-\pi,\pi)\subset \R  \hspace{2em} \psi(z = e^{i\theta}) = \theta \in \psi(V)$$

We already know this is smooth, and further $$\mathcal{B} = \mathcal{A}\times \mathcal{A}$$ will also induce a smooth structure on $S^{1}\times S^{1}$ since the atlas will use the smooth compatability of each component.

Then, consider the following map, where
$$f: S^{1}\times S^{1} \to S^{1} \hspace{2em} f(w = e^{i\theta},z = e^{i\phi}) = e^{i\theta}e^{i\phi} = e^{i(\theta+\phi)}\in S^{1}$$
is the group operation, and
$$\varphi \circ f \circ (\varphi\times\psi)^{-1}: \varphi(U)\times \psi(V) \subset \R\times\R \to \R$$
is one of the maps we need to show is smooth. In particular, we note that $\varphi,\psi$ are both smooth maps already, so, for $\theta\in (0,2\pi)\subset \R$ and $\phi \in (-\pi,\pi)\subset \R$
$$\varphi \circ f \circ (\varphi\times\psi)^{-1} (\theta,\phi) = \theta + \phi \in \R$$

which is smooth. This will hold for all other combinations, so we have that the group product is smooth.

Now we consider the inverse map, $\iota: S^{1} \to S^{1}$, where if $z\in S^{1}$, we get
$$\iota(z = e^{i\theta}) = e^{i(-\theta)} = e^{-i\theta}$$
which is clearly smooth, and since $\varphi,\psi$ are smooth with smooth inverses, we get that the composition will also be smooth. Thus the inverse map is also smooth.

Hence, since the group product and inverse map are both smooth, we get that $S^{1}\subset \C$ does form a Lie group under multiplication.

\textbf{(d)} First, we note that $GL(n,\R)$ is indeed a group under multiplication. In particular, we know that the group operation is closed, since this is simply the set of all $n\times n$ matrices over the real field with non-zero determinant, and the product of such matrices will result in a matrix also of determinant non-zero. Further, since the determinant is non-zero, we have the identity matrix, inverses for all elements, and associativity follows from the definition of group multiplication.

We now need that $GL(n,\R)$ forms a smooth manifold. To see this, we use the fact that $GL(n,\R) \subset M_{n\times n} \simeq \R^{n^{2}}$, and in particular, we know that this is the complement to the set of all $n\times n$ singular matrices, which is closed in $\R^{n^{2}}$ since it is the image of the smooth (and hence continuous) map of the determinant which will have the primage of closed sets be closed. Hence we have that we can get a single patch of $\R^{n^{2}}$ which uses the ambient topology, and hence is smooth. So, we can take the association,
$$GL(n,\R)\ni A \to (a_{1,1}, a_{1,2}, \dots, a_{1,n}, a_{2,1}, \dots, a_{n,n})\in \R^{n^{2}}$$
which is a bijection onto it's image. With this association, we can show that the product map is smooth and so is the inverse.

From now on, we let $det(a) = det(A)$ where $A\in GL(n,\R)$, $a \in \R^{n^{2}}$ and $A$ and $a$ are associated. 

In particular, we know that the product of two smooth manifolds will be a smooth manifold under the product topology of the two individual manifolds. In particular we know that the identity will be the map over the patch $U\subset \R^{n^{2}}$ where $\forall x\in U$ $det(x) \neq 0$. Then, we see that $id_{\R^{n^{2}}}: \R^{n^{2}} \to \R^{n^{2}}$ will be our map, and if we let
$$f: GL(n,\R)\times GL(n,\R) \subset \R^{n^{2}}\times \R^{n^{2}} \to GL(n,\R)\subset \R^{n^{2}}, f: (x,y)\mapsto xy\in GL(n,\R), \forall x,y\in U$$
then we want the following map to be smooth,
$$ id \circ f \circ id_{\R^{n^{2}}\times\R^{n^{2}}}: U\times U \to U \subset \R^{n^{2}}.$$
Take $x,y \in U$, then,
$$id \circ f \circ id_{\R^{n^{2}}\times\R^{n^{2}}}(x,y) = id \circ f(x,y) = id(xy) = xy$$
where the product is matrix multiplication. Each element of the resulting matrix will be a polynomial in the components of $x$ and $y$ by definition, and hence will be smooth. In particular (just to be sure), if $x = (x_{1,1}\dots,x_{i,j},\dots,x_{n,n})$ and $y = (y_{1,1},\dots,y_{i,j},\dots,y_{n,n})$, then
$$xy = \left(\sum_{m=1}^{n}x_{1,m}y_{m,1}, \dots, \sum_{m=1}^{n}x_{i,n}y_{j,n}, \dots, \sum_{m=1}^{n}x_{n,m}y_{m,n}\right)$$
which is smooth, so our product map is aswell.

Now we look at the inverse map. In particular, if $x\in U$ and $\iota: U\to U$ is the inverse map, then consider
$$id\circ \iota \circ id(x) = id(\iota(x))$$
but since $U$ is the set of invertable matrices, $\exists x^{-1}\in U$ and hence,
$$id(\iota(x)) = id(x^{-1}) = x^{-1}.$$
So all we have to check is that the inverse is smooth. We note that each component of the inverse is a polynomial in the components of $x$, but is divided by $det(x)$, so we have rational polynomials, which will still be smooth. Thus the inverse is smooth!

Hence, we can conclude that the general linear group over the reals, $GL(n,\R)$ is indeed a Lie group.

\newpage

\textbf{Question 3}

\textbf{(a)} Suppose $f\in C^{\infty}(N)$, $a,b \in \R$ and $X,Y\in T_{p}M$ for $p\in M$. Then,
$$F_{*}(aX + bY) = (aX + bY)(f\circ F) = aX(f\circ F) + bY(f\circ F) = aF_{*}(X)(f) + bF_{*}(Y)(f)$$
and hence $F_{*}$ is linear.

\textbf{(b)} Suppose that $f\in C^{\infty}(P)$, $X\in T_{p}M$. Then,
$$(G\circ F)_{*}(X)(f) = X(f\circ (G\circ F)) = X((f\circ G)\circ F) = F_{*}(X(f\circ G)) = G_{*}\circ F_{*}(X)(f)$$
as required.

\textbf{(c)} Suppose that $f\in C^{\infty}(M)$ and $X\in T_{p}M$. Then,
$$(Id_{M})_{*}(X)(f) = X(f\circ Id_{M}) = X(f) \in T_{p}M$$
as expected.

\textbf{(d)} By property \textbf{(a)} we have that $F_{*}$ is linear and hence we know that scalar multiplication and addition of derivations is preserved under the map $F_{*}$. Next, we need that $F_{*}$ is a bijection. Suppose $X,Y\in T_{p}M$ and $f\in C^{\infty}(N)$. 

First, suppose that $F_{*}(X)(f) = F_{*}(Y)(f)$, then,
$$F_{*}(X)(f) = X(f\circ F) \hspace{2em} F_{*}(Y)(f) = Y(f\circ F)$$
$$\implies X(f\circ F) = Y(f\circ F)$$
however, since $F$ is a bijection (by diffeomorphism), it is in particular injective and hence $\exists q \in N$ unique such that $F(p) = q$ and hence 
$$X(f(q)) = Y(f(q))\implies X = Y \hspace{1em}\forall p\in M$$
thus we can conclude that $F_{*}$ is injective.

Next, suppose $Z\in T_{q}N$ for some $q\in N$, but $F$ is surjective so $\exists p \in M$ such that $F(p) = q$. Then,
$$T_{q}N \ni Z(f(p)) = Z(f\circ F(p)) = F_{*}(Z)(f) \implies Z \in T_{p}M$$
hence we have that $F_{*}$ is surjective.

Thus, $F_{*}$ is a bijection that preserves the vector addition and scalar multiplication (via linearity). Thus $F_{*}$ is an isomorphism.

\newpage

\textbf{Question 4}

Suppose $N$ is an $n$-dimensional smooth manifold and $M$ is an $m$-dimensional smooth manifold. Further, suppose $F: N\to M$ is a diffeomorphism. 

Suppose that $p\in M$. Then, since $F$ is a diffeomorphism, it is a bijection, hence $\exists q\in N$ such that $F(q) = p$ and $F^{-1}(p) = q$. Hence, by definition of smooth manifold, $\exists \varphi: U \to \R^{n}$ where $q\in U$, $U$ open and $\varphi$ is a diffeomorphism.

We know that diffeomorphisms send open sets to open sets, since a diffeomorphism is a homeomorphism, hence we see that $p\in F^(U)$ is an open set in $M$. Then, we see that $p$ has an open neighbourhood that is diffeomorphic to $\R^{n}$ through the map
$$\varphi \circ F^{-1}: F(U)\subset M \to \R^{n}$$
This is true $\forall p\in M$, hence every point in $M$ has an open neighbourhood diffeomorphic to $\R^{n}$, and thus $M$ is of diminsion $n$.

Notice how a similar argument can be made $\forall q\in N$, and hence $N$ is of dimension $m$. Thus clearly we see that $m=n$.
\end{document}
