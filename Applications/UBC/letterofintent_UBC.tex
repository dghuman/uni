\documentclass[11pt]{article}
\usepackage[document]{ragged2e}
\usepackage[]{geometry}
\usepackage{fancyhdr,amsmath,amsthm,amssymb}
\usepackage[utf8]{inputenc}

\geometry{letterpaper,left=25mm,right=25mm}

\newcommand{\R}{\mathbb{R}}
\newcommand{\Z}{\mathbb{Z}}
\newcommand{\Q}{\mathbb{Q}}
\newcommand{\N}{\mathbb{N}}

\linespread{1.25}
\setlength{\parskip}{1em}
%\pagestyle{fancy}
%\fancyhf{}
%\lhead{Course ID: Assignment N}
%\rhead{Dilraj Ghuman $|$ 20564228}
%\cfoot{\thepage}
\begin{document}
Dear Graduate Coordinator,
My name Dilraj Ghuamn and I am applying for a Master's in physics to UBC. 
\section{Research}
I have had two major research positions over my undergraduate career. I worked as a USRA at SNOLab under Chris Jillings, employed by Laurentian, with the DEAP collaboration where I worked mainly on characterizing Photomultiplier Tubes (PMTs) for the experiment. The second research position was at TRIUMF under Mark Scott and Thomas Lindner in the T2K group working on the construction of the Multi-PMT prototype for the NuPrism extension of T2K.

My USRA at SNOLab with DEAP was over the Fall term of 2017, from September to December. The Dark Matter Experiment using Argon Pulseshape discrimination (DEAP) is a direct detection experiment for Weakly Interacting Massive Particles (WIMPs), which is a Dark Matter candidate. The experiment itself is 2 kilometers underground in Sudbury, Ontario and is a spherical acrylic vessel capable of carrying 3600 kg of liquid argon, the target for detection. During my time at SNOLab I worked under the supervision of Chris Jillings with whom I worked characterizing PMTs used in the DEAP experiment. In particular, I worked on 
\end{document}
