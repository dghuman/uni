\documentclass[11pt]{article}
\usepackage[document]{ragged2e}
\usepackage[]{geometry}
\usepackage{fancyhdr,amsmath,amsthm,amssymb}
\usepackage[utf8]{inputenc}

\geometry{letterpaper,left=25mm,right=25mm}

\newcommand{\R}{\mathbb{R}}
\newcommand{\Z}{\mathbb{Z}}
\newcommand{\Q}{\mathbb{Q}}
\newcommand{\N}{\mathbb{N}}

\linespread{1.25}
\setlength{\parskip}{1em}
%\pagestyle{fancy}
%\fancyhf{}
%\lhead{Course ID: Assignment N}
%\rhead{Dilraj Ghuman $|$ 20564228}
%\cfoot{\thepage}
\begin{document}
Dear Graduate Coordinator,
My name Dilraj Ghuman and I am applying for a Master's in physics to UBC. 
\section{Research}
I have had two major research positions over my undergraduate career. I worked as a USRA at SNOLab under Chris Jillings, employed by Laurentian, with the DEAP collaboration where I worked mainly on characterizing Photomultiplier Tubes (PMTs) for the experiment. The second research position was at TRIUMF under Mark Scott and Thomas Lindner in the T2K group working on the construction of the Multi-PMT prototype for the NuPrism extension of T2K.

My USRA at SNOLab with DEAP was over the Fall term of 2017, from September to December. The Dark Matter Experiment using Argon Pulseshape discrimination (DEAP) is a direct detection experiment for Weakly Interacting Massive Particles (WIMPs), which is a Dark Matter candidate. The experiment itself is 2 kilometers underground in Sudbury, Ontario and is a spherical acrylic vessel capable of carrying 3600 kg of liquid argon, the target for detection. During my time at SNOLab I worked under the supervision of Chris Jillings with whom I worked characterizing PMTs used in the DEAP experiment, and checking on Dark Rates.

My first task was to move the software used in PMT data monitering to a new cluster, where it was then updated to work properly. I then began in monitering the effeciency of PMTs through their collected data, and marking any underperforming or misbehaving PMTs. This task was important to the analysis of the data as it warned the analysts of data that should be considered incorrect. My final task was working on Dark Rate analysis for the PMTs and comparing with the results of another part of the group.

Many of the weekly results I obtained from PMT data monitering were presented at the weekly group meetings, and I presented my work on Dark Rates to parts of the collaboration in Europe as well.

My second relevant research experience was at TRIUMF working in T2K. T2K is a long baseline experiment based in Japan whose primary goal is to study neutrino osscilations. The experiment sends neutrino's produced in Tokai to a water cherenkov tank in Kamioka (hence T2K), and uses the detected neutrino flavour, which is based off of the neutrino's interaction via the Weak Force. The main goal of the term was to assist in the construction of the Multi-PMT (mPMT) for the NuPrism extension, which is meant to provide an intermediate measurement of the neutrino beam and reduce errors in calculations.

Though a large priority of my work at TRIUMF was in the contruction of the mPMT prototype via PMT testing and gelling, the time not spend doing hardware was spent on tagging events in simulation software for the NuPrism detector with the mPMTs. In particular, my goal was to tag pile-up events in the detector. My first attempt was with a likelihood measure in which the number of PMTs hit and the time of hit was recorded to fit a loglikelihood distribution and provide an estimate on the number of possible neutrino's that produced the hits. This worked for large time delays between the neutrinos, but ultimately could not distinguish finer differences. My second attempt was with a Hough Transform, where the goal was ultimately to fit rings on the detectors surface to count the number of possible neutrino's.

At the end of the term, I presented my work to my local group and at the TRIUMF Student Symposium, along with the weekly meetings.

\section{Statemtn of Academic Goals}
My research career up until now has been greatly focused on the experimental aspect of physics. This has given me a great appreciation for the difficulty of this area of research, but acedemically I have always found myself tending towards the more theoretical aspects with a focus on the underlying mathematics. This has been especially clear with the recent courses I have taken, am taking, and my past research experiences. My greatest research interest lies in Geometry, and in particular, it's application in physics.

The study of geometry itself is quite large and has it's roots in Pure Mathematics, where it comes to fruition. However, geometry holds great power in the area of theoretical physics and is evident in the study of black holes, QFT and string theory. These topics are all linked by a single goal, and that is the understanding of gravity; the area of research on which I would like to focus on.



\end{document}
