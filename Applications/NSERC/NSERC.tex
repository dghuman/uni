\documentclass[12pt]{article}
\usepackage[document]{ragged2e}
\usepackage[]{geometry}
\usepackage{fancyhdr,amsmath,amsthm,amssymb,fontspec}
%\usepackage[utf8]{inputenc}

\geometry{letterpaper,left=25mm,right=25mm}

\newcommand{\R}{\mathbb{R}}
\newcommand{\Z}{\mathbb{Z}}
\newcommand{\Q}{\mathbb{Q}}
\newcommand{\N}{\mathbb{N}}

\linespread{1.00}
\setlength{\parskip}{1em}
\setmainfont{Times New Roman}
\setlength{\parindent}{1cm}
%\pagestyle{fancy}
%\fancyhf{}
%\lhead{Course ID: Assignment N}
%\rhead{Dilraj Ghuman $|$ 20564228}
%\cfoot{\thepage}
\begin{document}

The proposed research for my Master's is to extend the work done in describing the dynamics of Dirac particles in a Kerr-Newman black hole geometry \cite{atmp}. It seems natural to be able to extend this behaviour of Dirac particles near a black hole's event horizon to higher dimensional spaces, and still expect similar results. In particular, in this black hole geometry, we see no rise of fixed orbits for dirac particles near the black hole event horizon, and instead expect either the particle to fall into the event horizon or to drift away \cite{cpam}.

A black hole is a an extremely dense body, so dense in fact that even light cannot escape from the interior of the mass \cite{gr}. A spherically symmetric mass is called a black hole if the radius of the mass is equal to the \textit{Shwarzschild Radius}, which comes from Schwarzschild's solution to Einstein's field equations, which diverge at such a radius \cite{gr}. This Schwarzschild space-time gives rise to a \textit{singularity}, the centre of the black hole, and an \textit{event horizon}, the boundry beyond which nothing escapes \cite{mbh}.

The Shwarzschild solution requires a spherically symmetric mass and is stationary. On the other hand, the Kerr solutions provides a broader family of solutions, wherein the family of solutions are dependent upon only the mass of the black hole and it's angular momentum to exactly describe the stellar object \cite{mbh}. This is an incredibly powerful metric, as it allows for relaxation of the spherical symmetry forced in the Shwarzschild solution, but further allows the exploration of the geometry surrounding all physically realizeable black holes that arise via gravitational collapse of stellar masses \cite{mbh}. This motivates studying the behaviour of particles in such a geometry.

A Dirac particle is a particle that can be modeled quantum mechanically by the \textit{Dirac Equation}. The Dirac Equation was Dirac's method of successfully introducing relativity to quantum mechanics \cite{dirac}, and gives rise to the notion of anti-particles. As such, many of the particles in the standard model can be classified as Dirac particles, with the most well known one being the electron, as it is the particle that Dirac used to introduce relativity \cite{dirac}. Thus, to study the long time dynamics of Dirac particles near the event horizon of a black hole in a Kerr space-time can be done through this geometrical approach, and represents true physics in the interstellar medium.

It has already been shown that time-periodic behaviour, such as closed orbits, for a dirac particle are not possible in this Kerr-Newmann geometry \cite{cpam}, and further the long-time dynamics conclude such a behaviour in the dirac particle near a Kerr-Newmann black hole \cite{atmp}. One should expect that such behaviour is not limited to the lower dimensional case, and should extend naturally to a higher dimensional space. 


\newpage
\begin{thebibliography}{9}
\bibitem{cpam}
  F. Finster, N. Kamran, J. Smoller, and S.-T. Yau, \textit{Non-existence of time-periodic solutions of the Dirac equation in an axisymmetric black hole geometry}, \texttt{gr-qc/9905047}, Comm. Pure Appl. Math. 53 (2000) 902-92.

\bibitem{atmp}
  F. Finster, N. Kamran, J. Smoller, S.-T. Yau, \textit{The Long-Time Dynamics Dirac Particles in the Kerr-Newman Black Hole Geometry}, \texttt{gr-qc/0005088}, Adv. Theor. Math. Phys. 7 (2003) 25-52.

\bibitem{mbh}
  S. Chandrasekhar. \textit{The Mathematical Theory of Blackholes}. Revised reprint of the 1983 original. International Series of Monographs on Physics ,69. Oxford Science Publications. The Clarendon Press, Oxford University Press, NewYork, 1992.	

\bibitem{gr}
  J. D. Nightingale, J. Foster. \textit{A Short Course in General Relativity}. Springer, New York, NY. Springer-Verlag New York 2006.

\bibitem{dirac}
  Dirac, P.A.M. \textit{Principles of Quantum Mechanics}. Internatial Series of Monographs on Physics (4th ed.). Oxford University Press (1982) [1958].
  
\end{thebibliography}

\end{document}
